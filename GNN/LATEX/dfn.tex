% !TEX root = ms.tex
\def\rvx{{\mathbf{x}}}


\newcommand{\la}[1]{{\color{red} LA: #1}}
\newcommand{\lz}[1]{{\color{purple} LZ: #1}}
\newcommand{\ns}[1]{{\color{blue} NS: #1}}

% tuples
\newcommand{\tps}[1]{\overrightarrow{\boldsymbol{#1}}}
% multisets
\newcommand{\mms}[1]{\{\mskip-5mu\{#1\}\mskip-5mu\}}

\newcommand{\bbmms}[1]{\big\{\mskip-10mu \big\{  #1 \big\} \mskip-10mu \big\}}
\newcommand{\bmms}[1]{\Big\{\mskip-10mu \Big\{  #1 \Big\} \mskip-10mu \Big\}}
\newcommand{\Bmms}[1]{\bigg\{\mskip-10mu \bigg\{  #1 \bigg\} \mskip-10mu \bigg\}}
\newcommand{\ms}[1]{\tilde{\boldsymbol{#1}}}
% sets 
\newcommand{\s}[1]{\hat{\boldsymbol{#1}}}

\newcommand{\cbit}{\begin{compactitem}}
\newcommand{\ceit}{\end{compactitem}}
\newcommand{\cben}{\begin{compactenum}}
\newcommand{\ceen}{\end{compactenum}}
\newcommand{\todo}[1]{{\textsf{\textcolor{blue}{TODO: #1}}}}

\newtheorem{observation}{Observation}
\newtheorem{conjecture}{Conjecture}
\newtheorem{problem}{Problem}
\newtheorem{algo}{Algorithm}

%\theoremstyle{definition}
\newtheorem{definition}{Definition}[section]

\newtheorem{lemma}{Lemma}
\newtheorem{theorem}{Theorem}[section]
\newtheorem{question}{Question}
\newtheorem{example}{Example}
\newtheorem{answer}{Answer}
%\newtheorem{proof}{Proof}
\newtheorem{insight}{Insight}
%\newcommand{\hide}[1]{}
% \newcommand{\notice}[1]{{\textsf{\textcolor{green}{{\em [#1]}}}}}
\newcommand{\notice}{\todo[inline]}
% \newcommand{\notice}{\todo}
\newcommand{\reminder}[1]{{\textsf{\textcolor{blue}{#1}}}}
%
\newcommand{\vectornorm}[1]{\left|\left|#1\right|\right|}

\newcommand{\cmark}{\ding{51}}%

\newcommand{\amethod}{{\sc $(\leq)$}\xspace}

\newcommand{\method}{{\sc $(k,c)(\leq)$-SetGNN}\xspace}
\newcommand{\methods}{{\sc $(k,c)(\leq)$-SetGNN$^*$}\xspace}

\newcommand{\JJnorm}{{\sc $\textrm{JJ-Norm}$}\xspace}
\newcommand{\PMP}{{$\textrm{PMP}$}\xspace}
\newcommand{\MMP}{{$\textrm{MMP}$}\xspace}
\newcommand{\PNY}{{$\textrm{PNY}$}\xspace}
\newcommand{\IMPaCT}{{$\textrm{IMPaCT}$}\xspace}

\newcommand{\balpha}{\bm{\alpha}}%
\newcommand{\blambda}{\bm{\lambda}}%
\newcommand{\btheta}{\bm{\theta}}%

\newcommand\scalemath[2]{\scalebox{#1}{\mbox{\ensuremath{\displaystyle #2}}}}

%\makeatletter
%\newcommand\footnoteref[1]{\protected@xdef\@thefnmark{\ref{#1}}\@footnotemark}
%\makeatother

    % \setlength{\marginparwidth}{0in}
    % \setlength{\marginparsep}{0in}
    % \newgeometry{margin=1in}



\newcommand{\mkclean}{
    \renewcommand{\reminder}{\hide}
    \renewcommand{\notice}{\hide}
    % \renewcommand{\todo}{\hide}
    % \usepackage[top=1in, bottom=1in, outer=0in, inner=0in, heightrounded, margin
    % parwidth=0in, marginparsep=0in]{geometry}
    % \usepackage[letterpaper]{geometry}
    % \PassOptionsToPackage{letterpaper}{geometry}
    % \setlength{\top}{1in}
    % \setlength{\bottom}{1in}
    % \setlength{\outer}{1in}
    % \setlength{\inner}{1in}
    % \newgeometry{top=1in, bottom=1in, outer=1in, inner=1in}
}


\makeatletter
\newcommand\footnoteref[1]{\protected@xdef\@thefnmark{\ref{#1}}\@footnotemark}
\makeatother

%\renewcommand{\algorithmicrequire}{\textbf{Input:}}
%\renewcommand{\algorithmicensure}{\textbf{Output:}}
%\renewcommand{\algorithmiccomment}[1]{\hfill$\blacktriangleright$ #1}

\newcommand{\cdash}{\multicolumn{1}{c}{--} }



% from cf: shorthands - also they make tighter lists
%     needs package paralist
% \newcommand{\cbit}{\begin{compactitem}}
% \newcommand{\ceit}{\end{compactitem}}
% \newcommand{\cben}{\begin{compactenum}}
% \newcommand{\ceen}{\end{compactenum}}

\newcommand{\beq}{\begin{equation}}
	\newcommand{\eeq}{\end{equation}}


\newcommand{\bit}{\begin{itemize}}
	\newcommand{\eit}{\end{itemize}}
\newcommand{\ben}{\begin{enumerate}}
	\newcommand{\een}{\end{enumerate}}

\newcommand{\QED}{ \hfill {\bf QED}}


\newcommand{\scale}{{Scalability\xspace}}
\newcommand{\effective}{{Effectiveness\xspace}}


\newcommand{\ny}{{Nystr\"{o}m}\xspace}


\newcommand{\kreg}{{$k$-regular\xspace}}

\newcounter{x}\setcounter{x}{1}
\newtheorem{inneraxiom}{{Axiom}}
\newenvironment{axiom}[1]
  {\renewcommand\theinneraxiom{\arabic{x} (#1)}\inneraxiom\stepcounter{x}}
  {\endinneraxiom}

% \newtheorem{lemma}{Lemma}

% symbols and notations used
\newcommand{\pv}{PV}
\newcommand{\outflag}{O} % outlier flag
\newcommand{\lbl}{Y}
\newcommand{\samples}{n}
\newcommand{\dimension}{m}
\newcommand{\xobs}{\bs{x}}


\newcommand{\udr}{{\sc UDR}\xspace}
\newcommand{\mc}{{\sc MC}\xspace}
\newcommand{\hits}{{\sc HITS}\xspace}
\newcommand{\hitse}{{\sc HITS-ens}\xspace}


\newcommand{\ndcg}{{\sc NDCG}\xspace}

\newcommand{\pk}{{\sc PK}\xspace}
\newcommand{\wl}{{\sc WL}\xspace}
\newcommand{\gtovec}{{\sc g2vec}\xspace}


\newcommand{\MMD}{{\sc MMD}\xspace}

\newcommand{\mC}{\mathcal{C}}
\newcommand{\mG}{\mathcal{G}}
\newcommand{\mS}{\mathcal{S}}
\newcommand{\mH}{\mathcal{H}}
\newcommand{\mP}{\mathcal{P}}
\newcommand{\mN}{\mathcal{N}}
\newcommand{\mB}{\mathcal{B}}
\newcommand{\mW}{\mathcal{W}}
\newcommand{\mV}{\mathcal{V}}
\newcommand{\mE}{\mathcal{E}}

\newcommand{\calK}{\mathcal{K}}
\newcommand{\calX}{\mathcal{X}}


\newcommand{\bW}{\mathbf{W}}
\newcommand{\bK}{\mathbf{K}}
\newcommand{\bH}{\mathbf{H}}
\newcommand{\bU}{\mathbf{U}}
\newcommand{\bS}{\mathbf{\Sigma}}
\newcommand{\bV}{\mathbf{V}}
\newcommand{\bD}{\bm{\Lambda}}


\newcommand{\bx}{\mathbf{x}}
\newcommand{\by}{\mathbf{y}}
\newcommand{\bh}{\mathbf{h}}
\newcommand{\bhp}{\mathbf{h}^\prime}



\newcommand{\R}{\mathbb{R}}
\newcommand{\E}{\mathbb{E}}


\newcommand{\bq}{\mathbb{Q}}
\newcommand{\bp}{\mathbb{P}}
\newcommand{\bpi}{\mathbb{P}_i}
\newcommand{\bpip}{\mathbb{P}_{j}}
\newcommand{\hbpi}{\widehat{\mathbb{P}}_i}
\newcommand{\hbpip}{\widehat{\mathbb{P}}_{j}}

%types of methods
\newcommand{\prem}{{preprocessing}\xspace}
\newcommand{\inm}{{in-processing}\xspace}



%colors
\newcommand{\red}[1]{\colorbox{red!50}{#1}}
\newcommand{\lred}[1]{\colorbox{red!20}{#1}}
\newcommand{\blue}[1]{\colorbox{blue!40}{#1}}
\newcommand{\green}[1]{\colorbox{green!40}{#1}}

\newcommand{\bs}[1]{\boldsymbol{#1}}

% creating danger symbol -- for wall clock plot to indicate methods do not follow axioms
\newcommand*{\TakeFourierOrnament}[1]{{%
\fontencoding{U}\fontfamily{futs}\selectfont\char#1}}
\newcommand*{\danger}{\TakeFourierOrnament{66}}

\newcommand{\xmark}{\ding{55}}%

\newcommand{\numdatasets}{{4\xspace}}

% comment command for neil and leman
\newcommand{\leman}[1]{{\textsf{\textcolor{red}{#1}}}}


\newcommand{\neil}[1]{{\textsf{\textcolor{black!20!orange!90!}{#1}}}}



