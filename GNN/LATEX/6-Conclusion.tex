In this study, we tackle the domain adaptation challenge in graph data induced by chronological splits by proposing invariant message passing functions, \IMPaCT. We effectively analyzed and addressed the domain adaptation problem in graph datasets with chronological splits, by presenting robust and realistic assumptions derived from observable properties in real-world graphs. Based on the assumptions we propose the \IMPaCT, to preserve the invariance of the 1st moment and 2nd moment of aggregated message during the message passing step, and demonstrate its adaptability and scalability. We validate the efficacy of our approach both on real-world citation graph and synthetic graph. Particularly on the citation graph ogbn-mag, we achieved a substantial improvement over the previous state-of-the-art, with a significant margin of 5.4\% when applying only 1st moment alignment method, and 6.1\% when combining 1st and 2nd moment alignment method.

\textbf{Limitations.} Our theoretical discussions were restricted to Spatial GNNs and assume that semantic aggregation satisfies the G-Lipschitz condition. Additionally, \IMPaCT does not guarantee that the invariant representations obtained are maximally informative. There is also a need to demonstrate the robustness and generalizability of \IMPaCT through experiments on a wider variety of baseline models. Lastly, the interpretation of experimental results for 2nd moment alignment methods was insufficient. Notably, when SGC was the baseline, 2nd order alignment methods saw increased performance gains with larger decay factors, whereas the opposite trend was observed when GCN was the baseline. These limitations should be addressed in future research.

% Our research fills a critical gap in effectively analyzing and addressing the domain adapatation problem in massive graph datasets, an area previously underexplored. 


