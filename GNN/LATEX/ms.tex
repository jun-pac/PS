\pdfoutput=1
\documentclass{article}
\PassOptionsToPackage{numbers, sort&compress}{natbib}
\usepackage[preprint]{neurips_2024}
% \usepackage[final]{neurips_2024}

\usepackage[linesnumbered,ruled]{algorithm2e}

\usepackage[utf8]{inputenc} % allow utf-8 input
\usepackage[T1]{fontenc}    % use 8-bit T1 fonts
\usepackage{url}            % simple URL typesetting
\usepackage{booktabs}       % professional-quality tables
\usepackage{amsfonts}       % blackboard math symbols
\usepackage{nicefrac}       % compact symbols for 1/2, etc.
\usepackage{xcolor}         % colors
\usepackage{graphicx}
\usepackage{bbm}
\usepackage{paralist}
\usepackage{xspace}
\usepackage{bm}
\usepackage{multirow}
\usepackage{amsmath}
\usepackage{amsthm}
\usepackage{amssymb}
\usepackage{subcaption}
\usepackage{makecell}
\usepackage{dsfont}

% \usepackage{algorithm}     % algorithm envir
% \usepackage{algpseudocode} % algorithm envir
% \renewcommand{\algorithmicrequire}{\textbf{Input:}}
% \renewcommand{\algorithmicensure}{\textbf{Output:}}

\usepackage{amsthm}
\usepackage{hyperref} 
\definecolor{mydarkblue}{rgb}{0,0.1,0.6}
\hypersetup{
    colorlinks=true,
    citecolor=mydarkblue
          }
          
% !TEX root = ms.tex
\def\rvx{{\mathbf{x}}}


\newcommand{\la}[1]{{\color{red} LA: #1}}
\newcommand{\lz}[1]{{\color{purple} LZ: #1}}
\newcommand{\ns}[1]{{\color{blue} NS: #1}}

% tuples
\newcommand{\tps}[1]{\overrightarrow{\boldsymbol{#1}}}
% multisets
\newcommand{\mms}[1]{\{\mskip-5mu\{#1\}\mskip-5mu\}}

\newcommand{\bbmms}[1]{\big\{\mskip-10mu \big\{  #1 \big\} \mskip-10mu \big\}}
\newcommand{\bmms}[1]{\Big\{\mskip-10mu \Big\{  #1 \Big\} \mskip-10mu \Big\}}
\newcommand{\Bmms}[1]{\bigg\{\mskip-10mu \bigg\{  #1 \bigg\} \mskip-10mu \bigg\}}
\newcommand{\ms}[1]{\tilde{\boldsymbol{#1}}}
% sets 
\newcommand{\s}[1]{\hat{\boldsymbol{#1}}}

\newcommand{\cbit}{\begin{compactitem}}
\newcommand{\ceit}{\end{compactitem}}
\newcommand{\cben}{\begin{compactenum}}
\newcommand{\ceen}{\end{compactenum}}
\newcommand{\todo}[1]{{\textsf{\textcolor{blue}{TODO: #1}}}}

\newtheorem{observation}{Observation}
\newtheorem{conjecture}{Conjecture}
\newtheorem{problem}{Problem}
\newtheorem{algo}{Algorithm}

%\theoremstyle{definition}
\newtheorem{definition}{Definition}[section]

\newtheorem{lemma}{Lemma}
\newtheorem{theorem}{Theorem}[section]
\newtheorem{question}{Question}
\newtheorem{example}{Example}
\newtheorem{answer}{Answer}
%\newtheorem{proof}{Proof}
\newtheorem{insight}{Insight}
%\newcommand{\hide}[1]{}
% \newcommand{\notice}[1]{{\textsf{\textcolor{green}{{\em [#1]}}}}}
\newcommand{\notice}{\todo[inline]}
% \newcommand{\notice}{\todo}
\newcommand{\reminder}[1]{{\textsf{\textcolor{blue}{#1}}}}
%
\newcommand{\vectornorm}[1]{\left|\left|#1\right|\right|}

\newcommand{\cmark}{\ding{51}}%

\newcommand{\amethod}{{\sc $(\leq)$}\xspace}

\newcommand{\method}{{\sc $(k,c)(\leq)$-SetGNN}\xspace}
\newcommand{\methods}{{\sc $(k,c)(\leq)$-SetGNN$^*$}\xspace}

\newcommand{\JJnorm}{{\sc $\textrm{JJ-Norm}$}\xspace}
\newcommand{\PMP}{{$\textrm{PMP}$}\xspace}
\newcommand{\MMP}{{$\textrm{MMP}$}\xspace}
\newcommand{\PNY}{{$\textrm{PNY}$}\xspace}
\newcommand{\IMPaCT}{{$\textrm{IMPaCT}$}\xspace}

\newcommand{\balpha}{\bm{\alpha}}%
\newcommand{\blambda}{\bm{\lambda}}%
\newcommand{\btheta}{\bm{\theta}}%

\newcommand\scalemath[2]{\scalebox{#1}{\mbox{\ensuremath{\displaystyle #2}}}}

%\makeatletter
%\newcommand\footnoteref[1]{\protected@xdef\@thefnmark{\ref{#1}}\@footnotemark}
%\makeatother

    % \setlength{\marginparwidth}{0in}
    % \setlength{\marginparsep}{0in}
    % \newgeometry{margin=1in}



\newcommand{\mkclean}{
    \renewcommand{\reminder}{\hide}
    \renewcommand{\notice}{\hide}
    % \renewcommand{\todo}{\hide}
    % \usepackage[top=1in, bottom=1in, outer=0in, inner=0in, heightrounded, margin
    % parwidth=0in, marginparsep=0in]{geometry}
    % \usepackage[letterpaper]{geometry}
    % \PassOptionsToPackage{letterpaper}{geometry}
    % \setlength{\top}{1in}
    % \setlength{\bottom}{1in}
    % \setlength{\outer}{1in}
    % \setlength{\inner}{1in}
    % \newgeometry{top=1in, bottom=1in, outer=1in, inner=1in}
}


\makeatletter
\newcommand\footnoteref[1]{\protected@xdef\@thefnmark{\ref{#1}}\@footnotemark}
\makeatother

%\renewcommand{\algorithmicrequire}{\textbf{Input:}}
%\renewcommand{\algorithmicensure}{\textbf{Output:}}
%\renewcommand{\algorithmiccomment}[1]{\hfill$\blacktriangleright$ #1}

\newcommand{\cdash}{\multicolumn{1}{c}{--} }



% from cf: shorthands - also they make tighter lists
%     needs package paralist
% \newcommand{\cbit}{\begin{compactitem}}
% \newcommand{\ceit}{\end{compactitem}}
% \newcommand{\cben}{\begin{compactenum}}
% \newcommand{\ceen}{\end{compactenum}}

\newcommand{\beq}{\begin{equation}}
	\newcommand{\eeq}{\end{equation}}


\newcommand{\bit}{\begin{itemize}}
	\newcommand{\eit}{\end{itemize}}
\newcommand{\ben}{\begin{enumerate}}
	\newcommand{\een}{\end{enumerate}}

\newcommand{\QED}{ \hfill {\bf QED}}


\newcommand{\scale}{{Scalability\xspace}}
\newcommand{\effective}{{Effectiveness\xspace}}


\newcommand{\ny}{{Nystr\"{o}m}\xspace}


\newcommand{\kreg}{{$k$-regular\xspace}}

\newcounter{x}\setcounter{x}{1}
\newtheorem{inneraxiom}{{Axiom}}
\newenvironment{axiom}[1]
  {\renewcommand\theinneraxiom{\arabic{x} (#1)}\inneraxiom\stepcounter{x}}
  {\endinneraxiom}

% \newtheorem{lemma}{Lemma}

% symbols and notations used
\newcommand{\pv}{PV}
\newcommand{\outflag}{O} % outlier flag
\newcommand{\lbl}{Y}
\newcommand{\samples}{n}
\newcommand{\dimension}{m}
\newcommand{\xobs}{\bs{x}}


\newcommand{\udr}{{\sc UDR}\xspace}
\newcommand{\mc}{{\sc MC}\xspace}
\newcommand{\hits}{{\sc HITS}\xspace}
\newcommand{\hitse}{{\sc HITS-ens}\xspace}


\newcommand{\ndcg}{{\sc NDCG}\xspace}

\newcommand{\pk}{{\sc PK}\xspace}
\newcommand{\wl}{{\sc WL}\xspace}
\newcommand{\gtovec}{{\sc g2vec}\xspace}


\newcommand{\MMD}{{\sc MMD}\xspace}

\newcommand{\mC}{\mathcal{C}}
\newcommand{\mG}{\mathcal{G}}
\newcommand{\mS}{\mathcal{S}}
\newcommand{\mH}{\mathcal{H}}
\newcommand{\mP}{\mathcal{P}}
\newcommand{\mN}{\mathcal{N}}
\newcommand{\mB}{\mathcal{B}}
\newcommand{\mW}{\mathcal{W}}
\newcommand{\mV}{\mathcal{V}}
\newcommand{\mE}{\mathcal{E}}

\newcommand{\calK}{\mathcal{K}}
\newcommand{\calX}{\mathcal{X}}


\newcommand{\bW}{\mathbf{W}}
\newcommand{\bK}{\mathbf{K}}
\newcommand{\bH}{\mathbf{H}}
\newcommand{\bU}{\mathbf{U}}
\newcommand{\bS}{\mathbf{\Sigma}}
\newcommand{\bV}{\mathbf{V}}
\newcommand{\bD}{\bm{\Lambda}}


\newcommand{\bx}{\mathbf{x}}
\newcommand{\by}{\mathbf{y}}
\newcommand{\bh}{\mathbf{h}}
\newcommand{\bhp}{\mathbf{h}^\prime}



\newcommand{\R}{\mathbb{R}}
\newcommand{\E}{\mathbb{E}}


\newcommand{\bq}{\mathbb{Q}}
\newcommand{\bp}{\mathbb{P}}
\newcommand{\bpi}{\mathbb{P}_i}
\newcommand{\bpip}{\mathbb{P}_{j}}
\newcommand{\hbpi}{\widehat{\mathbb{P}}_i}
\newcommand{\hbpip}{\widehat{\mathbb{P}}_{j}}

%types of methods
\newcommand{\prem}{{preprocessing}\xspace}
\newcommand{\inm}{{in-processing}\xspace}



%colors
\newcommand{\red}[1]{\colorbox{red!50}{#1}}
\newcommand{\lred}[1]{\colorbox{red!20}{#1}}
\newcommand{\blue}[1]{\colorbox{blue!40}{#1}}
\newcommand{\green}[1]{\colorbox{green!40}{#1}}

\newcommand{\bs}[1]{\boldsymbol{#1}}

% creating danger symbol -- for wall clock plot to indicate methods do not follow axioms
\newcommand*{\TakeFourierOrnament}[1]{{%
\fontencoding{U}\fontfamily{futs}\selectfont\char#1}}
\newcommand*{\danger}{\TakeFourierOrnament{66}}

\newcommand{\xmark}{\ding{55}}%

\newcommand{\numdatasets}{{4\xspace}}

% comment command for neil and leman
\newcommand{\leman}[1]{{\textsf{\textcolor{red}{#1}}}}


\newcommand{\neil}[1]{{\textsf{\textcolor{black!20!orange!90!}{#1}}}}





\newcommand{\revise}[1]{{\color{blue}#1}}
\usepackage{wrapfig}

% if you need to pass options to natbib, use, e.g.:
%     \PassOptionsToPackage{numbers, compress}{natbib}
% before loading neurips_2022




% \usepackage{neurips_2022}

% to compile a preprint version, e.g., for submission to arXiv, add add the
% [preprint] option:
    % \usepackage[preprint]{neurips_2022}


% to compile a camera-ready version, add the [final] option, e.g.:
%     \usepackage[final]{neurips_2022}


% to avoid loading the natbib package, add option nonatbib:
%    \usepackage[nonatbib]{neurips_2022}


    % hyperlinks
\usepackage{url}            % simple URL typesetting
\usepackage{xcolor}
\urlstyle{rm}

\setlength{\belowcaptionskip}{0pt}
\setlength{\abovedisplayskip}{0pt}
\setlength{\belowdisplayskip}{0pt}
\setlength{\abovedisplayshortskip}{0pt}
\setlength{\belowdisplayshortskip}{0pt}

\usepackage{titlesec}
\titlespacing*{\subsection}{0pt}{1pt}{1pt}
\titlespacing*{\section}{0pt}{2pt}{2pt}
\newtheorem{corollary}{Corollary}[theorem]


\newtheorem{innercustomthm}{Theorem}
\newenvironment{customthm}[1]
  {\renewcommand\theinnercustomthm{#1}\innercustomthm}
  {\endinnercustomthm}

\title{IMPaCT GNN : Invariant Message Passing for Domain Adaptation in Chronological Split Graphs}


% The \author macro works with any number of authors. There are two commands
% used to separate the names and addresses of multiple authors: \And and \AND.
%
% Using \And between authors leaves it to LaTeX to determine where to break the
% lines. Using \AND forces a line break at that point. So, if LaTeX puts 3 of 4
% authors names on the first line, and the last on the second line, try using
% \AND instead of \And before the third author name.


\author{%
  Sejun Park\\
  Seoul National University\\
  \texttt{skg4078@snu.ac.kr} \\
  % examples of more authors
  \And
 Jooyoung Park\\
 Seoul National University\\
 \texttt{jyp531@snu.ac.kr}\\
  \And
  Hyunwoo Park\\
  Seoul National University\\
  \texttt{hyunwoopark@snu.ac.kr} \\
}


\begin{document}


\maketitle


\begin{abstract}
This study addresses domain adaptation challenges in graph data arising from chronological splits. We propose invariant message passing methods to attain invariant representations, showcasing superiority over existing methods both theoretically and in performance on real datasets. Semi-supervised Node Classification (SSNC) tasks, often involving nodes with temporal information, particularly on chronologically split datasets, lack systematic research on leveraging temporal information effectively. Disparities in node connections based on temporal information create domain shifts, presenting an domain adaptation problem. Domain adaptation in transductive graph learning is crucial yet challenging, especially when extending from historical to recent environments. We rigorously apply the invariant principle within decoupled Graph Neural Networks (GNNs) and introduce robust assumptions reflecting temporal graph characteristics to derive effective environment sets. We propose Invariant Message Passing for Chronological-split Temporal graph(\IMPaCT) methods preserving the invariance of representation moments and analyze their ability to approximate invariant information. Experimental results on real and synthetic datasets, notably achieving a 6.1\% improvement on the ogbn-mag citation graph dataset, demonstrate the effectiveness of our methods. Additionally, introducing the Temporal Stochastic Block Model (TSBM) enables realistic replication of temporal graphs with different environments, further showcasing the applicability of our methods to general spatial GNNs, yielding significant performance gains.
\end{abstract}


\section{Introduction}
The task of Semi-supervised Node Classification (SSNC) often involves nodes with temporal information. For instance, in academic graph representing connections between papers, authors, and institutions, each paper node may contain information regarding the year of its publication. The focus of this study lies within such graph data, particularly on datasets where the train and test splits are arranged in chronological order. In other words, the separation between nodes available for training and those targeted for inference occurs temporally, requiring the classification of the labels of nodes with the most recent temporal information based on the labels of nodes with historical temporal information. While leveraging GNNs trained on historical data to classify newly added nodes is a common scenario in industrial and research settings \cite{liu2016predicting,bai2020temporal,pareja2020evolvegcn}, systematic research on effectively utilizing temporal information within chronological split graphs remains scarce.

\begin{figure}[hbt!]
	\vspace{-0.15in}
	\centering
	\includegraphics[width=0.90\textwidth]{figs/Chronological_split_PNG.png}
	\vspace{-0.1in}
	\caption{Illustrative explanation of chronological split dataset.}
	 \label{fig:chronological_split}
	 \vspace{-0.3in}
\end{figure}
\vspace{1in}
%Failure to appropriately utilize temporal information can lead to significant performance degradation when the model attempts to classify labels of recent data. The distribution of neighboring nodes�� classes may not remain constant over time, and the relative connectivity frequency of nodes also depends on temporal information. We conducted a toy experiment on the ogbn-mag dataset, an academic graph dataset having features with chronological split, to confirm the existence of such distribution shifts. The specific settings of this experiment can be found in Appendix \ref{apdx:toy_experiment}. Table \ref{table:toy} presents results of the toy experiment.
Failure to appropriately utilize temporal information can lead to significant performance degradation when the model attempts to classify labels of recent data. We conducted a toy experiment on the ogbn-mag dataset, an academic graph dataset having features with chronological split, to confirm the existence of such distribution shifts. The specific settings of this experiment can be found in Appendix \ref{apdx:toy_experiment}. Table \ref{table:toy} presents results of the toy experiment.


% \begin{figure}[hbt!]
% 	\vspace{-0.15in}
% 	\centering
% 	\includegraphics[width=0.99\textwidth]{figs/toy_experiment.png}
% 	\vspace{-0.1in}
% 	\caption{[Left] Neighboring node aggregation in chronological split, [Right] in random split.}
% 	 \label{fig:toy_exp}
% 	 \vspace{-0.3in}
% \end{figure}


% The performance degradation caused by chronological splits can be clearly observed by comparing the maximum accuracy achieved by GNN models when nodes are randomly separated for train/validation/test and when they are arranged chronologically.


\begin{table}[hbt!]
    \small
    \centering
        \vspace{-14pt}
        \caption{Accuracy of SeHGNN \cite{SeHGNN} on ogbn-mag when dataset split chronologically and randomly.}
        \vspace{2pt}
        \begin{tabular}{lll}
        \hline
        Split                                & wo/time emb                            & w/time emb                             \\ \hline
        {Chronological} & {0.5682 �� 0.001}  & {0.5580 �� 0.0009} \\ \hline
        {Random}        & {0.6302 �� 0.0011} & {0.6387 �� 0.0011} \\ \hline
        \end{tabular}
        \vspace{3pt}
        \label{table:toy}
        \vspace{-12pt}
    
        
    \end{table}
    
    %when datasets split chronologically and randomly. Chronological split includes data up to 2017 in the training set, data from 2018 in the validation set, and data thereafter in the test set. Time encoding utilized a 20-bit sinusoidal positional encoding.

% For the effectiveness of time positional encoding, an improvement in test accuracy is observed only when the dataset split is random, suggesting that the temporal information of nodes influences the distribution of neighboring node labels. However, in the context of chronological split datasets, a decrease in performance is noted. This can be intuitively explained within the context of chronological split datasets: during training, nodes with time positional encoding corresponding to recent temporal information are not visited as target nodes. Consequently, the inference process of test nodes encounters time positional encoding not encountered during training, necessitating assumptions for non-obvious extrapolation.
% This toy experiment's results clearly demonstrate the presence of distribution shift induced by chronological split and highlight the challenges in addressing this issue. The substantial difference in accuracy, 5.2\%, between the chronological split and random split settings underscores this point. Furthermore, the increase in accuracy when using time positional encoding in the Random split setting indicates that the temporal information of nodes influences the distribution of neighboring connections. However, time positional encoding does not contribute to the improvement of test accuracy in the chronological split setting. This discrepancy arises because the inference process of test nodes encounters time positional encoding not encountered during training. The distribution for recent nodes may exhibit an extrapolation property diverging from that of past nodes, thereby contributing to the challenging nature of this problem.

The substantial difference in accuracy, 5.2\%, between the chronological split and random split settings clearly demonstrates the presence of distribution shift induced by the chronological split. Time positional encoding contributes to obtain better test accuracy only in random split setting. This discrepancy arises because in the chronological split setting, the inference process of test nodes encounters time positional encodings not seen during training. The distribution for recent nodes may exhibit extrapolation properties that diverge from those of past nodes, thereby adding to the challenging nature of this problem. In our work, we presented robust and realistic assumptions on temporal graph, and proposed message passing methods, \IMPaCT, to learn invariant representation.








% These disparities create domain shift, leading to the Out-of-Distribution (OOD) generalization problem where representations learned from the training set do not generalize effectively to the test set. Thereby, effective utilization of temporal information can lead to significant performance gains. 
% Based on thorough analysis, we presented robust and realistic assumptions that reflect the characteristics of graphs with temporal information, and proposed invariant message passing methods to effectively obtain invariant information. We showcased substantial performance improvements in both real dataset and synthetic dataset, and theoretically proved superiority of our proposed method.



    
\textbf{Contributions} Our research contributes in the following ways:\\
1. We present robust and realistic assumptions that rooted in properties observable in real-world graphs, to effectively analyze and address the domain adaptation problem in graph datasets with chronological splits. 2. We propose the scalable invariant message passing methods, \IMPaCT, and established a theoretical upper bound of the distance between train and test distributions when our methods were used. 3. We propose Temporal Stochastic Block Model (TSBM) to generate realistic temporal graph, and systematically demonstrate the robustness and applicability to general spatial GNNs of \IMPaCT. 4. We showcase significant performance improvements on the real-world citation graph ogbn-mag, yielding significant margin of 6.1\% over current SOTA method.
% (1) We effectively analyzes and addresses the domain adaptation problem in massive datasets with chronological splits. Rather than relying on unprovable principles or define overly broad environment set, we present robust and realistic assumptions that rooted in properties observable in real-world graphs, and rigorously apply the invariant principle to graph domain adaptation within decoupled GNNs. (2) Based on the assumptions, we propose the invariant message passing methods, \IMPaCT, to guarantee the preservation of invariance of the aggregated message during the message passing step. We have demonstrated that this approach enables us to obtain invariant information and have shown its scalability and readiness for application. (3) In experiments with synthetic graphs, we introduce the Temporal Stochastic Block Model (TSBM) to generate realistic temporal graph, and systematically demonstrate the robustness of invariant message passing methods and their applicability to general spatial GNNs. (4) We showcase significant performance improvements on the real-world citation graph dataset ogbn-mag. Applying only 1st moment alignment yields a substantial improvement over the previous state-of-the-art, with a significant margin of 5.4\%, and 6.1\% when applying 2nd moment alignment.


\section{Related Work}
\label{sec:rel}
\subsection{Graph Neural Networks.} 
 Graph Neural Networks (GNNs) have gained significant attention across various domains, including recommender systems \cite{ying2018graph,gao2022graph}, biology \cite{barabasi2004network}, and chemistry \cite{wu2018moleculenet}. Spatial GNNs, such as GCN \cite{GCN}, GraphSAGE \cite{Graphsage}, GAT \cite{velickovic2017graph} and HGNN \cite{HGNN}, derive topological information by aggregating information from neighboring nodes through message passing and semantic aggregation.
\vspace{-2pt}
\begin{align}
    \small
    \small
    &M_v^{(k+1)} \leftarrow \text{AGG}(\{X_u^{(k)},\forall u\in \mathcal{N}_v\}) \\ &X_v^{(k+1)}\leftarrow \text{COMBINE}(\{X_v^{(k)},M_v^{(k+1)}\}),\ \forall v\in \bold{V}, k<K
\end{align}
\vspace{-15pt}

Here, $K$ is the number of GNN layers, $X_v^0 = X_v$ is initial feature vector of each node, and the final representation $X_v^K=Z_v$ serves as the input to the node-wise classifier. The AGG function performs topological aggregation by collecting information from neighboring nodes, while the COMBINE function performs semantic aggregation through processing the collected message for each node.
Scalability is a crucial issue when applying GNNs to massive graphs. The ego graph, which defines the scope of information influencing the classification of a single node, exponentially increases with the number of GNN layers. Therefore, to ensure scalability, algorithms must be meticulously designed to efficiently utilize computation and memory resources \cite{Graphsage,clustergnn,graphsaint}. Decoupled GNNs, whose process of collecting topological information occurs solely during preprocessing and is parameter-free, such as SGC \cite{SGC}, SIGN \cite{rossi2020sign}, and GAMLP \cite{GAMLP}, have recently demonstrated outstanding performance and scalability on many real-world datasets. Furthermore, SeHGNN \cite{SeHGNN} and RpHGNN \cite{RpHGNN} propose decoupled GNNs that efficiently apply to heterogeneous graphs by constructing separate embedding spaces for each metapath based on HGNN \cite{HGNN}.

%For instance Hamilton et al. \cite{Graphsage} introduced node-wise neighbor sampling, and Shi et al. \cite{clustergnn} and Zeng et al. \cite{graphsaint}, introduced graph-wise sampling, which splits the graph into smaller subgraphs based on its properties.


%Notably, SeHGNN proposes a decoupled GNN that efficiently applies to heterogeneous graphs by constructing different embedding spaces for each metapath \cite{SeHGNN}, while RpHGNN introduces random projection into the aggregation process to reduce information loss while bounding complexity \cite{RpHGNN}. These models exhibit excellent performance on large and critical datasets such as ogbn-mag and ogbn-papers100M.




% \textbf{Chronological Split.} The task of Semi-supervised Node Classification (SSNC) often involves nodes with temporal information. For instance, in academic graph representing connections between papers, authors, and institutions, each paper node may contain information regarding the year of its publication. The focus of this study lies within such graph data, particularly on datasets where the train and test splits are arranged in chronological order. In other words, the separation between nodes available for training and those targeted for inference occurs temporally, requiring the classification of the labels of nodes with the most recent temporal information based on the labels of nodes with historical temporal information. While leveraging GNNs trained on historical data to classify newly added nodes is a common scenario in industrial and research settings, systematic research on effectively utilizing temporal information within chronological split graphs remains scarce.

% Failure to appropriately utilize temporal information can lead to significant performance degradation when the model attempts to classify labels of recent data. The distribution of neighboring nodes�� classes may not remain constant over time, and the relative connectivity frequency of nodes also depends on temporal information. 

% We conducted a toy experiment on the ogbn-mag dataset, an academic graph dataset features a chronological split, to confirm the existence of such distribution shifts.\cite{OGB} The specific settings and results of this experiment can be found in Appendix 1.

% These disparities create distribution shift, leading to the Out-of-Distribution (OOD) generalization problem where representations learned from the training set do not generalize effectively to the test set. Thereby, effective utilization of temporal information can lead to significant performance gains. Based on thorough analysis, we presented robust and realistic assumptions that reflect the characteristics of graphs with temporal information, and proposed invariant message passing methods to effectively obtain invariant information. We showcased substantial performance improvements in both real dataset and synthetic dataset, and theoretically proved superiority of our proposed method.





\subsection{Domain adaptation} 
The objective of domain adaptation is to minimize the maximum risk achievable across possible environment sets $\epsilon$ by optimizing function $f$. For data $X$ and its corresponding label $Y$, the distribution $P(X,Y)$ is determined by the environment $e$. For error function $L$,

% \vspace{-15pt}
% \begin{align}
%     \text{Risk : } R(f\mid e) &= \mathbb{E}_{X,Y\sim P_e(X,Y)}{L(f(X),Y)}\\ 
%     \text{Empirical risk : } \hat{R}(f\mid e)&= {1\over n} \sum_i{L(f(X_i),Y_i)}
% \end{align}
% \vspace{-5pt}


\vspace{-18pt}
\begin{align}
    \small
    R(f\mid e) = \mathbb{E}_{X,Y\sim P_e(X,Y)}{L(f(X),Y)}, f^* \in {arg\,min}_{f} \sup_{e\in \epsilon} \hat{R}(f\mid e)
\end{align}
\vspace{-15pt}


where $R(f\mid e)$ is the risk and $f^*$ is the optimal function. Without direct or indirect assumptions about the data generation process, it is practically infeasible to optimize the risk minimization objective \cite{wu2022handling,arjovsky2019invariant}. Therefore, recent studies \cite{rojas2018invariant,chen2022invariance,wu2022handling,shift_robust} adopt the Invariance Principle and the Sufficient Principle, which assume the existence of an invariant generator $\Phi$ satisfying the following:\\
(a) Invariance principle: $\forall e, e' \in \epsilon, P_e(Y\mid \Phi(X)) = P_{e'}(Y\mid \Phi(X))$.\\
(b) Sufficient principle: $Y=f^*(\Phi(X))+u$, $u \perp X$, where $\perp$ indicates statistical independence.
% This objective has two significant challenges. Firstly, it is impossible to know the environment set $\epsilon$ perfectly. In most cases, we can only empirically access $e\in \epsilon^{tr}$ since we usually have realized datasets. The second problem is the computational expense of directly optimizing the above objective. Even with a sufficient environment set $\epsilon$ for training, calculating the supremum of $\hat{R}(f\mid e)$ over every ${e\in \epsilon}$ requires computing the risk for all environments. Therefore, many studies on domain adaptation simplify the problem to a computationally tractable level by accepting the Invariant principle \cite{rojas2018invariant,arjovsky2019invariant,wu2022handling} and the Sufficient principle \cite{chen2022invariance,wu2022handling,shift_robust}.




\subsection{Domain adaptation in graph} 
Despite its significance, studies on domain adaptation in GNNs are relatively scarce. Notably, to our knowledge, no studies propose invariant learning applicable to large graphs. For example, EERM \cite{wu2022handling} defines a graph editor that modifies the graph to obtain invariant features through reinforcement learning, which cannot be applied to decoupled GNNs. SR-GNN \cite{shift_robust} adjusts the distribution distance of representations using a regularizer, with computational complexity proportional to the square of the number of nodes, making it challenging to apply to large graphs. This scarcity is attributed by several factors: data from different environments may have interdependencies, and the extrapolating nature of environments complicates the problem. In this section, we delve into the characteristics of domain adaptation in graphs and outline the challenges we aim to address.

% In our study, environment is defined by the temporal information $t$ associated with the node, and the environment set encompasses possible temporal information, $\bold{T}=\{\dots, t_{max}-1, t_{max}\}$. A variable is invariant if and only if its distribution does not depend on $t$. Despite its importance, studies on domain adaptation in GNNs are relatively scarce. Furthermore, to our knowledge, no studies propose invariant learning methods applicable to large-scale graphs. Notably, EERM \cite{wu2022handling} defines a graph editor that adversarially modifies the graph to obtain invariant features through reinforcement learning, which cannot be applied to decoupled GNNs. SR-GNN \cite{shift_robust} adjusts the distribution distance of representations using a regularizer, with computational complexity proportional to the square of the number of nodes. This scarcity can be attributed to several factors: data from different environments may have interdependencies and extrapolating nature of environments. In this section, we delve into the characteristics of domain adaptation in graphs and outline the challenges we aim to address through our research.

In transductive graph learning tasks, defining separate environments is challenging due to interdependencies among the data. Confining the discussion to spatial GNNs, this problem can be mitigated by defining the ego-graph $C_g(v) = \left(v, \mathcal{N}_k(v), \{(u_{src},u_{dst}) \in E \mid u_{src}, u_{dst} \in \mathcal{N}_k(v)\} \right)$, where $\mathcal{N}_k(i)$ denotes neighbor nodes within a distance of $k$ from node $v$ and $E$ denotes the set of edges. Additionally, if the process of topological aggregation from the ego-graph $C_g(v)$ is learnable, extracting invariant information cannot be modeled separately from the downstream function $f$. Thus, we confine our analysis to decoupled GNNs. Here, the objective can be rewritten as follows:




\vspace{-15pt}
\begin{align}
    \small
    {\min}_{f} \sup_{e\in \epsilon} \hat{R}\big(f(\Phi \circ C_g) \mid e\big), \text{ s.t. } x\perp u
\end{align}
\vspace{-15pt}

In our study, the environment is defined by the temporal information $t$ associated with the nodes, and the environment set encompasses possible temporal information, $\bold{T} = \{\dots, t_{max} - 1, t_{max}\}$. A variable is considered invariant if and only if its distribution does not depend on $t$. When domain shift arises from chronological split, $e^{te}$ cannot be considered part of $\epsilon^{tr}$. Thus, constructing $\epsilon^{all}$, which effectively includes $\epsilon^{te}$, is essential. However, we are not interested in all possible $\epsilon^{all}$, as overly general cases can lead to suboptimal risk bounds. Theorems \ref{thm:env1} and \ref{thm:env2} address this concern.


% In our study, focusing on chronological split, the environment becomes the temporal information associated with the nodes. For clarity, let's consider the $t$, temporal information possessed by the nodes, as the environment, and the set of possible temporal information $\bold{T}$ as the environment set. In other words, if the distribution of any value or attribute $x_i$ assigned to node $i$ does not depend on the node's temporal information $t_i$, then it is considered invariant. While other environments may exist in the graph that affect the distribution of data and labels, analyzing only temporal information is sufficient since the train/test split occurs solely based on time information.

% \textbf{Interdependency among environments.}
%In such cases, it is challenging to analyze the environment and its effects adequately. Fortunately, by confining the discussion to spatial GNNs, this problem can be relieved by defining the ego-graph $C_g(x)$, which is the set of information influencing the prediction of a specific node $i$.
% \vspace{-15pt}
% \begin{align}
%     C_g(i) =\big\{i,\ \mathcal{N}_k(i),\ \{(u,v)\mid u,v\in \mathcal{N}_{k}(i)\}\big\}
% \end{align}
% \vspace{-5pt}

% Here, the $\mathcal{N}_k(i)$ denotes the set of neighbor nodes within a distance of $k$ from node $i$. The inclusion of the target node in the definition of the ego-graph is needed because, even if the subgraph remains the same, the topological information varies depending on which node serves as the center of the subgraph.


% \textbf{Time and memory complexity.}
% In the realm of transductive graph learning or domain adaptation, practical considerations of time and memory complexity are paramount.


% \textbf{\textit{Extrapolating} nature of environments.}
% Earlier, it was stated that without direct or indirect assumptions about $e^{te}$, it is practically infeasible to optimize risk minimization objective. Building upon the aforementioned discussion, two approaches emerge: (1) assuming the inclusion of $e^{te}$ in $\epsilon^{tr}$, or (2) augmenting or generalizing $\epsilon^{tr}$ to $\epsilon^{all}$. However, in the case of (1), when a chronological split occurs, $e^{te}$ denotes the most recent temporal information, thus it clearly cannot be considered a part of $\epsilon^{tr}$. From this perspective, chronological split can be seen as a task necessitating \textit{extrapolation} on the underlying environment. As for approach (2), it is crucial to generalize $\epsilon^{all}$ effectively to reflect the causal data generation process akin to actual data.
% Without direct or indirect assumptions about $e^{te}$, it is practically infeasible to optimize risk minimization objective \cite{wu2022handling,arjovsky2019invariant}. Therefore, assuming the inclusion of $e^{te}$ in $\epsilon^{tr}$ or generalizing $\epsilon^{tr}$ to $\epsilon^{all}$, which includes $\epsilon^{te}$ based on assumptions about the graph generation process, is necessary. However, in the case of domain shift caused by chronological split, $e^{te}$ clearly cannot be considered a part of $\epsilon^{tr}$. Therefore, it is essential to construct $\epsilon^{all}$ by effectively reflecting the causal influence of the environment on graph generation, which includes $\epsilon^{te}$.


% For instance, in many graph Out-of-Distribution (OOD) generalization studies, either covariate shift or concept shift has been assumed \cite{GOOD}, where $P_e(X|Y)=P_{e'} (X|Y)$ but $P_e(X)\neq P_{e'} (X),\ \forall e,e'\in \epsilon$, it is referred to as covariate shift, and vice versa for concept shift. However, since real data often involves both covariate and concept shifts simultaneously, assuming only one of them is unrealistic \cite{GOOD,bazhenov2024evaluating}. By generalizing $\epsilon^{all}$ based on such unrealistic assumptions, the learned invariant information cannot be regarded as a causal factor influencing the label distribution. 
%Wu et al. \cite{wu2022handling} resort to employing the graph permutation invariant assumption to obtain $\epsilon^{all}$. However, while permutation invariance is realistic, it is not particularly useful for analyzing chronological splits. Permutation invariance is a broadly applied concept that does not adequately account for the differences in topological and semantic information due to chronological information.

% Lastly, we aim to demonstrate that defining an unnecessarily broad environment set leads to suboptimal bounds. All proofs are provided in Appendix \ref{apdx:proofs}.

\begin{theorem}\label{thm:env1}
\textbf{Larger environment set gives larger optimal risk.} \\
Suppose that the optimal function of minimizing the maximal risk exists for both environment sets $\epsilon_1$ and $\epsilon_2$ which satisfies $\epsilon_1 \subseteq \epsilon_2$. Let $f_1^* \in  {arg\,min}_{f} \sup_{e\in \epsilon_1} \hat{R}(f\mid e)$ and $f_2^* \in  {arg\,min}_{f} \sup_{e\in \epsilon_2} \hat{R}(f\mid e)$. Then, $\sup_{e\in \epsilon_1} \hat{R}(f_1^* \mid e) \leq \sup_{e\in \epsilon_2} \hat{R}(f_2^* \mid e)$. Equivalently, ${\min}_{f} \sup_{e\in \epsilon_1} \hat{R}(f\mid e) \leq {\min}_{f} \sup_{e\in \epsilon_2} \hat{R}(f\mid e)$.
\end{theorem}


\begin{theorem}
\label{thm:env2}\textbf{The optimal minimizer function learned on a larger environment set gives a larger test error for such test environments satisfying the condition of the statement.} \\
Let $\delta = \sup_{e\in \epsilon_2} \hat{R}(f_2^*\mid e) - \sup_{e\in \epsilon_1} \hat{R}(f_1^*\mid e) \geq 0$. Non-negativity is obtained by Theorem \ref{thm:env1}. For $e_{te} \in \epsilon_1 \subseteq \epsilon_2$ such that $\hat{R}(f_2^* \mid e_{te}) \geq \sup_{e\in \epsilon_2} \hat{R}(f_2^* \mid e) - \delta$, $\hat{R}(f_1^* \mid e_{te}) \leq \hat{R}(f_2^* \mid e_{te})$ holds.
\end{theorem}


% It has been demonstrated that tight environment generalization is superior. 
All proofs are provided in Appendix \ref{apdx:proofs}. Thus, it is essential to construct $\epsilon^{all}$ by effectively reflecting the causal influence of the environment on graph generation, which includes $\epsilon^{te}$. We propose realistic assumptions derived from the influence of temporal information on the distribution in real-world graphs. These assumptions directly address considerations regarding $e^{te}$, implying that they offer a superior risk-minimizing approach compared to any other assumption, as supported by Theorems \ref{thm:env1} and \ref{thm:env2}. Leveraging these assumptions, we have devised Invariant Message Passing (\IMPaCT), scalable methods capable of effectively extracting invariant information.

%This implies the necessity of introducing assumptions that accurately specify and explain the distribution shift between $e^{te}$ and the remaining environments. 
% Broadening the environment set that don't address the real causes of distribution differences between $e^{te}$ and other environments is inefficient.




%\subsection{Invariant representations on graph}
% Through the preceding discussions, we have introduced the challenges of domain adaptation in graph data. Particularly, we presented the difficulties arising from the \textit{extrapolating} nature of chronological splits, which complicates the introduction of existing methodologies and assumptions. Moreover, we underscored the significant time and memory requirements in handling domain adaptation problems within large graphs, imposing substantial constraints.




% In this study, we introduced realistic assumptions outlined in Assumption \ref{assumption}, which are derived from the influence of temporal information on the distribution. These assumptions directly consider factors related to $e^{te}$, suggesting that if validated as realistic, they offer a superior risk-minimizing approach compared to any other assumption, as supported by Theorems \ref{thm:env1} and \ref{thm:env2}. Leveraging these assumptions, we have devised Invariant Message Passing (\IMPaCT), scalable methods capable of effectively extracting invariant information.





% In this study, we proposed realistic assumptions based on the influence of temporal information on the distribution in real graph data, as outlined in Assumption \ref{assumption}. Since these assumptions inherently involve direct considerations regarding $e^{te}$, if these assumptions are validated as realistic, they represent a superior risk-minimizing approach compared to any other assumption, by Theorems \ref{thm:env1} and \ref{thm:env2}.


% In practical GNNs, the process of aggregating information from the ego-graph $C_g(i)$ to classify the target node $i$ is inherently learnable. This implies that the extraction of invariant information, as depicted in the equation above, cannot be separated from the downstream function $f$. To simplify the discussion, we have confined our analysis to decoupled GNNs. Therefore, if the final aggregated information remains invariant, it corresponds to $\Phi(C_g(i))$. In this scenario, $\Phi(C_g(i))$ can be assumed to belong to $\mathbb{R}^h$, where $h$ represents the dimensionality of the final representation. Though our mathematical discussion is limited to decoupled GNNs, but through experiments on real graphs, we will demonstrate that \IMPaCT can also be applied to general spatial GNNs, yielding significant performance improvements. 

% In conclusion, through the \IMPaCT devised in our study, we derive invariant representations $\Phi(C_g(i))$ by directly assuming the impact of chronological information on the connectivity distributions. This process enables models trained on the training dataset to generalize to the test data without the need for specific assumptions about causation, such as the principle of "Invariance as causation". This approach holds significant importance in demonstrating the viability of solving domain adaptation problems based on assumptions that can be empirically validated, rather than on unprovable principles.

% The performance degradation caused by chronological splits can be clearly observed by comparing the maximum accuracy achieved by GNN models when nodes are randomly separated for train/validation/test and when they are arranged chronologically.

% For the effectiveness of time positional encoding, an improvement in test accuracy is observed only when the dataset split is random, suggesting that the temporal information of nodes influences the distribution of neighboring node classes. However, in the context of chronological split datasets, a decrease in performance is noted. This can be intuitively explained within the context of chronological split datasets: during training, nodes with time positional encoding corresponding to recent temporal information are not visited as target nodes. Consequently, the inference process of test nodes encounters time positional encoding not encountered during training, necessitating assumptions for non-obvious extrapolation.

\section{Assumptions}
%\subsection{Assumptions}
Here are 3 assumptions introduced in this study. Let $\bold{Y}$ denote the set of labels.
%The set $\bold{T}$ consists of discrete time elements, ranging from infinitely past times to the most recent time, with uniform intervals. 

\vspace{-18pt}

\begin{flalign}\label{assumption}
    \small
    &\textit{Assumption 1}: P_{e^{te}}(Y) = P_e(Y), \ \forall e \in \epsilon^{tr}\\
    &\textit{Assumption 2}: P_{e^{te}}(X| Y) = P_e(X| Y), \ \forall e \in \epsilon^{tr}\\
    &\textit{Assumption 3}: \mathcal{P}_{y t} \left(\tilde{y}, \tilde{t}\right) = f(y, t) g\left(y, \tilde{y}, | \tilde{t}-t|\right), \ \forall y, \tilde{y} \in \bold{Y}, \forall t, \tilde{t} \in \bold{T}
\end{flalign}

\vspace{-2pt}

\begin{wrapfigure}{R}{0.42\textwidth}
    \vspace{-15pt}
    \centering
    \includegraphics[width=0.42\textwidth]{figs/scale_factor_f.png}
    \caption{Graphical representation of functions $f$ and $g$. The shaded bars denote relative connectivity. Target node has label $y$, and only consider cases neighboring nodes with a labels $\tilde y$. The function $g(y,\tilde y,|\tilde t-t|)$ determines extent to which relative connectivity varies, and its scale is adjusted by the function $f(y, t)$.}
    \label{fig:scale_factor_f}
    \vspace{-16pt}
\end{wrapfigure}
From now on, we will denote $y$ and $t$ as the label and time of the target node, and $\tilde{y}$ and $\tilde{t}$ as the label and time of neighboring nodes, unless specified otherwise. Relative connectivity $\mathcal{P}_{y t} \left(\tilde{y}, \tilde{t}\right)$ denotes the probability distribution of label and time pairs of neighboring nodes. Hence, $\sum_{\tilde{y} \in \bold{Y}}\sum_{\tilde{t} \in \bold{T}} \mathcal{P}_{y t} \left(\tilde{y}, \tilde{t}\right)=1$. 
 %This does not signify the actual probability of connection but rather represents the relative proportions of attributes among neighboring nodes. Additionally, the functions $f(y_1,t_1)$ and $g(y_1, y_2, | t_1- t_2|)$ indicate separability functions rather than probability density functions.


Assumptions 1 and 2 posit that the initial features and labels allocated to each node originate from same distributions. Assumption 3 assumes separability in the distribution of neighboring nodes. It is based on the observation that the proportion of nodes at time $\tilde{t}$ within the set of neighboring nodes of the target node at time $t$ decreases as the time difference $|\tilde{t} - t|$ increases. $g\left(y, \tilde{y}, | \tilde{t}-t|\right)$ is the function representing the proportion of neighboring nodes as a function that decays as $|\tilde{t} - t|$ increases. However, distributions of relative time differences among neighboring nodes vary depending on the target node's time $t$. $f(y ,t)$ is to adjust relative proportion value $g\left(y, \tilde{y}, | \tilde{t}-t|\right)$ to construct $\mathcal{P}_{y t} \left(\tilde{y}, \tilde{t}\right)$ as a probability density function. These assumptions are rooted in properties observable in real-world graphs. The motivation and analysis on real-world temporal graphs are provided in Appendix \ref{apdx:assumptions}.


%Incorporating Assumptions 1 and 2 yields $P_e(X, Y) = P(X, Y)$; however, interpreting this as the absence of distribution shift would be erroneous. Even if the joint distribution of the initial feature $X^{(0)}$ and label remain identical, the topological information within ego-graphs varies with the target node's temporal context. Failure to adequately address such shifts results in the aggregated message distribution shifting with each GNN aggregation layer.


% Assumption 3 is based on the observation that the proportion of nodes at time $t_2$ within the set of neighboring nodes of the target node at time $t_1$ decreases as the time difference $|t_2 - t_1|$ between them increases. $g(y_1,y_2,|t_2 - t_1|)$ is the function representing the proportion of neighboring nodes as a function that decays relative to the time difference $|t_2 - t_1|$. However, assuming $g(y1, y2, |t_2-t_1|)$ directly as a joint distribution is unrealistic. This is because the distribution of relative time differences among neighboring nodes varies depending on the target node's time $t_1$. For instance, if $t_1=1$, neighboring nodes can have relative times of $0,1,\dots,t_{max}-1$, while if $t_1=\lfloor t_{max}/2\rfloor$, the possible relative times of neighboring nodes become $0,1,\dots,\lfloor(t_{max}+1)/2\rfloor$. Therefore, to ensure that $\mathcal{P}_{y_1 t_1}(y_2, t_2)$ becomes a probability density function, the relative proportion value $g(y_1, y_2, |t_2 - t_1|)$ needs to be adjusted. The function $f(y_1 ,t_1)$ plays the role of converting these relative proportion values into probability density function values.







% These assumptions are rooted in properties observable in real-world graphs. For instance, in the academic paper citation graph utilized in this study, labels represent the categories of papers, while features comprise vector embeddings of paper abstracts. While the joint distribution of paper categories and abstracts may remain stable with minor temporal changes, the probability of two papers being linked via citation decreases significantly with the temporal gap between them. Hence, in citation graphs, the probability distribution of connections between nodes evolves much more sensitively to time than to features or labels.




\subsection{Invariant Message Passing for Chronological-split Temporal graph, \IMPaCT}

Based on these assumptions, we introduce the 1st moment alignment methods, \MMP and \PMP, and the 2nd moment alignment methods, \PNY and \JJnorm, to preserve the invariance of the 1st and 2nd moments in the message passing process. To be more specific, proposed invariant message passing function ensures the aggregated message $M_{v}^{(k+1)}$ to approximately satisfy $P_e(M_v^{(k+1)}| Y) =P_{e^{te}}(M_v^{(k+1)}| Y),\ \forall e\in \epsilon^{tr}$ when the representations $X_v^{(k)},\forall v\in \bold{V}$ at the $k$-th layer satisfies $P_e(X_v^{(k)}| Y) =P_{e^{te}}(X_v^{(k)}| Y),\ \forall e\in \epsilon^{tr}$. Given Assumption 2, where $P_e(Y)=P(Y),\ \forall e\in \epsilon^{tr}$, it follows that our methods ensure the aggregated message to be approximately invariant when the previous layer��s representations are invariant. Furthermore, we demonstrated that final representations remain approximately invariant when applying \IMPaCT methods to GNNs composed of multiple layers, in Section \ref{sec:firstorder}. To model the chronological split, we designate the set of nodes with time $t_{max}$ as the test set. Therefore, labels of nodes with time $t_{max}$ are unknown.


% Furthermore, we demonstrated final representations remain approximately invariant when applying invariant message passing methods to decoupled GNNs composed of multiple layers featuring nonlinear semantic aggregation at each layer. Which means, even if representations $X_{v}^{(k+1)}= \sigma(M_{v}^{(k+1)})$ obtained through a nodewise semantic aggregation function, $\sigma:\R^{f^{(k+1)}} \rightarrow \R^{h^{(k+1)}}$, $P_e(X_v^{(k+1)}, Y) =P_{e^{te}}(X_v^{(k+1)}, Y),\ \forall e\in \epsilon^{tr}$ holds under certain bounds.  Since the final representation obtained during the aggregation of topological information represents $\Phi(C_g(v))$, it follows that $P_e(\Phi(C_g(v)), Y) = P_{e^{te}}(\Phi(C_g(v)), Y),\ \forall e\in \epsilon^{tr}, v\in \bold{V}$. Therefore, using \IMPaCT, we can obtain approximately invariant information $\Phi(C_g(v))$. 
% As the joint distribution of inputs and outputs of the downstream function $f:\mathbb{R}^h\rightarrow\bold{Y}$ remains consistent across environments, the model trained on the training data inherently generalizes to the test dataset.

% \vspace{-15pt}
% \begin{align}
%     f^*={\arg\min}_{f} {1\over | \bold{V^{tr}}|} \sum _{v\in \bold{V^{tr}}} \mathcal{L}\big(f(\Phi(C_g(v))),Y_v\big)
% \end{align}
% \vspace{-10pt}


% To model the chronological split, we designate the set of nodes with time information $t_{max}$ as the test set. In alternative terms, within the set of environments $\epsilon=\{\dots,t_{max}\}$, the test environment comprises $e^{te} = {t_{max}}$, while the train environment consists of $e^{tr} = \{\dots,t_{max}-1\}$. Therefore, in subsequent discussions, we presume that the labels of nodes with time $t_{max}$ are unknown during the training process.



\section{First moment alignment methods}\label{sec:1st}
\label{sec:firstorder}
Message passing refers to the process of aggregating representations from neighboring nodes in the previous layer. Here, we assume the commonly used averaging message passing procedure. For any arbitrary target node $v\in\bold{V}$ with label $y$ and time $t$,

\vspace{-15pt}
\begin{align}
    M_{ v}^{(k+1)}={\sum_{\tilde{y}\in \bold{Y}}\sum_{\tilde{t}\in\bold{T}}\sum_{w\in\mathcal{N}_{v}\left(\tilde{y}, \tilde{t}\right)}X_w \over \sum_{\tilde{y}\in \bold{Y}}\sum_{\tilde{t}\in\bold{T}} |\mathcal{N}_{v}\left(\tilde{y}, \tilde{t}\right)|} , X_w \overset{\text{IID}}{\sim} {x_{\tilde{y}\tilde{t}}^{(k)}} \; \text{for} \; \forall w\in\mathcal{N}_{v}\left(\tilde{y}, \tilde{t}\right)
\end{align}
\vspace{-5pt}

\begin{wrapfigure}{R}{0.2\textwidth}
    \vspace{-15pt}
    \centering
    \includegraphics[width=0.2\textwidth]{figs/PMP.png}
    \caption{Graphical explanation of \PMP}
    \label{fig:PMP}
    \vspace{-30pt}
\end{wrapfigure}

where $\mathcal{N}_{v}\left(\tilde{y}, \tilde{t}\right) = \{w \in \bold{V} \mid w \; \text{is a neighbor of} \; v \; \text{which has label} \; \tilde{y} \; \text{and time} \; \tilde{t} \}$. 
$M_v^{(k+1)}$ is the aggregated message at node $v$ in the $k+1$-th layer, and $x_{yt}^{(k)}$ is the distribution of representations from the previous layer. For simplification, the term "$X_w \overset{\text{IID}}{\sim} {x_{\tilde{y}\tilde{t}}^{(k)}} \; \text{for} \; \forall w\in\mathcal{N}_{v}\left(\tilde{y}, \tilde{t}\right)$" will be omitted in definitions in the subsequent discussions.\\ %, and we will consider $v$ as the target node with label $y$ and $t$ if there are no other specifications. \\
\textit{\textbf{The first moment of aggregated message.}} If the representations from the previous layer have means which are consistent across time, i.e., ${\mathbb{E}}_{X\sim {x_{\tilde{y}\tilde{t}}^{(k)}}}\left[X\right]=\mu_{X}^{(k)}(\tilde{y})$, we can calculate the approximate expectation, defined in Appendix \ref{apdx:firstmm}, as $\hat{\mathbb E}[{M_{ v}^{(k+1)}}]=\sum_{\tilde{y}\in \bold{Y}}\sum_{\tilde{t}\in\bold{T}}\mathcal{P}_{y t}\left(\tilde{y}, \tilde{t}\right)\mu_{X}^{(k)}(\tilde{y})$.

% \vspace{-15pt}
% \begin{align}
%     \hat{\mathbb E}[{M_{ v}^{(k+1)}}]=\sum_{\tilde{y}\in \bold{Y}}\sum_{\tilde{t}\in\bold{T}}\mathcal{P}_{y t}\left(\tilde{y}, \tilde{t}\right)\mu_{X}^{(k)}(\tilde{y})
% \end{align}
% \vspace{-5pt}

% and let us call it \textit{the first moment of aggregated message}. See Appendix \ref{apdx:firstmm} for the details of subtle difference between the true expectation $\mathbb{E}$ and the approximate expectation $\hat{\mathbb{E}}$. \\
% We can observe that in contrast to ${\mathbb{E}}_{X\sim {x_{\tilde{y}\tilde{t}}^{(k)}}}\left[X\right]$, $\hat{\mathbb E}[{M_{ v}^{(k+1)}}]$ remains dependent on the node's time $t$ due to  $\mathcal{P}_{y t}\left(\tilde{y}, \tilde{t}\right)$. Our objective is to modify the spatial aggregation method to ensure that the final representation is obtained by collecting all invariant topological information. 
We can observe that $\hat{\mathbb E}[{M_{ v}^{(k+1)}}]$ depends on node's time $t$ due to  $\mathcal{P}_{y t}\left(\tilde{y}, \tilde{t}\right)$. Our objective is to modify the spatial aggregation method to ensure invariance of 1st moment is preserved.
%Here, we propose an improved message passing method to ensure that the 1st moment of the aggregated message obtained through message passing is invariant with respect to time.

\subsection{Persistent Message Passing: \PMP}
We propose Persistent Message Passing (\PMP) as one approach to achieve 1st moment alignment,.

For the target node $v$ with time $t$, consider the time $\tilde t$ of some neighboring node. For $\Delta=  | \tilde t - t|$ where $0<\Delta \le | t_{max}-t|$, both $t + \Delta$ and $t - \Delta$ neighbor nodes can exist. However, nodes with $\Delta >|t_{max}-t|$ or $\Delta = 0$ are only possible when $\tilde t=t-\Delta$. Let $\bold{T}_{t}^{\text{double}} = \{\tilde t \in \bold{T} \mid 0<|\tilde t - t| \le | t_{max}-t|\}$ and $\bold{T}_{t}^{\text{single}} = \{\tilde t \in \bold{T} \mid | \tilde t - t| > | t_{max}-t| \; \text{or} \; \tilde t = t\}$. As discussed, the target node receives twice the weight from $\tilde t \in \bold{T}_{t}^{\text{double}}$ against $\tilde t \in \bold{T}_{t}^{\text{single}}$. Motivation behind \PMP is to correct this by double weighting the neighbor nodes with time in $\bold{T}_{t}^{\text{single}}$, as depicted in Figure \ref{fig:PMP}.


% \vspace{-15pt}
% \begin{align}
%     \bold{T}_{t_i}^{\text{double}} = \{t_j \in \bold{T} \mid 0<| t_j - t_i| \le | t_{max}-t_i|\}\\
%     \bold{T}_{t_i}^{\text{single}} = \{t_j \in \bold{T} \mid | t_j - t_i| > | t_{max}-t_i| \text{or} t_j = t_i\}
% \end{align}
% \vspace{-15pt}




\begin{definition}\label{def:env1}
The \PMP from the $k$-th layer to the $k+1$-th layer of target node $v$ is defined as:

\vspace{-15pt}
\begin{align}\label{eqn:pmp}
M_{ v}^{pmp(k+1)} &= {\sum_{\tilde{y}\in \bold{Y}}\sum_{\tilde{t}\in\bold{T}_{t}^{\text{single}}}\sum_{w\in\mathcal{N}_{v}\left(\tilde{y}, \tilde{t}\right)}2X_w+\sum_{\tilde{y}\in \bold{Y}}\sum_{\tilde{t}\in\bold{T}_{t}^{\text{double}}}\sum_{w\in\mathcal{N}_{v}\left(\tilde{y}, \tilde{t}\right)}X_w \over \sum_{\tilde{y}\in \bold{Y}}\sum_{\tilde{t}\in\bold{T}_{t}^{\text{single}}}2|\mathcal{N}_{v}\left(\tilde{y}, \tilde{t}\right)|+\sum_{\tilde{y}\in \bold{Y}}\sum_{\tilde{t}\in\bold{T}_{t}^{\text{double}}}|\mathcal{N}_{v}\left(\tilde{y}, \tilde{t}\right)|}
\end{align}
\vspace{-5pt}
\end{definition}

\begin{theorem}\label{thm:pmp}
The 1st moment of aggregated message obtained by \PMP layer is invariant, if the 1st moment of previous representation is invariant. \\
\textbf{Sketch of proof} Let $\mathbb{E}_{X\sim{x_{\tilde{y} \tilde{t}}^{(k)}}}\left[X\right]=\mu_{X}^{(k)}(\tilde{y})$ as a function invariant with $\tilde{t}$. Then we can get

\vspace{-15pt}
\begin{align}
\hat{\mathbb E}\left[{M_{ v}^{pmp(k+1)}}\right]={\sum_{\tilde{y}\in \bold{Y}}\sum_{\tau\ge 0}g(y, \tilde{y}, \tau)\mu_{X}^{(k)}(\tilde{y})\over\sum_{\tilde y\in \bold{Y}}\sum_{\tau\ge 0}g(y, \tilde{y}, \tau)}
\end{align}
\vspace{-5pt}
which is also invariant with $t$. See Appendix \ref{apdx:PMP} for details and implementation.
\end{theorem}



\subsection{Mono-directional Message Passing: \MMP}
\PMP is not the only method for achieving 1st order alignment. There can be infinitely many ways to adjust the distribution of absolute relative times to be consistent regardless of the target node's time. We introduce Mono-directional Message Passing (\MMP) as one such approach. \MMP aggregates information only from neighboring nodes whose times are the same as or earlier than the target node.

\begin{definition}\label{def:env3}
The \MMP from the $k$-th layer to the $k+1$-th layer of target node $v$ is defined as:

\vspace{-15pt}
\begin{align}
    M_{v}^{mmp(k+1)} = {\sum_{\tilde{y}\in \bold{Y}}\sum_{\tilde{t}\le t}\sum_{w\in\mathcal{N}_{v}(\tilde y, \tilde t)}X_w \over \sum_{\tilde{y}\in \bold{Y}}\sum_{\tilde{t}\le t}|\mathcal{N}_{v}(\tilde y, \tilde t)|}
\end{align}
\vspace{-5pt}
\end{definition}

\begin{theorem}\label{thm:mmp}
The 1st moment of aggregated message obtained by \MMP layer is invariant, if the 1st moment of previous representation is invariant.\\
\textbf{Sketch of proof} Let $\mathbb{E}_{X\sim{x_{\tilde{y} \tilde{t}}^{(k)}}}\left[X\right]=\mu_{X}^{(k)}(\tilde{y})$ as a function invariant with $\tilde{t}$. Then we can calculate

\vspace{-15pt}
\begin{align}
\hat{\mathbb E}\left[{M_{ v}^{mmp(k+1)}}\right]={\sum_{\tilde{y}\in \bold{Y}}\sum_{\tau\ge 0}g(y, \tilde{y}, \tau)\mu_{X}^{(k)}(\tilde{y})\over\sum_{y\in \bold{Y}}\sum_{\tau\ge 0}g(y, \tilde{y}, \tau)}
\end{align}
\vspace{-5pt}
which is also invariant with $t$. See Appendix \ref{apdx:MMP} for details and implementation.
\end{theorem}

\textbf{Comparison between \PMP and \MMP.} Both \PMP and \MMP adjust the weights of messages collected from neighboring nodes that meet certain conditions, either doubling or ignoring their impact. They can be implemented easily by reconstructing the graph according to the conditions without any modifications to the existing code. However, \MMP collects less information since it only gathers information only from the past during message passing, resulting a smaller ego-graph.
% the effective ego-graph size that affects the final representation decreases exponentially with the number of layers. A decrease in the amount of collected information leads to an increase in the variance of the final representation, resulting in reduced prediction accuracy.

It is important to note that not all invariant information is meaningful. Our ultimate goal is to obtain a maximally informative invariant representation \cite{arjovsky2019invariant}. Since \MMP arbitrarily reduces the effective ego-graph, it cannot be said to meet the requirements for being maximally informative. Therefore, \PMP will be used as the 1st moment alignment method in the subsequent discussions. Furthermore, from Theorem \ref{thm:pmp}, we will denote $\hat{\mathbb{E}}\left[{M_{v}^{pmp(k+1)}}\right]$ as $\mu_{M}^{pmp(k+1)}(y)$ for target node $v$ with label $y$.

% It is important here that not all invariant information is our interest. For instance, if the final representation is constant, it is invariant, but it does not provide meaningful information for classification. In other words, we are interested in invariant representation that is maximally informative \cite{arjovsky2019invariant}. Thus, since \MMP is arbitrarily reducing the effective ego-graph, so it cannot be said that it satisfies the requirements for maximally informative. Based on these analysis, \PMP will be used as the 1st order alignment method in the subsequent discussions. Furthermore, according to Theorem \ref{thm:pmp}, from now on we will write $\hat{\mathbb E}\left[{M_{ v}^{pmp(k+1)}}\right]$ as $\mu_{M}^{pmp(k+1)}(y)$ for target node $v$ with label $y$.

% To summarize, \PMP not only performs better experimentally compared to \MMP as an invariant message passing method but is also the most straightforward and intuitive method to satisfy the necessary condition for maximizing informativeness. \PMP has almost no additional overhead compared to traditional averaging message passing and can be easily applied in practice by simply duplicating edges belonging to $\mathcal{N}_i^{\text{single}}$, that is, by reconstructing the graph. Moreover, it is more expressive than \MMP due to its larger effective ego-graph size. Given these advantages, \PMP will be used as the 1st order alignment method in the subsequent discussions.

\subsection{Theoretical analysis of \PMP when applied in multi-layer GNNs.}\label{sec:analysis}
We will assume the messages and representation to be scalar in this discussion. Let $\mathcal{M}^{(k)}$ as the space of messages at $k$-th layer, and $\mathcal{X}^{(k)}$ as the space of representations at $k$-th layer. We modeled \PMP with probability measures as in Appendix \ref{apdx:pmp_modeling}. Now suppose that, (i) $|M_v^{(k)}|\le C$ almost surely for $\forall v\in V$, $M_v^{(k)}\sim m_{yt}^{(k)}$, and (ii) $\text{var}(M_v^{(k)})\le V$ for $\forall v\in \bold{V}$. Since we are considering 1st moment alignment method, \PMP, we may assume $\mathbb{E}[M_v^{(k)}]=\hat{\mathbb{E}}[M_v^{(k)}]=\mu_{M}^{(k)}(y)$ for $M_v^{(k)}\sim m_{yt}^{(k)},\ \forall y\in \bold{Y}, \forall t\in \bold{T}$. Here, $W_1$ is the Wasserstein-1 metric of probability measures, and we assume G-Lipschitz condition for semantic aggregation functions, $f^{(k)},\ \forall k\in\{1,2,\dots,K\}$.

\begin{theorem}\label{thm:thm1}
    $W_1 (m_{yt}^{(k)},m_{yt��}^{(k)})\le \mathcal{O}(C^{1/3}V^{1/3})$
\end{theorem}
\begin{theorem}\label{thm:thm2}
    If $m_{yt}^{(k)}$ is sub-Gaussian with constant $\tau$, then $W_1(m_{yt}^{(k)}, m_{yt'}^{(k)})\le \mathcal{O}(\tau\sqrt{\log C})$.
\end{theorem}
\begin{theorem}\label{thm:thm3}
    If $\forall y\in \bold{Y}, \forall t, t'\in \bold{T}$, $W_1(m_{yt}^{(k)},m_{yt'}^{(k)})\le W,$ then for $\forall y\in \bold{Y}, \forall t, t'\in \bold{T}$, $W_1(m_{yt}^{(k+1)},m_{yt'}^{(k+1)})\le {G\over G^{(k)}}W$ where $G^{(k)}>1$ is a constant only depending on the layer $k$.
\end{theorem}
\begin{corollary}\label{thm:cor3}
    $W_1(m_{yt}^{(k)},m_{yt_{max}}^{(k)})\le {G^{K-1}\over {G^{(1)}G^{(2)}\dots G^{(K-1)}}}\mathcal{O}(\min\{C^{1/3}V^{1/3},\tau\sqrt{\log C}\})$
\end{corollary}
All proofs are provided in Appendix \ref{apdx:pmp_theory}. Here we ensured that the $W_1$ distance between train and test distributions of final representations are bounded when \PMP is applied in multi-layer GNNs. This is a meaningful theoretical analysis since Wasserstein distances play a  significant role on domain adaptation bounds, both theoretically and practically \cite{shen2018wasserstein,arjovsky2017wasserstein}.

\section{Second moment alignment methods}
While 1st order alignment methods like \PMP and \MMP preserve the invariance of the 1st moment of the aggregated message, they do not guarantee such property for the 2nd moment. 

Let's suppose that the 1st moment of the previous layer's representation $X$ is invariant with node's time $t$, and 2nd moment of the initial feature is invariant. That is, $\forall\tilde{y} \in\bold{Y}, \forall\tilde{t}\in\bold{T}$, $\mathbb{E}[X]=\mu_{X}^{pmp(k)}(\tilde{y})$ for $X\sim {x_{\tilde{y} \tilde{t}}^{pmp(k)}}$ and $\Sigma_{XX}^{pmp(0)}\left(\tilde{y},\tilde{t}\right)=
\Sigma_{XX}^{pmp(0)}\left(\tilde{y},t_{max}\right)$ where $\Sigma_{XX}^{pmp(k)}\left(\tilde{y},\tilde{t}\right)=
\text{var}({X})=\mathbb{E}\left[(X-\mu_{X}^{pmp(k)}\left(\tilde{y},\tilde{t}\right))(X-\mu_{X}^{pmp(0)}\left(\tilde{y},\tilde{t}\right))^{\top}\right]$ for $X\sim {x_{\tilde{y} \tilde{t}}^{pmp(k)}}$. Given that the invariance of 1st moment is preserved after message passing by \PMP or \MMP, one naive idea for aligning the 2nd moment is to calculate the covariance matrix of the aggregated message $M_{v}^{pmp(k+1)}$ for each time $t$ of node $v$ and adjust for the differences. However, when $t=t_{max}$, we cannot directly estimate $\text{var}(M_{v}^{pmp (k+1)})$ since the labels are unknown for nodes in the test set. We introduce \PNY and \JJnorm, the methods for adjusting the aggregated message obtained using the \PMP method to achieve invariant property even for the 2nd moment, when the invariance for 1st moment is preserved.

\textbf{\textit{The second moment of aggregated message}}. The \textit{approximate} variance of $M_{v}^{pmp(k+1)}$ can also be calculated rigorously by using the definition of approximate variance in Appendix \ref{apdx:PMP}, as:
\vspace{-4pt}
{\scriptsize
\begin{align}
\hat{\text{var}}(M_{v}^{pmp(k+1)}) = {\sum_{\tilde{y}\in \bold{Y}}\left(\sum_{\tilde{t}\in\bold{T}_{t}^{\text{single}}}4\mathcal{P}_{yt}\left(\tilde{y}, \tilde{t}\right)+\sum_{\tilde{t}\in\bold{T}_{t}^{\text{double}}}\mathcal{P}_{yt}\left(\tilde{y}, \tilde{t}\right)\right)\Sigma_{XX}^{pmp(k)}\left(\tilde{y},\tilde{t}\right)
\over
\left(\sum_{\tilde{y}\in \bold{Y}}\sum_{\tilde{t}\in\bold{T}_{t}^{\text{single}}}2\mathcal{P}_{yt}\left(\tilde{y}, \tilde{t}\right)+\sum_{\tilde{y}\in \bold{Y}}\sum_{\tilde{t}\in\bold{T}_{t}^{\text{double}}}\mathcal{P}_{yt}\left(\tilde{y}, \tilde{t}\right)\right)^2|\mathcal{N}_{yt}|}
\end{align}
}%

\vspace{-5pt}
Hence, we can write $\hat{\text{var}}(M_{v}^{pmp(k+1)})$=$\Sigma^{pmp(k+1)}_{MM}(y,t)$. Since $\Sigma^{pmp(k+1)}_{MM}(y,t)$ is a covariance matrix, it is positive semi-definite, orthogonally diagonalized as $\Sigma^{pmp(k+1)}_{MM}(y,t) = U_{y t}\Lambda_{y t}U_{y t}^{-1}$.


\subsection{Persistent Numerical Yield: \PNY}
\vspace{-5pt}

If we can specify $\mathcal{P}_{y t}(\tilde{y}, \tilde{t})$ for $\forall y, \tilde{y} \in \bold{Y}, \forall t, \tilde{t} \in \bold{T}$, transformation of covariance matrix during the \PMP process could be calculated. \PNY numerically estimates the transformation of the covariance matrix during the \PMP process, and determines an affine transformation to correct this variation.


\begin{definition}\label{def:env1}
The \PNY from the $k$-th layer to the $k+1$-th layer of target node $v$ is defined as:

For affine transformation matrix $A_{t} = U_{y t_{max}} \Lambda_{y t_{max}}^{1/2} \Lambda_{y t}^{-1/2} U_{y t}^{\top}$, \\

\vspace{-23pt}
\begin{align}
M_v^{PNY(k+1)} = A_{t}(M_v^{pmp (k+1)}-\mu_{M}^{pmp(k+1)}(y))+\mu_{M}^{pmp(k+1)}(y)
\end{align}

\vspace{-12pt}
\end{definition}

Note that $M_v^{pmp (k+1)}$ is a random vector defined as \ref{eqn:pmp}, so $M_v^{PNY(k+1)}$ is also a random vector.

\begin{theorem}\label{thm:pny}
The 1st and 2nd moments of aggregated message after \PNY transform is invariant, if the 1st and 2nd moments of previous representations are invariant.\\

\vspace{-10pt}

\textbf{Sketch of proof} $\hat{\mathbb E}\left[{M_{ v}^{PNY(k+1)}}\right]$=$\mu_{M}^{pmp(k+1)}(y)$, $\hat{\text{var}}(M_{v}^{pmp(k+1)})$=$\Sigma^{pmp(k+1)}_{MM}(y,t_{max})$ holds,

so the 1st and 2nd moments of representations are invariant with $t$. See Appendix \ref{apdx:PNY} for details.
\end{theorem}


\subsection{Junction and Junction normalization: \JJnorm}
A drawback of \PNY is its complexity in handling covariance matrices, requiring computation of covariance matrices and diagonalization for each label and time of nodes, leading to high computational overhead. Additionally, estimation of $\mathcal{P}_{yt}\left(\tilde{y},\tilde{t}\right)$ when $t$ or $\tilde{t}$ is $t_{max}$, necessitates solving overdetermined nonlinear systems of equations as Appendix \ref{apdx:rel_con}, making it difficult to analyze.

% The function $g\left(y, \tilde{y}, |\tilde{t}-t|\right)$ represents how the proportion of neighboring nodes varies with the relative time difference, Assuming it to be consistent to $y$ and $\tilde{y}$ significantly simplifies the alignment of the 2nd order moment. Here, we introduce \JJnorm as a practical implementation of this idea. In \JJnorm, we introduce an additional Assumption 4:
Assuming the function $g\left(y, \tilde{y}, |\tilde{t}-t|\right)$ to be consistent to $y$ and $\tilde{y}$ significantly simplifies the alignment of the 2nd moment. Here, we introduce \JJnorm as a practical implementation of this idea.


\vspace{-15pt}
\begin{align}\label{assumption4}
    \textit{Assumption 4}: g\left(y, \tilde{y}, \Delta \right) = g\left(y', \tilde{y}', \Delta \right) , \forall y, \tilde{y}, y', \tilde{y}' \in \bold{Y}, \forall\Delta \in \{|t_2 -t_1 | \mid t_1, t_2\in \bold{T}\}
\end{align}

\vspace{-5pt}

Moreover, we will only consider GNNs with linear semantic aggregation functions. Formally,

\vspace{-15pt}

\begin{align}
&M_v^{pmp(k+1)} \leftarrow \text{\PMP}(X_w^{pmp(k)},w\in \mathcal{N}_v)\\
&X_v^{pmp(k+1)} \leftarrow A^{(k+1)}M_v^{pmp(k+1)}, \ \forall k<K, v\in \bold{V}
\end{align}

\vspace{-5pt}


\begin{lemma}\label{lem:jj}
$\forall t \in \bold{T}$, there exists a constant $\alpha_{t}^{(k+1)}$>$0$ only depending on $t$ and layer $k+1$ such that

\vspace{-20pt}
\begin{align}
    \left(\alpha_{t}^{(k+1)}\right)^2\Sigma^{pmp(k+1)}_{MM}(y,t)=\Sigma^{pmp(k+1)}_{MM}(y,t_{max}), \forall y \in \bold{Y}
\end{align}
\vspace{-15pt}

The covariance matrix of the aggregated message differs only by a constant factor depending on the layer $k$ and time $t$. Proof of this lemma is in Appendix \ref{apdx:JJnormlemma}.

\end{lemma}

% \begin{figure}[hbt!]
%     \vspace{-10pt}
% 	\centering
% 	\includegraphics[width=0.99\textwidth]{figs/JJ_norm_hor.png}
% 	\vspace{-0.1in}
% 	\caption{Graphical explanation of \JJnorm. Under assumption 4, covariance matrices of aggregated message on each community differs only by a constant factor $\alpha_t$.}
% 	 \label{fig:JJ}
%   \vspace{-10pt}
% \end{figure}


\begin{definition}\label{lem:def1}
We define the constant of the final layer $\alpha_{t}$=$\alpha_{t}^{(K)} > 0$ as the JJ constant of node $v$. 

\end{definition}


\begin{definition}\label{lem:def2}
The \JJnorm is a normalization of the aggregated message to node $v \in \bold{V}\setminus\bold{V}_{\cdot,t_{max}}$ after the final layer of \PMP defined as:

\vspace{-20pt}

\begin{align}
M_v^{JJ} = \alpha_{t}\left(M_v^{pmp(K)}-\mu_M^{JJ}(y, t)\right)+\mu_M^{JJ}(y, t)
\end{align}
\vspace{-10pt}

where $\bold{V}_{y,t} = \{u \in \bold{V} \mid u \; \text{has label} \; y \; \text{and time} \; t\}$, $\bold{V}_{\cdot,t} = \{u \in \bold{V} \mid u \; \text{has time} \; t\}$, $\mu_M^{JJ}(y, t)={1\over {\mid \bold{V}_{y,t} \mid}}\sum_{w\in \bold{V}_{y,t}}M_w^{pmp(K)}$, $\mu_M^{JJ}(\cdot, t)={1\over {\mid \bold{V}_{\cdot,t} \mid}}\sum_{w\in \bold{V}_{\cdot,t}}M_w^{pmp(K)}$, and $\alpha_t$ is the JJ constant.

\vspace{-10pt}

\end{definition}

The 1st and 2nd moments of aggregated message processed by \JJnorm, $\hat{\hat{\mathbb E}}\left[M_v^{JJ}\right]$ and $\hat{\hat{\text{var}}}\left(M_v^{JJ}\right)$, are defined differently from the definition above. Refer to Appendix \ref{apdx:moments_in_jjnorm} for details.

\begin{theorem}\label{thm:jj}
The 1st and 2nd moments of aggregated message processed by \JJnorm is invariant.

\vspace{-5pt}

\textbf{Sketch of proof} We can calculate $\hat{\hat{\mathbb E}}\left[M_v^{JJ}\right]=\mu_{M}^{pmp(K)}(y)$, and $\hat{\hat{\text{var}}}\left(M_v^{JJ}\right)=\Sigma^{pmp(K)}_{MM}(y,t_{max})$.

% \vspace{-15pt}
% \begin{align}
% \hat{\hat{\mathbb E}}\left[M_v^{JJ}\right]=\mu_{M}^{pmp(K)}(y) - \sum_{\tilde{y}\in \bold{Y}}P(\tilde{y})\mu_{M}^{pmp(K)}(\tilde{y})\\
% \hat{\hat{\text{var}}}\left(M_v^{JJ}\right)=\Sigma^{pmp(K)}_{MM}(y,t_{max})
% \end{align}
% \vspace{-10pt}

So the 1st and 2nd moments of aggregated messages are invariant. See Appendix \ref{apdx:moments_in_jjnorm} for details.

\end{theorem}

\vspace{-10pt}
% Unlike \PNY, which involves indirect estimation of $\hat{\mathcal{P}}_{y t}\left(\tilde{y}, \tilde{t}\right)$, \JJnorm provides a more direct and efficient method to obtain an estimate $\hat{\alpha}_{t}$ of $\alpha_{t}$. Refer to Appendix \ref{apdx:JJnorm} for details.
We further present an unbiased estimator $\hat{\alpha}_{t}$ of $\alpha_{t}$. Refer to Appendix \ref{apdx:JJnorm} for derivation.
{\scriptsize
\begin{align}
\hat{\alpha}_t = { {1\over \mid{\bold{V}_{\cdot,t_{max}}}\mid-1}\sum\limits_{i\in \bold{V}_{\cdot,t_{max}}}(M_v^{pmp(K)}-\mu_M^{JJ}(\cdot,t_{max}))^2  -{1\over \mid{\bold{V}_{\cdot,t}}\mid-1} \sum\limits_{y\in \bold{Y}}\sum\limits_{i \in \bold{V}_{y,t}}(\mu_M^{JJ}(y,t) -\mu_M^{JJ}(\cdot,t) )^2  \over{1\over \mid{\bold{V}_{\cdot,t}}\mid-1} \sum_{y\in\bold{Y}} \sum_{i \in \bold{V}_{y,t}}(M_v^{JJ}-\mu_M^{JJ}(y,t))^2}
\end{align}
}%
\vspace{-5pt}

% Such a property can be applied in most decoupled GNNs that solely collect topological information linearly, such as in Yang et al. \cite{SeHGNN} or Hu et al. \cite{RpHGNN}. By applying \JJnorm only once at the final representation instead of at each layer's message passing process, alignment of the 2nd moment becomes efficient. This is because the condition for applying \JJnorm at the final representation, $\beta_{t}^{(K)}\Sigma_{XX}^{pmp (K)}(y,t_{max})= \Sigma_{XX}^{pmp (K)}(y,t)$, is automatically satisfied. This property endows \JJnorm, when used in conjunction with \PMP in most decoupled GNNs, with very high scalability and adaptability. As it can obtain invariant information without modifying the message passing function of the baseline model, it enables effective operation with minimal cost when applied to chronological split datasets.


\section{Experiments}
\subsection{Synthetic chronological split dataset}
\textbf{Temporal Stochastic Block Model(TSBM).} To assess the robustness and generalizability of proposed invariant message passing methods on graphs satisfying assumptions 1, 2, and 3, we conducted experiments on synthetic graphs. In order to create repeatable and realistic chronological graphs, we defined the Temporal Stochastic Block Model(TSBM) as our graph generation algorithm. TSBM can be regarded as a special case of the Stochastic Block Model(SBM) that incorporates temporal information of nodes \cite{holland1983stochastic,deshpande2018contextual}. In the SBM, the probability matrix $\bold{P}_{y \tilde y}$ represents the probability of a connection between two nodes $i$ and $j$, where $y$ and $\tilde y$ denote the communities to which the nodes belong. Our study extends this concept to account for temporal information, differentiating communities based on both node labels and time. In the Temporal Stochastic Block Model (TSBM), the connection probability is represented by a 4-dimensional tensor $\bold{P}_{t \tilde t y \tilde y}$. We ensured that Assumptions 1, 2, and 3 were satisfied. Specifically, the feature assigned to each node $\bold{x} \in \mathbb{R}^f$ was sampled from distributions depending solely on the label, defined as $\bold{x}_i = \mu(y) + k_{y} Z_i$. Here, $Z_i \in \mathbb{R}^f$ is an IID standard normal noise and $k_{y}$ represents the variance of features. To satisfy Assumption 2, the time and label of each node were determined independently. To satisfy Assumption 3, we first considered the possible forms of $g(y, \tilde y, |t - \tilde t|)$ and then determined $\bold{P}_{t \tilde t y \tilde y}$ accordingly. We used an exponentially decaying function with decay factor $\gamma_{y, \tilde y}$, defined as:

\vspace{-15pt}
\begin{align}
    g(y, \tilde y, |t - \tilde t|) = \gamma_{y, \tilde y}^{|t - \tilde t|} g(y , \tilde y, 0),\ \forall |t - \tilde t|>0
\end{align}
\vspace{-5pt}

\vspace{-10pt}
\textbf{Experimental Setup.} For our experiments, we employed the simplest and most fundamental decoupled GNN, Simple Graph Convolution (SGC) \cite{SGC}, as the baseline model. Additionally, we investigated whether the methods proposed in this study could improve the performance of general spatial GNNs. Hence, we used a 2-layer Graph Convolutional Network (GCN) \cite{GCN} that performs averaging message passing as another baseline model. We applied the \MMP, \PMP, \PMP+\PNY, and \PMP+\JJnorm methods to the baselines and compared their performance. Here, since the semantic aggregation of GCN is nonlinear, layer-wise \JJnorm was applied, i.e. \JJnorm could not be applied only to the aggregated message in the last layer but was applied to the aggregated message in each layer.  To test the generalizability of \JJnorm, which is based on Assumption 4, we conducted experiments on graphs that both satisfy and do not satisfy this assumption. Furthermore, for cases where Assumption 4 was satisfied, common decay factor $\gamma$ can be defined. A smaller $\gamma$ corresponds to a graph where the connection probability decreases drastically. We also compared the trends in the performance of each \IMPaCT method by varying the value of $\gamma$. Detailed settings are provided in Appendix \ref{apdx:synthetic_setup}. The results on synthetic graphs are presented in Table \ref{table:synthetic} and Figure \ref{fig:synthetic}.% This essentially interprets a multilayer GCN as a cascade of single-layer GCNs.


% For cases where Assumption 4 was not satisfied, $\gamma_{y_i, y_j}$ was sampled from a uniform distribution $[0.4, 0.7]$. For cases where Assumption 4 was satisfied, all decay factors were the same, i.e., $\gamma_{y_i, y_j} = \gamma, \ \forall y_i, y_j \in \bold{Y}$. In this case, $\gamma$ indicates the extent to which the connection probability varies with the time difference between two nodes. A smaller $\gamma$ corresponds to a graph where the connection probability decreases drastically. We also compared the trends in the performance of each \IMPaCT method by varying the value of $\gamma$. The baseline SGC consisted of 2 layers of message passing and 2 layers of MLP, with the hidden layer dimension set to 16. The baseline GCN also consisted of 2 layers with the hidden layer dimension set to 16. Adam optimizer was used for training with a learning rate of $0.01$ and a weight decay of $0.0005$. Each model was trained for 200 epochs, and each data was obtained by repeating experiments on 200 random graph datasets generated through TSBM. For the hyperparameters $\mathcal{K}$ and $\mathcal{G}$, we used $\mathcal{K} = 0.6$ and $\mathcal{G} = 0.24$. The training of both models was conducted on a 2x Intel Xeon Platinum 8268 CPU with 48 cores and 192GB RAM. 
\begin{figure}[!hbt]
    \vspace{-5pt}
	\centering
	\includegraphics[width=\textwidth]{figs/synthetic_hor.png}
	\caption{ The left graphs show the performance gain of \IMPaCT over the baseline. The right graphs illustrate the gain of 2nd moment alignment methods over the 1st moment alignment method \PMP.}
	 \label{fig:synthetic}
  \vspace{-10pt}
\end{figure}


\vspace{3pt}
\subsection{Real world chronological split dataset}

To assess the performance of invariant message passing on real-world datasets, we utilized the ogbn-mag dataset from the OGB \cite{OGB}. The statistics of ogbn-mag dataset is summarized in Appendix \ref{apdx:toy_experiment}. We employed CLGNN \cite{CLGNN} as the baseline model, which is based on RpHGNN \cite{RpHGNN} and incorporates a curriculum learning approach. RpHGNN, a decoupled GNN, effectively resolves the trade-off between the amount of information and computational overhead in message passing by random projection squashing technique. Since overall semantic aggregation processes in RpHGNN are linear, \JJnorm can be applied. However, due to the immense size of the graph, the application of \PNY was challenging. Therefore, we applied the \MMP, \PMP, and \PMP+\JJnorm to baseline and compared the performance. Each experiment was repeated 9 times, and the hyperparameters were set same as those in Wong et al. \cite{CLGNN}. Entire training process took 5h 42m, on a Tesla P100 GPU machine with 28 Intel Xeon 2680 CPUs and 128GB of RAM. Results are shown in Table \ref{table:ogbn-mag}. 

%The ogbn-mag dataset features a chronological split and consists of a heterogeneous graph. The task for ogbn-mag graph involves classifying the venue of papers (conference or journal) into 349 different classes. The graph comprises four types of nodes, including papers, authors, institutions, and fields of study, along with four distinct types of edges. 
% In heterogeneous graphs, not all nodes may possess temporal information. We can only apply invariant message passing methods when message passing occurs between nodes of types with temporal information.

% In the case of \MMP, learning was unstable in some cases. Therefore, both results excluding cases of failure in convergence and statistics including cases of failure in convergence were presented. 
% Preprocessing was performed on a 48 core Intel Xeon Platinum 8268 CPU with 768GB of RAM. Training took place on a Tesla P100 GPU with 28 Intel Xeon E5-2680 CPUs and 128GB of RAM. 




% The time complexity of \IMPaCT methods is as shown in Table \ref{table:scalability}. For detailed analyses, refer to Appendix \ref{apdx:scalability}. All methods exhibit linear complexity with respect to the number of nodes and edges. When employed in Decoupled GNNs, the operations of \IMPaCT methods occur only during preprocessing. When applied to general spatial GNNs, however, for \PNY and \JJnorm, the scalability may vary depends on the $f$, $|\bold{Y}|$, $|\bold{T}|$, and type of baseline model. 
% \JJnorm, when used in conjunction with \PMP in most decoupled GNNs, with very high scalability and adaptability. As it can obtain invariant information without modifying the message passing function of the baseline model, it enables effective operation with minimal cost when applied to chronological split datasets.

% In particular, \PMP demonstrated superior performance in experiments using both real and synthetic graphs, surpassing the baseline by a significant margin. It offers a method applicable with a single operation for reconstructing the graph, regardless of the model, if the graph can be reconstructed. Therefore, it exhibits good adaptability and scalability, robustly addressing domain adaptation induced by chronological split. Thus, if the graph exhibits differences in environments due to temporal information, we recommend starting with \PMP to make the representation's 1st moment invariant during training.

% \PMP demonstrated superior performance in experiments using both real and synthetic graphs, and it offers a method applicable with a single operation for reconstructing the graph. Other guidelines for selecting \IMPaCT models in chronological split dataset are provided in Appendix \ref{apdx:scalability}.


\subsection{Results}


\begin{wraptable}{R}{0.55\textwidth}
    \vspace{0pt}
        \fontsize{8}{8.4}\selectfont
        \centering
        \caption{The results on the ogbn-mag dataset. All data was obtained by averaging the results from nine different seeds. The symbol \textdagger denotes statistics obtained after excluding training results that failed to converge.}
    
        \begin{tabular}{ccc}
        \hline
        Methods                      & Valid                                  & Test                                   \\ \hline
        {MLP}   & {0.2626 �� 0.0016} & {0.2692 �� 0.0026} \\ 
        {SeHGNN+ComplEx}    & {0.5917 �� 0.0009} & {0.5719   �� 0.0012} \\ 
        {RpHGNN+LP+CR+LINE} & {0.5973 �� 0.0008} & {0.5773 �� 0.0012}   \\
        {CLGNN(baseline)} & {0.8021 �� 0.0020} & {0.7956 �� 0.0047} \\ \hline
        CLGNN+MMP\textdagger                   & {0.7603 �� 0.0397} & {0.7601 �� 0.0389} \\
        CLGNN+MMP                    & {0.3056 �� 0.3920} & {0.3047 �� 0.3926} \\ 
        CLGNN+PMP                    & {0.8537 �� 0.0049} & {0.8497 �� 0.0048} \\ 
        \textbf{CLGNN+PMP+JJnorm}             & \textbf{0.8608 �� 0.0049} & \textbf{0.8569 �� 0.0059} \\ \hline
        \end{tabular}
        \vspace{3pt}
    
        \label{table:ogbn-mag}
        \vspace{-10pt}
    \end{wraptable}


\begin{table}[!h]
    \fontsize{7}{8.4}\selectfont
    \centering
    \vspace{-15pt}
    \caption{Prediction accuracy and training time on synthetic graphs generated by TSBM. "Fixed $\gamma_{y_i y_j}$" represents scenarios satisfying Assumption 4, while "Random $\gamma_{y_i y_j}$" represents scenarios that do not. Time is reported for the entire training process over 200 epochs.}

    \begin{tabular}{l|lllll|lllll}
\hline
           & \multicolumn{5}{l|}{SGC}                  & \multicolumn{5}{l}{GCN}                   \\ \hline
           & Baseline & MMP   & PMP   & PNY   & JJnorm & Baseline & MMP   & PMP   & PNY   & JJnorm \\ \hline
Fixed $\gamma_{y_i y_j}$  & 0.2243 & 0.15   & 0.2653 & 0.2758 & \textbf{0.2777} & 0.2035 & 0.1439 & 0.2245 & 0.2178          & \textbf{0.2311} \\ \hline
Random $\gamma_{y_i y_j}$ & 0.1331 & 0.1063 & 0.1854 & 0.1832 & \textbf{0.1862} & 0.1298 & 0.1022 & 0.1565 & \textbf{0.1613} & 0.1609          \\ \hline
Time (sec) & 0.325    & 0.315 & 0.303 & 1.178 & 0.538  & 0.771    & 0.728 & 0.773 & 268.5 & 448.81 \\ \hline
\end{tabular}

    \label{table:synthetic}
    \vspace{-5pt}
\end{table}


\textbf{Experimental results.} Our experiments using both real and synthetic graphs demonstrated the superior performance of \IMPaCT methods. By employing \PMP+\JJnorm, we achieved a significant performance improvement of 6.1\% over the current state-of-the-art methods on the ogbn-mag dataset. In 1st moment alignment, \PMP consistently outperformed the baseline model significantly, irrespective of the settings. Specifically, on the ogbn-mag dataset, applying \PMP resulted in a 5.1\% increase in accuracy compared to the baseline. In the experiments with synthetic graphs, \PMP provided a performance gain of 4.7\% with SGC and 2.4\% with GCN over their baselines. 2nd moment alignment methods generally performed better than 1st moment alignment methods alone. In ogbn-mag, \JJnorm outperformed \PMP by 1.0\%. In synthetic datasets, \JJnorm almost always performed better than \PNY, except in cases where Assumption 4 did not hold and the baseline model was GCN. Experimental results further support the generalizability of our methods. Even when using general spatial GNNs as the baseline model, \IMPaCT provided significant performance gains. Additionally, \JJnorm improved performance over \PMP even when Assumption 4 did not hold. %We present guidelines for selecting \IMPaCT models in chronological split datasets in Appendix \ref{apdx:scalability}.

\begin{table}[]
    \vspace{-15pt}
    \fontsize{7}{8.4}\selectfont
    %\vspace{-10pt}

    \center
    \caption{Scalability of \IMPaCT methods. "Graph." indicates method is applied by graph reconstruction. $N_c=|\bold{Y}||\bold{T}|$, $R$ is number of training epochs, and \JJnorm\textdagger indicates layer-wise \JJnorm.}

    \begin{tabular}{lllll}
    \hline
     &
      \multicolumn{2}{l}{{ Decoupled GNN}} &
      \multicolumn{2}{l}{{ General spatial GNN}} \\ \cline{2-5} 
    \multirow{-2}{*}{Method} &
      { Graph.} &
      { Message passing} &
      { Graph.} &
      { Message passing} \\ \hline
    \MMP &
      { $\mathcal{O}(|E|)$} &
      { $\mathcal{O}(|E|fK)$} &
      { $\mathcal{O}(|E|)$} &
      { $\mathcal{O}(R|E|fK)$} \\ \hline
    \PMP &
      { $\mathcal{O}(|E|)$} &
      { $\mathcal{O}(|E|fK)$} &
      { $\mathcal{O}(|E|)$} &
      { $\mathcal{O}(R|E|fK)$} \\ \hline
    \PNY &
      { -} &
      { $\mathcal{O}(Kf^2(N_cf+N_c^2+N)+|E|f)$} &
      { -} &
      { $\mathcal{O}(RKf^2(N_cf+N_c^2+N)+R|E|f)$} \\ \hline
    \JJnorm &
      { -} &
      { $\mathcal{O}(Nf)$} &
      { -} &
      { $\mathcal{O}(RNf)$} \\
    \JJnorm \textdagger&
      { -} &
      { $\mathcal{O}(NfK)$} &
      { -} &
      { $\mathcal{O}(RNfK)$} \\ \hline
    \end{tabular}

    \vspace{3pt}
    \label{table:scalability}
    \vspace{-14pt}

\end{table}
\textbf{Scalability.} The time complexity of \IMPaCT methods is shown in Table \ref{table:scalability}. For detailed analyses, refer to Appendix \ref{apdx:scalability}. All methods exhibit linear complexity with respect to the number of nodes and edges. When employed in Decoupled GNNs, the operations of \IMPaCT methods occur only during preprocessing. Specifically, \PMP and \JJnorm can be implemented with only graph modification and final representation correction, thus offering not only very high scalability but also adaptability. However, when applied to general spatial GNNs, the operations of \IMPaCT are multiplied by the number of epochs, making it challenging to consistently maintain scalability for \PNY and \JJnorm. Nevertheless, since most operations of \IMPaCT can be parallelized, the actual training time can vary significantly depending on the implementation.


\section{Conclusions}
In this study, we tackle the domain adaptation challenge in graph data induced by chronological splits by proposing invariant message passing functions, \IMPaCT. We effectively analyzed and addressed the domain adaptation problem in graph datasets with chronological splits, by presenting robust and realistic assumptions derived from observable properties in real-world graphs. Based on the assumptions we propose the \IMPaCT, to preserve the invariance of the 1st moment and 2nd moment of aggregated message during the message passing step, and demonstrate its adaptability and scalability. We validate the efficacy of our approach both on real-world citation graph and synthetic graph. Particularly on the citation graph ogbn-mag, we achieved a substantial improvement over the previous state-of-the-art, with a significant margin of 5.4\% when applying only 1st moment alignment method, and 6.1\% when combining 1st and 2nd moment alignment method.

\textbf{Limitations.} Our theoretical discussions were restricted to Spatial GNNs and assume that semantic aggregation satisfies the G-Lipschitz condition. Additionally, \IMPaCT does not guarantee that the invariant representations obtained are maximally informative. There is also a need to demonstrate the robustness and generalizability of \IMPaCT through experiments on a wider variety of baseline models. Lastly, the interpretation of experimental results for 2nd moment alignment methods was insufficient. Notably, when SGC was the baseline, 2nd order alignment methods saw increased performance gains with larger decay factors, whereas the opposite trend was observed when GCN was the baseline. These limitations should be addressed in future research.

% Our research fills a critical gap in effectively analyzing and addressing the domain adapatation problem in massive graph datasets, an area previously underexplored. 






\clearpage
\bibliographystyle{abbrvnat}
\bibliography{reference}

%%%%%%%%%%%%%%%%%%%%%%%%%%%%%%%%%%%%%%%%%%%%%%%%%%%%%%%%%%%%
% \clearpage
% \section*{Checklist}
% \input{99-Checklist}
%%%%%%%%%%%%%%%%%%%%%%%%%%%%%%%%%%%%%%%%%%%%%%%%%%%%%%%%%%%%

\appendix
\clearpage
\section{Appendix}

\subsection{Motivation for Assumptions}\label{apdx:assumptions}
In this study, we make three assumptions regarding temporal graphs.
\begin{align}
    \textit{Assumption 1}: &P_{e^{te}}(Y) = P_e(Y), \ \forall e \in \epsilon^{tr}\\
    \textit{Assumption 2}: &P_{e^{te}}(X\mid Y) = P_e(X\mid Y), \ \forall e \in \epsilon^{tr}\\
    \textit{Assumption 3}: &\mathcal{P}_{y t} (\tilde y ,\tilde t) = f(y, t) g(y, \tilde y, \mid \tilde t-t\mid), \ \forall t, \tilde t \in \bold{T}, y, \tilde y\in \bold{Y}
\end{align}
% These assumptions are rooted in properties observable in real-world graphs. For instance, in the academic paper citation graph utilized in this study, labels represent the categories of papers, while features comprise vector embeddings of paper abstracts. While the joint distribution of paper categories and abstracts may remain stable with minor temporal changes, the probability of two papers being linked via citation decreases significantly with the temporal gap between them. Hence, in citation graphs, the probability distribution of connections between nodes evolves much more sensitively to time than to features or labels.

Combining Assumption 1 and Assumption 2, we derive $P_{e^{te}}(X, Y) = P_e(X, Y), \ \forall e \in \epsilon^{tr}$. This is a fundamental assumption in machine learning problems, implying that the initial feature distribution does not undergo significant shifts. However, in the real world graph dataset such as ogbn-mag dataset, features are embeddings derived from the abstracts of papers using a language model. It is crucial to verify whether these features remain constant over time, as per our assumption. Assumption 3 is a specific assumption derived from a close observation of the characteristics of temporal graphs, which requires justification based on actual data. To address these questions, we conducted a visual analysis based on the real-world temporal graph data from the ogbn-mag dataset. Statistics for ogbn-mag are provided in Appendix \ref{apdx:toy_experiment}.

\subsubsection{Invariance of initial features}
First, to verify whether the distribution of features change over time, we calculated the average node features for each community, i.e., for each unique (label, time) pair. Our objective was to demonstrate that the distance between mean features of nodes with the same label but different times is significantly smaller than the distance between mean features of nodes with different labels. Given that the features are high-dimensional embeddings, using simple norms as distances might be inappropriate. Therefore, we employed the unsupervised learning method t-SNE \cite{van2008visualizing} to project these points onto a 2D plane, verifying that nodes with the same label but different times form distinct clusters. 
For the t-SNE analysis, we set the maximum number of iterations to 1000, perplexity to 30, and learning rate to 50.

We computed the mean feature for each community defined by the same (label, time) pair. Points corresponding to communities with the same label are represented in the same color. Thus, there are $|\bold{T}|$ points for each color, resulting in a total of $|\bold{Y}||\bold{T}|$ points in the left part of figure \ref{fig:tsne}.

Given that the number of labels is $|\bold{Y}|=349$, it is challenging to discern trends in a single graph displaying all points. The figure on the right considers only the 15 labels with the most nodes, redrawing the graph for clarity.

\begin{figure}[hbt!]
	\centering
	\includegraphics[width=0.80\textwidth]{figs/tsne.png}
	\vspace{-0.1in}
	\caption{2D projection of each community's mean feature by t-SNE. Points corresponding to communities with the same label are represented in the same color. [Left] Plot for all 349 labels, [Right] Plot for 15 labels with the most nodes.}
	 \label{fig:tsne}
\end{figure}

The clusters of nodes with the same color are clearly identifiable. While this analysis only consider 1st moment of initial feature of nodes, and does not confirm invariance for statistics other than the mean, it does show that the distance between mean features of nodes with the same label but different times is much smaller than the distance between mean features of nodes with different labels.



\subsubsection{Motivation for Assumption 3}

Assumption 3 posits the separability of relative connectivity. Verifying this hypothesis numerically without additional assumptions about the connection distribution is challenging. Therefore, we aim to motivate Assumption 3 through a visualization of relative connectivity.

Consider fixing $ y $ and $ \tilde y $, and then examining the estimated relative connectivity $\mathcal{P}_{y, t} (\tilde y, \tilde t)$ as a function of $ t $ and $ \tilde t $. Since $\mathcal{P}_{y, t} (\tilde y, \tilde t) = f(y, t) g(y, \tilde y, \mid \tilde t - t \mid)$, the graph of $\mathcal{P}_{y, t} (\tilde y, \tilde t)$ for different $ t $ should have similar shapes, differing only by a scaling factor determined by $ f(y, t) $. In other words, by appropriately adjusting the scale, graphs for different $ t $ should overlap.

Given $ |\bold{Y}| = 349 $, plotting this for all label pairs $ y, \tilde y $ is impractical. Therefore we plotted graphs for few labels connected by a largest number of edges, plotting their relative connectivity. Although the ogbn-mag dataset is a directed graph, we treated it as undirected for defining neighboring nodes.

Graphs in different colors represent different target node times $ t $, with the X-axis showing the relative time $ \tilde t - t $ for neighboring nodes. Nodes with times $ t = 2018 $ and $ t = 2019 $ were excluded since they belong to the validation and test datasets, respectively. The data presented in graph \ref{fig:ERC} are the unscaled relative connectivity.

While plotting these graphs for all label pairs is infeasible, we calculated the weighted average relative connectivity for cases where $ y = \tilde y $ and $ y \neq \tilde y $ to understand the overall distribution. Specifically, for each $ t $, we plotted the following values:
\begin{figure}[hbt!]
	\centering
	\includegraphics[width=0.99\textwidth]{figs/ERC.png}
	\vspace{-0.1in}
	\caption{Estimated relative connectivity. [Left] when $y=1$ and $\tilde y=1$, [Right] when $y=311$ and $\tilde y=1$.}
	 \label{fig:ERC}
\end{figure}

\begin{align}
P_{\text{Same label}}(t, \tilde t) = \frac{|\{(u, v) \in E \mid u,v \text{ have same label}, u\text{ has time } t, v \text{ has time } \tilde t\}|}{|\{(u, v) \in E \mid u,v \text{ have same label}\}|}
\end{align}

\begin{align}
P_{\text{Diff label}}(t, \tilde t) = \frac{|\{(u, v) \in E \mid u,v \text{ have different label}, u\text{ has time } t, v\text{ has time }\tilde t\}|}{|\{(u, v) \in E \mid u,v \text{ have different label}\}|}
\end{align}

These statistics represent the weighted average relative connectivity for each $ y, t $ pair, weighted by the number of communities defined by each $( y, t )$ pair. Data for $ t = 2018 $ and $ t = 2019 $ were excluded, and no scaling corrections were applied. 
\begin{figure}[hbt!]
	\centering
	\includegraphics[width=0.99\textwidth]{figs/WARC.png}
	\vspace{-0.1in}
	\caption{Weighted average relative connectivity. [Left] when $y=\tilde y$, [Right] when $y\neq \tilde y$.}
	 \label{fig:WARC}
\end{figure}

The graphs \ref{fig:ERC}, \ref{fig:WARC} reveal that the shape of graphs for different $ t $ are similar and symmetric, supporting Assumption 3. Although this analysis is not a formal proof, it serves as a necessary condition that supports the validity of the separability assumption.





\subsection{Toy experiment}
\label{apdx:toy_experiment}
The purpose of toy experiment was to compare test accuracy obtained when dataset was split chronologically and split randomly regardless of time information. We further investigated whether incorporating temporal information in the form of time positional encoding significantly influences the distribution of neighboring nodes.

We conduct this toy experiment on ogbn-mag, a chronological heterogeneous graph within the Open Graph Benchmark \cite{OGB}, comprising Paper, Author, Institution, and Fields of study nodes. Only paper nodes feature temporal information. Detailed node and edge statistics of ogbn-mag dataset are provided in table \ref{table:edge} and \ref{table:node}. In this graph, paper nodes are divided into train, validation, and test nodes based on publication year, with the objective of classifying test and validation nodes into one of 349 labels. The performance metric is accuracy, representing the proportion of correctly labeled nodes among all test nodes. 

Initial features were assigned only to paper nodes. In the chronological split dataset, nodes from the year 2019 were designated as the test set, while nodes from years earlier than 2018 were assigned to the training set. Time positional embedding was implemented using sinusoidal signals with 20 floating-point numbers, and these embeddings were concatenated with the initial features.



\begin{table}[!h]
    \small
        \vspace{-10pt}
        \centering
        \caption{Type and number of nodes in obgn-mag.}
    
        \begin{tabular}{llll}
        \hline
        \multicolumn{1}{l|}{Node   type}                        & \#Train nodes                 & \#Validation nodes  & \#Test nodes \\ \hline
        \multicolumn{1}{l|}{Paper}         & {59,965} & {64,879} & 41,939 \\ \hline
        \multicolumn{1}{l|}{Author}      & {1,134,649} & {0} & 0               \\ \hline
        \multicolumn{1}{l|}{Institution} & {8,740}     & {0} & 0               \\ \hline
        \multicolumn{1}{l|}{Field of study} & {59,965} & {0}      & 0      \\ \hline
        \end{tabular}
        \label{table:node}
    
    \end{table}
\begin{table}[!h]
    \small
        \vspace{-10pt}
        \centering
        \caption{Type and number of edges in obgn-mag. * indicates the type of edges connect nodes with temporal information.}
    
        \begin{tabular}{cccc}
        \hline
        {Source   type} & {Edges type}     & \multicolumn{1}{c|}{Destination   type} & \#Edges   \\ \hline
        {Author} & {affiliated with} & \multicolumn{1}{c|}{Institution} & 1,043,998 \\ 
        {Author} & {writes}          & \multicolumn{1}{c|}{Paper}       & 7,145,660 \\ 
        Paper         & cites*         & \multicolumn{1}{c|}{Paper}              & 5,416,271 \\ 
        Paper         & has a topic of & \multicolumn{1}{c|}{Fields of study}    & 7,505,078 \\ \hline
        \end{tabular}
        \label{table:edge}
    
    \end{table}
    

SeHGNN \cite{SeHGNN} was employed as baseline model for experimentation. The rationale for employing SeHGNN lies in its ability to aggregate semantics from diverse metapaths, thereby ensuring expressiveness, while also enabling fast learning due to neighbor aggregation operations being performed only during preprocessing. Each experiment was conducted four times using different random seeds. The hyperparameters and settings used in the experiments were identical to those presented by Yang et al. \cite{SeHGNN}.

Preprocessing was performed on a 48 core 2X Intel Xeon Platinum 8268 CPU machine with 768GB of RAM. Training took place on a NVIDIA Tesla P100 GPU machine with 28 Intel Xeon E5-2680 V4 CPUs and 128GB of RAM.




\subsection{Proofs of theorems in Section \ref{sec:rel}}\label{apdx:proofs}
\subsubsection{Proof of Theorem \ref{thm:env1}}
\input{proofs/theorem-risk1}

\subsubsection{Proof of Theorem \ref{thm:env2}}
\input{proofs/theorem-risk2}







\subsection{First and Second moment of averaging message passing}\label{apdx:firstmm}
\subsubsection{First moment as approximate of expectation}
We define the first moment of averaging message as the following steps: \\

(a) Take the expectation of the averaged message. \\
(b) Approximate $|\mathcal{N}_{v}\left(\tilde{y}, \tilde{t}\right)|$ as $P_{yt}\left(\tilde{y}, \tilde{t}\right)|\mathcal{N}_{v}|$ until the discrete values, i.e., the number of elements terms $|\mathcal{N}_{v}\left(\tilde{y}, \tilde{t}\right)|$ and $|\mathcal{N}_{v}|$ disappear. \\

Because of step (b), we are defining the "approximate of expectation" as the first moment of a message. Denote the first moment of averaged message ${M_{ v}^{(k+1)}}$ as $\hat{\mathbb E}[{M_{ v}^{(k+1)}}]$. Deliberate calculations are as follows:
\begin{align}
    \hat{\mathbb E}[{M_{ v}^{(k+1)}}] &\overset{(a)}{=} {\mathbb E}\left[{\sum_{\tilde{y}\in \bold{Y}}\sum_{\tilde{t}\in\bold{T}}\sum_{w\in\mathcal{N}_{v}\left(\tilde{y}, \tilde{t}\right)}X_w \over \sum_{\tilde{y}\in \bold{Y}}\sum_{\tilde{t}\in\bold{T}} |\mathcal{N}_{v}\left(\tilde{y}, \tilde{t}\right)|}\right]\\&\overset{(b)}{=}
    \mathbb{E}\left[{\sum_{\tilde{y}\in \bold{Y}}\sum_{\tilde{t}\in\bold{T}}\sum_{w\in\mathcal{N}_{v}\left(\tilde{y}, \tilde{t}\right)}X_w \over \sum_{\tilde{y}\in \bold{Y}}\sum_{\tilde{t}\in\bold{T}} P_{yt}\left(\tilde{y}, \tilde{t}\right)|\mathcal{N}_{v}|}\right]\\&=
    {1 \over |\mathcal{N}_{v}|}\mathbb{E}\left[{\sum_{\tilde{y}\in \bold{Y}}\sum_{\tilde{t}\in\bold{T}}\sum_{w\in\mathcal{N}_{v}\left(\tilde{y}, \tilde{t}\right)}X_w}\right]\\&=
    {1 \over |\mathcal{N}_{v}|}\sum_{\tilde{y}\in \bold{Y}}\sum_{\tilde{t}\in\bold{T}}\mathbb{E}\left[\sum_{w\in\mathcal{N}_{v}}\left(\tilde{y}, \tilde{t}\right)X_w\right]\\&=
    {1 \over |\mathcal{N}_{v}|}\sum_{\tilde{y}\in \bold{Y}}\sum_{\tilde{t}\in\bold{T}}|\mathcal{N}_{v}\left(\tilde{y}, \tilde{t}\right)|\mu_{X}^{(k)}\left(\tilde{y}\right)\\&=
    \sum_{\tilde{y}\in \bold{Y}}\sum_{\tilde{t}\in\bold{T}}{|\mathcal{N}_{v}\left(\tilde{y}, \tilde{t}\right)| \over |\mathcal{N}_{v}|}\mu_{X}^{(k)}\left(\tilde{y}\right)\\&\overset{(b)}{=}
    \sum_{\tilde{y}\in \bold{Y}}\sum_{\tilde{t}\in\bold{T}}\mathcal{P}_{y t}\left(\tilde{y}, \tilde{t}\right)\mu_{X}^{(k)}(\tilde{y})
\end{align}

The final term of the equation above does not incorporate any discrete values $|\mathcal{N}_{v}\left(\tilde{y}, \tilde{t}\right)|$ and $|\mathcal{N}_{v}|$, so the step ends.

Note that we can calculate the first moment reversely as follows:
\begin{align}
    M_{ v}^{(k+1)} &= {\sum_{\tilde{y}\in \bold{Y}}\sum_{\tilde{t}\in\bold{T}}\sum_{w\in\mathcal{N}_{v}\left(\tilde{y}, \tilde{t}\right)}X_w \over \sum_{\tilde{y}\in \bold{Y}}\sum_{\tilde{t}\in\bold{T}} |\mathcal{N}_{v}\left(\tilde{y}, \tilde{t}\right)|} \\&=   {\sum_{\tilde{y}\in\bold{Y}}\sum_{\tilde{t}\in\bold{T}}{|\mathcal{N}_{v}\left(\tilde{y}, \tilde{t}\right)| \over |\mathcal{N}_{v}|}\sum_{w\in\mathcal{N}_{v}\left(\tilde{y}, \tilde{t}\right)}{X_w \over |\mathcal{N}_{v}\left(\tilde{y}, \tilde{t}\right)|} \over \sum_{\tilde{y}\in \bold{Y}}\sum_{\tilde{t}\in\bold{T}} {|\mathcal{N}_{v}\left(\tilde{y}, \tilde{t}\right)| \over |\mathcal{N}_{v}|}} \\&\simeq
    {\sum_{\tilde{y}\in\bold{Y}}\sum_{\tilde{t}\in\bold{T}}{\mathcal{P}_{y t}\left(\tilde{y}, \tilde{t}\right)}\sum_{w\in\mathcal{N}_{v}\left(\tilde{y}, \tilde{t}\right)}{X_w \over |\mathcal{N}_{v}\left(\tilde{y}, \tilde{t}\right)|} \over \sum_{\tilde{y}\in \bold{Y}}\sum_{\tilde{t}\in\bold{T}} {\mathcal{P}_{y t}\left(\tilde{y}, \tilde{t}\right)}} \\&=
    \sum_{\tilde{y}\in\bold{Y}}\sum_{\tilde{t}\in\bold{T}}\left({\mathcal{P}_{y t}\left(\tilde{y}, \tilde{t}\right)}\sum_{w\in\mathcal{N}_{v}\left(\tilde{y}, \tilde{t}\right)}{X_w \over |\mathcal{N}_{v}\left(\tilde{y}, \tilde{t}\right)|}\right)
\end{align}
Take the expectation on both sides to derive
\begin{align}
    \mathbb{E}\left[M_{ v}^{(k+1)}\right] &\simeq
	\mathbb{E}\left[\sum_{\tilde{y}\in\bold{Y}}\sum_{\tilde{t}\in\bold{T}}\left({\mathcal{P}_{y t}\left(\tilde{y}, \tilde{t}\right)}\sum_{w\in\mathcal{N}_{v}\left(\tilde{y}, \tilde{t}\right)}{X_w \over |\mathcal{N}_{v}\left(\tilde{y}, \tilde{t}\right)|}\right)\right]\\
	&=\sum_{\tilde{y}\in\bold{Y}}\sum_{\tilde{t}\in\bold{T}}\left({\mathcal{P}_{y t}\left(\tilde{y}, \tilde{t}\right)}\sum_{w\in\mathcal{N}_{v}\left(\tilde{y}, \tilde{t}\right)}{\mathbb{E}[X_w] \over |\mathcal{N}_{v}\left(\tilde{y}, \tilde{t}\right)|}\right)\\
	&=\sum_{\tilde{y}\in\bold{Y}}\sum_{\tilde{t}\in\bold{T}}\left({\mathcal{P}_{y t}\left(\tilde{y}, \tilde{t}\right)}\mu_X^{(k)}(\tilde y)\right)
\end{align}

\subsubsection{Second moment as approximate of variance}\label{apdx:secondmm}

We define the second moment of averaging message as the following steps:\\

(a) Take the variance of the averaged message.\\
(b) Approximate $|\mathcal{N}_v\left(\tilde{y}, \tilde{t}\right)|$ as $\mathcal{P}_{yt}(\tilde{y}, \tilde{t})|\mathcal{N}_v|$ until the discrete value terms $|\mathcal{N}_v\left(\tilde{y}, \tilde{t}\right)|$ disappear.\\

Because of step (b), we are defining the "approximate of variance" as the second moment of a message. Denote the second moment of averaged message $M_v^{(k+1)}$ as $\hat{\text{var}} (M_v^{(k+1)})$. Deliberate calculations are as follows:

\begin{align}
\hat{\text{var}}(M_v^{(k+1)})&\overset{(a)}=\text{var}\left({\sum_{\tilde y \in \bold{Y}}\sum_{\tilde t \in \bold{T}}\sum_{w\in \mathcal{N}_v(\tilde y,\tilde t)}X_w \over \sum_{\tilde y\in \bold{Y}}\sum_{\tilde t\in \bold{T}}|\mathcal{N}_v(\tilde y, \tilde t)|}\right)\\
&\overset{(b)}=\text{var}\left({\sum_{\tilde y \in \bold{Y}}\sum_{\tilde t \in \bold{T}}\sum_{w\in \mathcal{N}_v(\tilde y,\tilde t)}X_w \over \sum_{\tilde y\in \bold{Y}}\sum_{\tilde t\in \bold{T}}\mathcal{P}_{yt}(\tilde y, \tilde t)|\mathcal{N}_v|}\right)\\
&={\sum_{\tilde y \in \bold{Y}}\sum_{\tilde t \in \bold{T}}\sum_{w\in \mathcal{N}_v(\tilde y,\tilde t)}\text{var}(X_w) \over \left(\sum_{\tilde y\in \bold{Y}}\sum_{\tilde t\in \bold{T}}\mathcal{P}_{yt}(\tilde y, \tilde t)|\mathcal{N}_v|\right)^2}\\
&={1\over |\mathcal{N}_v|^2}{\sum_{\tilde y \in \bold{Y}}\sum_{\tilde t \in \bold{T}}\sum_{w\in \mathcal{N}_v(\tilde y,\tilde t)}\text{var}(X_w)}
\end{align}

If we assume $\text{var}(X_w)=\Sigma_{XX}^{(k)}(\tilde y)$ for $\forall w \in \mathcal{N}_v(\tilde y, \tilde t)$,

\begin{align}
\hat{\text{var}}(M_v^{(k+1)})&={1\over |\mathcal{N}_v|^2}{\sum_{\tilde y \in \bold{Y}}\sum_{\tilde t \in \bold{T}}\sum_{w\in \mathcal{N}_v(\tilde y,\tilde t)}\Sigma_{XX}^{(k)}(\tilde y)}\\
&={1\over |\mathcal{N}_v|^2}{\sum_{\tilde y \in \bold{Y}}\sum_{\tilde t \in \bold{T}} |\mathcal{N}_v(\tilde y,\tilde t)|\Sigma_{XX}^{(k)}(\tilde y)}\\
&\overset{(b)}={1\over |\mathcal{N}_v|^2}{\sum_{\tilde y \in \bold{Y}}\sum_{\tilde t \in \bold{T}} |\mathcal{N}_v|\mathcal{P}_{yt}(\tilde y,\tilde t)\Sigma_{XX}^{(k)}(\tilde y)}\\
&={1\over |\mathcal{N}_v|}{\sum_{\tilde y \in \bold{Y}}\sum_{\tilde t \in \bold{T}} \mathcal{P}_{yt}(\tilde y,\tilde t)\Sigma_{XX}^{(k)}(\tilde y)}
\end{align}










\subsection{Explanation of \PMP}
\label{apdx:PMP}
\subsubsection{1st moment of aggregated message obtained by \PMP layer.}
We define the 1st moment of \PMP with the identical steps of the 1st moment of averaging message passing, as in Appendix \ref{apdx:firstmm}.

\subsubsection{Proof of Theorem \ref{thm:pmp}}

From now on, we will denote $y$ and $t$ as the label and time belongs to target node, $v$, if there are no other specifications.

Suppose that the 1st moment of the representations from the previous layer is invariant. In other words, $\mu_{X}^{(k)}(y,t)=\mu_{X}^{(k)}(y,t_{max}),\ \forall t\in\bold{T}$. 

Formally, when defined as $\mathcal{N}^{\text{single}}_v$=$\{u\in \mathcal{N}_v \big| u \text{ has time in } \bold{T}^{\text{single}}_v \}$, and $\mathcal{N}^{\text{double}}_v$=$\{u\in \mathcal{N}_v\big|u \text{ has time in } \bold{T}^{\text{double}}_v\}$, the message passing mechanism of \PMP can be expressed as:

\begin{align}
	M_{ v}^{pmp(k+1)} &= {\sum_{\tilde{y}\in \bold{Y}}\sum_{\tilde{t}\in\bold{T}_{t}^{\text{single}}}\sum_{w\in\mathcal{N}_{v}\left(\tilde{y}, \tilde{t}\right)}2X_w+\sum_{\tilde{y}\in \bold{Y}}\sum_{\tilde{t}\in\bold{T}_{t}^{\text{double}}}\sum_{w\in\mathcal{N}_{v}\left(\tilde{y}, \tilde{t}\right)}X_w \over \sum_{\tilde{y}\in \bold{Y}}\sum_{\tilde{t}\in\bold{T}_{t}^{\text{single}}}2|\mathcal{N}_{v}\left(\tilde{y}, \tilde{t}\right)|+\sum_{\tilde{y}\in \bold{Y}}\sum_{\tilde{t}\in\bold{T}_{t}^{\text{double}}}|\mathcal{N}_{v}\left(\tilde{y}, \tilde{t}\right)|}
\end{align}

The representations from the previous layer are invariant, i.e., $\mathbb{E}_{X\sim {x_{yt}^{(k)}}}\left[X\right]=\mu_{X}^{(k)}(y)$. Here, $x_{yt}^{(k)}$ indicates the distribution of representation for the nodes in previous layer, whose label is $y$ and time is $t$. The first moment of the aggregated message is as follows. This first moment is calculated rigorously as shown in Appendix \ref{apdx:firstmm}.

\begin{align}
\hat{\mathbb E}\left[{M_{ i}^{pmp(k+1)}}\right] &= {\sum_{\tilde y\in \bold{Y}}\sum_{\tilde t\in\bold{T}_{t}^{\text{single}}}2\mathcal{P}_{y t}(\tilde y, \tilde t) \mu_{X}^{(k)}(\tilde y)+\sum_{\tilde y\in \bold{Y}}\sum_{\tilde t\in\bold{T}_{t}^{\text{double}}}\mathcal{P}_{y t}(\tilde y, \tilde t) \mu_{X}^{(k)}(\tilde y)\over \sum_{\tilde y\in \bold{Y}}\sum_{\tilde t\in\bold{T}_{t}^{\text{single}}}2\mathcal{P}_{y t}(\tilde y, \tilde t)+\sum_{\tilde y\in \bold{Y}}\sum_{\tilde t\in\bold{T}_{t}^{\text{double}}}\mathcal{P}_{y t}(\tilde y, \tilde t)}\\&={\sum_{\tilde y\in \bold{Y}}\left(\sum_{\tilde t\in\bold{T}_{t}^{\text{single}}}2\mathcal{P}_{y t}(\tilde y, \tilde t)+\sum_{\tilde t\in\bold{T}_{t}^{\text{double}}}\mathcal{P}_{y t}(\tilde y,\tilde t)\right)\mu_{X}^{(k)}(\tilde y)\over\sum_{\tilde y\in \bold{Y}}\left(\sum_{\tilde t\in\bold{T}_{t}^{\text{single}}}2\mathcal{P}_{y t}(\tilde y, \tilde t)+\sum_{\tilde t\in\bold{T}_{t}^{\text{double}}}\mathcal{P}_{y t}(\tilde y, \tilde t)\right)}
\end{align}

By assumption 3,

\begin{align}
&\sum_{\tilde t\in\bold{T}_{t}^{\text{single}}}2\mathcal{P}_{y t}(\tilde y, \tilde t)+\sum_{\tilde t\in\bold{T}_{t}^{\text{double}}}\mathcal{P}_{y t}(\tilde y, \tilde t) \\&=f(y, t )\left(\sum_{\tilde t\in\bold{T}_{t}^{\text{single}}}2g(y, \tilde y, |\tilde t - t|)+\sum_{\tilde t\in\bold{T}_{t}^{\text{double}}}g(y, \tilde y, |t - t|)\right)\\&=f(y, t )\left(2g(y, \tilde y, 0)+2\sum_{\tau>|t_{max}-t |}g(y, \tilde y,\tau)+\sum_{0<\tau\le|t_{max}-t|}g(y, \tilde y, \tau)\right) \\&= 2f(y, t )\sum_{\tau\ge 0}g(y, \tilde y, \tau)
\end{align}


Substituting this into the previous expression yields,

\begin{align}
\hat{\mathbb E}\left[{M_{ i}^{pmp(k+1)}}\right]={\sum_{\tilde y\in \bold{Y}}\sum_{\tau\ge 0}g(y, \tilde y, \tau)\mu_{X}^{(k)}(\tilde y)\over\sum_{y\in \bold{Y}}\sum_{\tau\ge 0}g(y, \tilde y, \tau)}
\end{align}

Since there is no $t$ term in this expression, the mean of this aggregated message is invariant with respect to the target node's time.

\begin{algorithm}
    \caption{\name Persistent Message Passing as neighbor aggregation}
    \label{alg:agg}
        \SetKwInOut{Input}{Input}\SetKwInOut{Output}{Output}
        \Input{~Undirected graph $\mathcal{G}(\bold{V},\bold{E})$; input features $X_v, \forall v\in \bold{V}$; number of layers $K$; node time function $time:\bold{V}\rightarrow \mathrm{R}$; maximum time value $t_{\max}$; minimum time value $t_{\min}$; aggregate functions $\textsc{agg}$; combine functions $\textsc{combine}$; multisets of neighborhood $\mathcal{N}_v, \forall v \in \bold{V}$}
        \BlankLine
        \Output{~Final embeddings $\bold{z}_v, \forall v \in \bold{V}$}
        \BlankLine
        \BlankLine
        $\mb{h}^0_v \leftarrow \mb{X}_v, \forall v \in \bold{V}$\;
        \For{$k=0...K-1$}{
            \For{$v \in \bold{V}$}{
                $\mathcal{N'}(v) \leftarrow \mathcal{N}(v)$\;
                \uIf{$|time(u)-time(v)|>\min(t_{\max}-time(v),time(v)-t_{\min})$}{
                    $\mathcal{N'}(v).\text{insert}(u)$ \;
                }
                $\mb{M}^{(k+1)}_{v} \leftarrow \textsc{agg}(\{\mb{h}_u^{(k)}, \forall u \in \mathcal{N'}(v)\})$\;
                $\mb{X}^{(k+1)}_{v} \leftarrow \textsc{combine}(\{\mb{X}^{(k)}_v, \mb{M}^{(k+1)}_{v}\})$\;
            }
        }
        $\mb{z}_v\leftarrow \mb{X}^{K}_v, \forall v \in \bold{V}$ \;
    \end{algorithm}

\begin{algorithm}
    \caption{\name Persistent Message Passing as graph reconstruction}
    \label{alg:recon}
        \SetKwInOut{Input}{Input}\SetKwInOut{Output}{Output}
        \Input{~Undirected graph $\mathcal{G}(\bold{V},\bold{E})$; adjacency matrix $A^{\mathcal{G}} \in \mathbb{R}^{N\times N}$; node time function $time:\bold{V}\rightarrow \mathbb{R}$; maximum time value $t_{\max}$; minimum time value $t_{\min}$}
        \Output{~New directed graph  $\mathcal{G'}(\bold{V},\bold{E'})$; new adjacency matrix $A^{\mathcal{G'}}$}
        \BlankLine
        $A^{\mathcal{G'}}\leftarrow A^{\mathcal{G}}$ \;
        \For{$(u,v) \in \bold{V}^2$}{
            \uIf{$|time(u)-time(v)|>\min(t_{\max}-time(v),time(v)-t_{\min})$}{
                $A^{\mathcal{G'}}_{uv} \leftarrow  2A^{\mathcal{G'}}_{uv}$ \;
            }
        }
    \end{algorithm}


\subsubsection{2nd moment of aggregated message obtained by PMP layer}

We define the 2nd moment of PMP incorporating the steps of the 2nd moment of averaging message passing as in Appendix \ref{apdx:firstmm}, and define an additional step as: \\

(c) Consider $|\mathcal{N}_v|$  as a value only dependent to $y$ and $t$, namely $|\mathcal{N}_{yt}|$. \\

Background of step (c) is that in practice, $|\mathcal{N}_v|$ can vary for each node, but they will follow a distribution determined by the node's label $y$ and time $t$. For simplicity in our discussion, we will use the expectation of these values within each community as in step (c).

This 2nd moment is calculated rigorously as shown in Appendix \ref{apdx:firstmm}.

\vspace{-15pt}

\begin{align}
\hat{\text{var}}(M_{v}^{pmp(k+1)}) = {\sum_{\tilde{y}\in \bold{Y}}\left(\sum_{\tilde{t}\in\bold{T}_{t}^{\text{single}}}4\mathcal{P}_{yt}\left(\tilde{y}, \tilde{t}\right)+\sum_{\tilde{t}\in\bold{T}_{t}^{\text{double}}}\mathcal{P}_{yt}\left(\tilde{y}, \tilde{t}\right)\right)\Sigma_{XX}^{pmp(k)}(\tilde{y})
\over
\left(\sum_{\tilde{y}\in \bold{Y}}\sum_{\tilde{t}\in\bold{T}_{t}^{\text{single}}}2\mathcal{P}_{yt}\left(\tilde{y}, \tilde{t}\right)+\sum_{\tilde{y}\in \bold{Y}}\sum_{\tilde{t}\in\bold{T}_{t}^{\text{double}}}\mathcal{P}_{yt}\left(\tilde{y}, \tilde{t}\right)\right)^2|\mathcal{N}_{yt}|}
\end{align}


\vspace{-5pt}

Therefore, we can write $\hat{\text{var}}(M_{v}^{pmp(k+1)})$=$\Sigma^{pmp(k+1)}_{MM}(y,t)$.



\subsection{Explanation of \MMP}
\label{apdx:MMP}
\subsubsection{1st moment of aggregated message obtained by \MMP layer.}
We define the 1st moment of \MMP with the identical steps of the 1st moment of averaging message passing, as in Appendix \ref{apdx:firstmm}.


\subsubsection{Proof of Theorem \ref{thm:mmp}}
Suppose that the 1st moment of the representations from the previous layer is invariant. In other words, $\mu_{X}^{(k)}(y,t)=\mu_{X}^{(k)}(y,t_{max}),\ \forall t\in\bold{T}$. The message passing mechanism of \PMP can be expressed as follows:

\begin{figure}[hbt!]
	\vspace{0.15in}
	\centering
	\includegraphics[width=0.20\textwidth]{figs/MMP.png}
	\vspace{-0.1in}
	\caption{Graphical explanation of Mono-directional Message Passing(\MMP).}
	 \label{fig:MMP}
	 \vspace{0.5in}
\end{figure}

\begin{align}
    M_{v}^{mmp(k+1)} = {\sum_{\tilde{y}\in \bold{Y}}\sum_{\tilde{t}\le t}\sum_{v\in\mathcal{N}_{v}(\tilde y,\tilde t)}X_w \over \sum_{\tilde{y}\in \bold{Y}}\sum_{\tilde{t}\le t}|\mathcal{N}_{v}(\tilde y, \tilde t)|}
\end{align}


Applying assumption 3 as in \PMP, the expectation is as follows. This expectation is calculated rigorously as shown in Appendix \ref{apdx:firstmm}.

\begin{align}
\hat{\mathbb E}\left[{M_{ i}^{mmp(k+1)}}\right] = {\sum_{\tilde y\in \bold{Y}}\sum_{\tilde t\le t }\mathcal{P}_{y t}(\tilde y, \tilde t) \mu_{X}^{(k)}(\tilde y)\over \sum_{\tilde y\in \bold{Y}}\sum_{\tilde t\le t }\mathcal{P}_{y t}(\tilde y, \tilde t)}={\sum_{\tilde y\in \bold{Y}}\sum_{\tau\ge 0}g(y, \tilde y,\tau) \mu_{X}^{(k)}(\tilde y)\over \sum_{\tilde y\in \bold{Y}}\sum_{\tau\ge 0}g(y, \tilde y,\tau)}
\end{align}

This also lacks the $t$ term, thus it is invariant.






\subsection{Mathematical modeling of \PMP.}\label{apdx:pmp_modeling}
Define probability measure space $(\mathcal{M}^{(k)},\sum_{\mathcal{M}^{(k)}},m_{yt}^{(k)})$, $(\mathcal{X}^{(k)},\sum_{\mathcal{X}^{(k)}},x_{yt}^{(k)})$ where $\sum_{\mathcal{M}^{(k)}}$, and $\sum_{\mathcal{X}^{(k)}}$are $\sigma$-algebras with probability measures $m_{yt}^{(k)}$ and $x_{yt}^{(k)}$respectively.

That is, $m_{yt}^{(k)}$ is the probability measure of the message of node with label $y$ and time $t$, and $x_{yt}^{(k)}$ is the probability measure of the representation of node with label $y$ and time $t$, as defined previously.

\subsubsection{$m_{yt}^{(k)}$ to $x_{yt}^{(k)}$}

$f^{(k)}$ is the function which transfers the message $M_v^{(k)}\in \mathcal{M}^{(k)}$ to the $k$-th layer representation $X_v^{(k)}\in \mathcal{X}^{(k)}$.

Hence, $f^{(k)}:\mathcal{M}^{(k)}\rightarrow \mathcal{X}^{(k)}$ gives a pushforward of measure as $x_{yt}^{(k)}=(f_*^{(k)})(m_{yt}^{(k)}):\sum_{\mathcal{X}^{(k)}}\rightarrow[0, 1]$, given by $\left((f_*^{(k)})(m_{yt}^{(k)})\right)(B)=m_{yt}^{(k)}\left((f^{(k)})^{-1}(B)\right),\text{ for }\forall B\in \sum_{\mathcal{X}^{(k)}}$

Here, we assume $f^{(k)}$ is G-Lipschitz for $\forall k\in\{1,2,\dots,K\}$.

\subsubsection{$x_{yt}^{(k)}$ to $m_{yt}^{(k+1)}$} 
This is given as the message passing function of \PMP. That is,

\begin{align}
m_{yt}^{(k+1)}={\sum_{\tilde y\in \bold{Y}}\sum_{\tilde t\in \bold{T}_t^{single}}2\mathcal{P}_{yt}(\tilde y, \tilde t)x_{\tilde y \tilde t}^{(k)}+ \sum_{\tilde y\in \bold{Y}}\sum_{\tilde t\in \bold{T}_t^{double}}\mathcal{P}_{yt}(\tilde y, \tilde t)x_{\tilde y \tilde t}^{(k)} \over \sum_{\tilde y\in \bold{Y}}\sum_{\tilde t\in \bold{T}_t^{single}}2\mathcal{P}_{yt}(\tilde y, \tilde t)+ \sum_{\tilde y\in \bold{Y}}\sum_{\tilde t\in \bold{T}_t^{double}}\mathcal{P}_{yt}(\tilde y, \tilde t)}
\end{align}






\subsection{Theoretical analysis of \PMP when applied in multi-layer GNNs.}\label{apdx:pmp_theory}
\subsubsection{Lemmas}
\begin{lemma}\label{lem:lem1}
\begin{align}
    \forall \epsilon >0, P(|M_v^{(k)}-M_{v'}^{(k)}|>\epsilon)\le {8V \over \epsilon^2}\text{ for }M_v^{(k)}\sim m_{yt}^{(k)},M_{v'}^{(k)}\sim m_{yt'}^{(k)}
\end{align}
\end{lemma}
\begin{proof}
By chebyshev��s inequality, 

$P(|M_v^{(k)}-\mu_M^{(k)}(y)|>{\epsilon\over 2})\le {4V \over \epsilon^2}$, $P(|M_{v'}^{(k)}-\mu_M^{(k)}(y)|>{\epsilon\over 2})\le {4V \over \epsilon^2}$. 

Therefore,
\begin{align}
P(|M_v^{(k)}-M_{v'}^{(k)}|&>{\epsilon})\\
&\le P(|M_v^{(k)}-\mu_{M}{(y)}|+|M_{v'}^{(k)}-\mu_{M}{(y)}|>\epsilon) \ \because\text{Triangle inequality}\\
&\le P(|M_v^{(k)}-\mu_{M}{(y)}|>{\epsilon\over 2}\text{ or }|M_{v'}^{(k)}-\mu_{M}{(y)}|>{\epsilon\over 2})\\
&\le P(|M_v^{(k)}-\mu_{M}{(y)}|>{\epsilon\over 2})+P(|M_{v'}^{(k)}-\mu_{M}{(y)}|>{\epsilon\over 2})\le {8V\over \epsilon^2}
\end{align}
\end{proof}

\begin{lemma}\label{lem:lem2}
\begin{align}
W_1 (x_{yt}^{(k)}, x_{yt'}^{(k)})\le G\ W_1(m_{yt}^{(k)},m_{yt'}^{(k)})
\end{align}
\end{lemma}
\begin{proof}
    Follows directly from G-Lipshitz property of $f^{(k)}$ and definition of pushforward measures.
\end{proof}


\begin{lemma}\label{lem:lem3}
$\mu_1,\dots,\mu_n$ are distributions with cumulative distribution functions $F_1, \dots, F_n$. If $W_1(\mu_i, \mu_j)\le D,\ \forall i,j$,  

For arbitrary real numbers satisfying $0<\eta_i, \nu_i<S,\ s.t. \ \eta_1 + \dots+\eta_n=\nu_1+\dots+\nu_n=S$, 

\begin{align}
W_1 (\eta_1\mu_1+\dots +\eta_n \mu_n,\nu_1 \mu_1+\dots+\nu_n\mu_n)<(S-\delta)D
\end{align}

for some positive real number $\delta$.
\end{lemma}
\begin{proof}
\begin{align}
&\int_{\mathbb{R}}\big|\sum_{i=1}^{n}(\eta_i-\nu_i)F_i(x) \big|dx\\
&=\int_{\mathbb{R}}\big|\sum_{i=1}^{n}\delta_iF_i(x)\big|dx\text{, where }\delta_i=\eta_i-\nu_i\\
&=\int_{\mathbb{R}}\big|\sum_{\{i|\delta_i\ge 0\}}\delta_iF_i(x)+\sum_{\{j|\delta_j< 0\}}\delta_jF_j(x)\big|dx\\
&=\int_{\mathbb{R}}\big|\sum_{\{i|\delta_i\ge 0\}}\delta_i \big(\delta_{i,1}( F_i(x)-F_{i,1}(x))+\cdots+\delta_{i,n(i)}( F_i(x)-F_{i,n(i)}(x))\big) \big|dx
\end{align}
for some $\delta_{i,1},\dots,\delta_{i,n(i)}>0,\ s.t.\ \delta_{i,1}+\cdots+\delta_{i,n(i)}=1$.

\begin{align}
&\int_{\mathbb{R}}\big|\sum_{\{i|\delta_i\ge 0\}}\delta_i \big(\delta_{i,1}( F_i(x)-F_{i,1}(x))+\cdots+\delta_{i,n(i)}( F_i(x)-F_{i,n(i)}(x))\big) \big|dx\\
&\le \int_{\mathbb{R}}\sum_{\{i|\delta_i\ge 0\}}\delta_i \big(\delta_{i,1}|F_i(x)-F_{i,1}(x)|+\cdots+\delta_{i,n(i)}| F_i(x)-F_{i,n(i)}(x)|\big) dx\\
&\le \sum_{\{i|\delta_i\ge 0\}}\delta_i \big(\delta_{i,1}+\cdots+\delta_{i,n(i)}\big)D\\
&=\sum_{\{i|\delta_i\ge 0\}}\delta_i D\\
&=\sum_{\{i|\eta_i-\nu_i\ge 0\}}(\eta_i-\nu_i)D\\
&<\sum_{\{i|\eta_i-\nu_i\ge 0\}}(\eta_i)D\\
&<SD
\end{align}
\end{proof}



\subsubsection{Proof of Theorem \ref{thm:thm1}.}
{\scriptsize
\begin{align}
\mathbb{E}[|M_v^{(k)}-M_{v'}^{(k)}|]=\mathbb{E}\left[|M_v^{(k)}-M_{v'}^{(k)}|\mathds{1}_{\{|M_v^{(k)}-M_{v'}^{(k)}|\le \epsilon\}}\right]+\mathbb{E}\left[|M_v^{(k)}-M_{v'}^{(k)}|\mathds{1}_{\{|M_v^{(k)}-M_{v'}^{(k)}|> \epsilon\}}\right]\le\epsilon+{16CV\over \epsilon^2}
\end{align}
}%

since $\mathbb{E}\left[|M_v^{(k)}-M_{v'}^{(k)}|\mathds{1}{\{|M_v^{(k)}-M_{v'}^{(k)}|\le \epsilon\}}\right]\le \epsilon$, and \\
$\mathbb{E}\left[|M_v^{(k)}-M_{v'}^{(k)}|\mathds{1}{\{|M_v^{(k)}-M_{v'}^{(k)}|> \epsilon\}}\right]\le 2 C\ P(|M_v^{(k)}-M_{v'}^{(k)}|>\epsilon)\le {16CV\over \epsilon^2}$ by Lemma \ref{lem:lem1}.

Plugging in $2(4CV)^{1/3}$ to $\epsilon$ gives us, $\mathbb{E}[|M_v^{(k)}-M_{v'}^{(k)}|]\le 3(4CV)^{1/3}$.

\begin{align}
\therefore W_1 (m_{yt}^{(k)},m_{yt'}^{(k)})\le\mathbb{E}[|M_v^{(k)}-M_{v'}^{(k)}|]\le \mathcal{O}(C^{1/3}V^{1/3})
\end{align}


\subsubsection{Proof of Theorem \ref{thm:thm2}}
By Hoeffding��s inequality, $P(|M_v^{(k)}-M_{v'}^{(k)}|>{\epsilon\over 2})\le 2 \exp(-{\epsilon^2\over{8\tau^2}})$.

So with the same steps of Theorem \ref{thm:thm1}, $\mathbb{E}[|M_v^{(k)}-M_{v'}^{(k)}|]\le \epsilon + 4C\exp(-{\epsilon^2\over{8\tau^2}})$.

Plug in $(8\tau^2 \log C)^{1/2}$ to $\epsilon$. Then $\mathbb{E}[|M_v^{(k)}-M_{v'}^{(k)}|]\le (8\tau^2 \log C)^{1/2} + 4$.

\begin{align}
\therefore W_1 (m_{yt}^{(k)},m_{yt'}^{(k)})\le \mathcal O(\tau\sqrt{\log C})
\end{align}

\subsubsection{Proof of Theorem \ref{thm:thm3}}
{\scriptsize
\begin{align}
m_{yt}^{(k+1)}&={\sum_{\tilde y\in \bold{Y}}\sum_{\tilde t\in \bold{T}_t^{single}}2\mathcal{P}_{yt}(\tilde y, \tilde t)x_{\tilde y \tilde t}^{(k)}+ \sum_{\tilde y\in \bold{Y}}\sum_{\tilde t\in \bold{T}_t^{double}}\mathcal{P}_{yt}(\tilde y, \tilde t)x_{\tilde y \tilde t}^{(k)} \over \sum_{\tilde y\in \bold{Y}}\sum_{\tilde t\in \bold{T}_t^{single}}2\mathcal{P}_{yt}(\tilde y, \tilde t)+ \sum_{\tilde y\in \bold{Y}}\sum_{\tilde t\in \bold{T}_t^{double}}\mathcal{P}_{yt}(\tilde y, \tilde t)}\\
&={\sum_{\tilde y\in \bold{Y}}\sum_{\tilde t\in \bold{T}_t^{single}}2f(y,t)g(y,\tilde y,|\tilde t-t|)x_{\tilde y \tilde t}^{(k)}+ \sum_{\tilde y\in \bold{Y}}\sum_{\tilde t\in \bold{T}_t^{double}}f(y,t)g(y,\tilde y,|\tilde t-t|)x_{\tilde y \tilde t}^{(k)} \over \sum_{\tilde y\in \bold{Y}}\sum_{\tilde t\in \bold{T}_t^{single}}2f(y,t)g(y,\tilde y,|\tilde t-t|)+ \sum_{\tilde y\in \bold{Y}}\sum_{\tilde t\in \bold{T}_t^{double}}f(y,t)g(y,\tilde y,|\tilde t-t|)}\\
&={\sum_{\tilde y\in \bold{Y}}\sum_{\tilde t\in \bold{T}_t^{single}}2g(y,\tilde y,|\tilde t-t|)x_{\tilde y \tilde t}^{(k)}+ \sum_{\tilde y\in \bold{Y}}\sum_{\tilde t\in \bold{T}_t^{double}}g(y,\tilde y,|\tilde t-t|)x_{\tilde y \tilde t}^{(k)} \over \sum_{\tilde y\in \bold{Y}}\sum_{\tilde t\in \bold{T}_t^{single}}2g(y,\tilde y,|\tilde t-t|)+ \sum_{\tilde y\in \bold{Y}}\sum_{\tilde t\in \bold{T}_t^{double}}g(y,\tilde y,|\tilde t-t|)}\\
&\overset{let}= \sum_{\tilde y\in\bold{Y}}\sum_{\tilde t\in \bold{T}}\lambda_{yt\tilde y \tilde t}x_{\tilde y \tilde t}^{(k)}
\end{align}
}%

Where $0<\lambda_{yt\tilde y \tilde t}<1$ is effective message passing weight in \PMP, hence satisfying $\sum_{\tilde y\in \bold{Y}}\sum_{\tilde t\in \bold{T}}\lambda_{yt\tilde y \tilde t}=1$.

Furthermore, since $\sum_{\tilde t\in \bold{T}_t^{single}}2g(y,\tilde y,|\tilde t-t|)+\sum_{\tilde t\in \bold{T}_t^{double}}g(y,\tilde y,|\tilde t-t|)=2\sum_{\tau\le 0}g(y,\tilde y,\tau)$, the following relation holds:

\begin{align}
\sum_{\tilde t\in \bold{T}}\lambda_{yt\tilde y \tilde t}={\sum_{\tau\ge 0}g(y,\tilde y,\tau)\over\sum_{y'\in \bold{Y}}\sum_{\tau\ge 0}g(y,y',\tau)}
\end{align}

Thus, $\sum_{\tilde t\in \bold{T}}\lambda_{yt\tilde y \tilde t}=\sum_{\tilde t\in \bold{T}}\lambda_{yt'\tilde y \tilde t}, \ \forall t,t'\in \bold{T}$. We can let $\sum_{\tilde t\in \bold{T}}\lambda_{yt\tilde y \tilde t}=\rho_{y\tilde y}$.

\begin{align}
W_1(m_{yt}^{(k)},m_{yt_{max}}^{(k)})&=W_1 \left( \sum_{\tilde y\in \bold{Y}}\sum_{\tilde t\in \bold{T}}\lambda_{yt\tilde y\tilde t}x_{\tilde y\tilde t}^{(k)},\sum_{\tilde y\in \bold{Y}}\sum_{\tilde t\in \bold{T}}\lambda_{yt'\tilde y\tilde t}x_{\tilde y\tilde t}^{(k)}\right)\\
&=\int_{\mathbb{R}}\big| \sum_{\tilde y\in \bold{Y}}\sum_{\tilde t\in \bold{T}}\lambda_{yt\tilde y\tilde t}F_{\tilde y\tilde t}^{(k)}(x)-\sum_{\tilde y\in \bold{Y}}\sum_{\tilde t\in \bold{T}}\lambda_{yt'\tilde y\tilde t}F_{\tilde y\tilde t}^{(k)}(x)\big|dx\\
&=\int_{\mathbb{R}}\big| \sum_{\tilde y\in \bold{Y}}\sum_{\tilde t\in \bold{T}}(\lambda_{yt\tilde y\tilde t}-\lambda_{yt'\tilde y\tilde t})F_{\tilde y\tilde t}^{(k)}(x)\big|dx
\end{align}

Where $F_{\tilde y \tilde t}^{(k)}$ is the cumulative distribution function of $x_{\tilde y \tilde t}^{(k)}$.

By Lemma \ref{lem:lem2} and Lemma \ref{lem:lem3}, 

\begin{align}
\int_{\mathbb{R}}\big| \sum_{\tilde t\in \bold{T}}(\lambda_{yt\tilde y\tilde t}-\lambda_{yt'\tilde y\tilde t})F_{\tilde y\tilde t}^{(k)}(x)\big|dx\le (\rho_{y\tilde y}-\epsilon_{y\tilde y t t'})GW
\end{align}

For some $0<\epsilon_{y\tilde y t t'}<\rho_{y\tilde y}$.

\begin{align}
\therefore W_1(m_{yt}^{(k)},m_{yt_{max}}^{(k)})&\le \int_{\mathbb{R}}\sum_{\tilde y\in \bold{Y}}\sum_{\tilde t\in \bold{T}}\big| (\lambda_{yt\tilde y\tilde t}-\lambda_{yt'\tilde y\tilde t})F_{\tilde y\tilde t}^{(k)}(x)\big|dx\\
&\le \sum_{\tilde y\in \bold{Y}}(\rho_{y\tilde y}-\epsilon_{y\tilde y t t'})GW\\
&=G(1-\sum_{\tilde y\in \bold{Y}}\epsilon_{y\tilde y t t'})W
\end{align}

Let $\epsilon_{ytt'}=\sum_{\tilde y\in \bold{Y}}\epsilon_{y\tilde y t t'}$ and $\min_{y\in \bold{Y}, t,t'\in \bold{T}}\epsilon_{ytt'}=\epsilon$. 

Then, $W_1(m_{yt}^{(k)},m_{yt'}^{(k)})\le G(1-\epsilon)W$.

Let $G^{(k)}={1\over {1-\epsilon}}>1$.

Then, $\forall y,t,t', \ W_1 (m_{yt}^{(k+1)},m_{yt_{max}}^{(k+1)})\le {G\over G^{(k)}}W$















\subsection{Estimation of relative connectivity}
\label{apdx:rel_con}
When $t \neq t_{max}$ and $\tilde t \neq t_{max}$, $\mathcal{P}_{y t} (\tilde y ,\tilde t)$ has the following best unbiased estimator:
\begin{align}
\hat{\mathcal{P}}_{y t} (\tilde y ,\tilde t)={\sum_{u\in \{u'\in \bold{V} | u'\text{ has label }y, u'\text{ has time }t\}}| \mathcal{N}_u(\tilde y, \tilde t) |\over \sum _{u\in \{u'\in \bold{V} | u'\text{ has label }y, u'\text{ has time } t\}}|\mathcal{N}_u|} , \ \forall t, \tilde t \neq t_{max}
\end{align}

We can regard this problem as a nonlinear overdetermined system $\hat{\mathcal{P}}_{y t} \left(\tilde{y}, \tilde{t}\right) = f(y, t) g\left(y, \tilde{y}, | \tilde{t}-t|\right), \ \forall y, \tilde{y} \in \bold{Y}, \forall t, \tilde{t} \in \bold{T}$, with the constraint of $\sum_{\tilde{y} \in \bold{Y}}\sum_{\tilde{t} \in \bold{T}} \hat{\mathcal{P}}_{y t} \left(\tilde{y}, \tilde{t}\right)=1$.\\




When $t=t_{max}$ or $\tilde t=t_{max}$ is not feasible due to the unavailability of labels in the test set, we utilize assumption 3 to compute $\hat{\mathcal{P}}_{y t} (\tilde y ,\tilde t)$ for this cases. Let's first consider the following equation:

\begin{align}
\sum_{\tilde y\in\bold Y}\mathcal{P}_{yt}(\tilde y, t) = \sum_{\tilde y \in \bold{Y}} f(y, t)g(y, \tilde y, 0) =f(y, t)\sum_{\tilde y \in \bold{Y}}g(y, \tilde y, 0)
\end{align}

Earlier, when introducing assumption 3, we defined $\sum_{\tilde y \in \bold{Y}}g(y, \tilde y, 0)=1$. Therefore, when $t<t_{max}$, we can express $f(y, t)$ as follows:

\begin{align}
f(y, t)=\sum_{\tilde y\in\bold Y}\mathcal{P}_{yt}(\tilde y, t)
\end{align}

For any $\Delta \in \{|\tilde t -t | \mid t, \tilde t\in \bold{T}\}$, we have:

\begin{align}
\sum_{t< t_{max}-\Delta}\mathcal{P}_{yt}(\tilde y, t+\Delta) =\sum_{t< t_{max}-\Delta}f(y, t)g(y, \tilde y, \Delta)
\end{align}

\begin{align}
\sum_{t<t_{max}}\mathcal{P}_{yt}(\tilde y, t-\Delta) =\sum_{t<t_{max}}f(y, t)g(y, \tilde y, \Delta)
\end{align}

The reason we consider up to $t= {t_{max}-1-\Delta}$ in the first equation and up to $t = t_{max}-1$ in the second equation is because we assume situations where ${\mathcal{P}}_{y t} (\tilde y ,\tilde t)$ cannot be estimated when $t=t_{max}$ or $\tilde t=t_{max}$. Utilizing both equations aims to construct an estimator using as many measured values as possible when $t\neq t_{max}$.

Thus,

\begin{align}
g(y, \tilde y, \Delta)= {\sum_{t< t_{max}-\Delta}\mathcal{P}_{yt}(\tilde y, t+\Delta)+\sum_{t<t_{max}} \mathcal{P}_{yt}(\tilde y, t-\Delta)\over \sum_{t< t_{max}-\Delta}f(y, t)+\sum_{t<t_{max}}f(y, t)}
\end{align}

Since $f(y, t)=\sum_{\tilde y\in\bold Y}\mathcal{P}_{yt}(\tilde y, t)$,

\begin{align}
g(y, \tilde y, \Delta)= {\sum_{t< t_{max}-\Delta}\mathcal{P}_{yt}(\tilde y, t+\Delta)+\sum_{t<t_{max}} \mathcal{P}_{yt}(\tilde y, t-\Delta)\over \sum_{t< t_{max}-\Delta}\sum_{y'\in\bold Y}\mathcal{P}_{yt}(y', t)+\sum_{t<t_{max}}\sum_{y'\in\bold Y}\mathcal{P}_{yt}(y', t)}
\end{align}

For any $y, \tilde y \in \bold{Y}$ and $\Delta \in \{|\tilde t -t | \mid t, \tilde t\in \bold{T}\}$, we can construct an estimator $\hat{g}(y, \tilde y, \Delta)$ for $g(y, \tilde y, \Delta)$ as follows:

\begin{align}
\hat{g}(y, \tilde y, \Delta)= {\sum_{t< t_{max}-\Delta}\hat{\mathcal{P}}_{yt}(\tilde y, t+\Delta)+\sum_{t<t_{max}} \hat{\mathcal{P}}_{yt}(\tilde y, t-\Delta)\over \sum_{t< t_{max}-\Delta}\sum_{y'\in\bold Y}\hat{\mathcal{P}}_{yt}(y', t)+\sum_{t<t_{max}}\sum_{y'\in\bold Y}\hat{\mathcal{P}}_{yt}(y', t)}
\end{align}

This estimator is designed to utilize as many measured values $\hat{\mathcal{P}}_{y t} (\tilde y ,\tilde t)$ as possible, excluding cases where $t=t_{max}$ or $\tilde t=t_{max}$.

\begin{align}
\mathcal P_{y t}(\tilde y, \tilde t)= {\mathcal P_{y t}(\tilde y, \tilde t)\over \sum_{y'\in \bold{Y}}\sum_{t'\in\bold{T}}\mathcal{P}_{y t}(y', t')}={g(y, \tilde y, |\tilde t-t|)\over \sum_{y'\in \bold{Y}}\sum_{t'\in\bold{T}}g(y, y', |t'-t|)}
\end{align}

Therefore, for all $y, \tilde y \in \bold{Y}$ and $|\tilde t - t |\in\{|\tilde t -t | \mid t, \tilde t\in \bold{T}\}$, we can define the estimator $\hat{\mathcal P}_{y t}(\tilde y, \tilde t)$ of $\mathcal P_{y t}(\tilde y, \tilde t)$ as follows:

\begin{align}
\hat{\mathcal P}_{y t}(\tilde y, \tilde t)={\hat{g}(y, \tilde y, |\tilde t-t|)\over \sum_{y'\in \bold{Y}}\sum_{t'\in\bold{T}}\hat{g}(y, y', |t'-t|)}
\end{align}


\begin{algorithm}
    \caption{\name Estimation of relative connectivity.}
    \label{alg:rel_con}
            \SetKwInOut{Input}{Input}\SetKwInOut{Output}{Output}
        \Input{~ Neighboring node sets $\mathcal{N}_{u},\ \forall u \in \bold{V}$; node time function $time:V\rightarrow \bold{T}$; train, test split $V^{tr}=\{v\mid v\in V, time(v) < t_{\max}\}$ and $V^{te}=\{v\mid v\in \bold{V}, time(v) = t_{\max}\}$; node label function $label:\bold{V}^{tr} \rightarrow \bold{Y}$.}
        \BlankLine
        \Output{~Estimated relative connectivity $\hat{\mathcal{P}}_{y, t}(\tilde y ,\tilde t)$, $\forall y, \tilde y\in \bold{Y},\ t, \tilde t \in \bold{T}$.}
    
        \BlankLine
        \BlankLine
        \textbf{Estimate $\hat{\mathcal{P}}_{y, t} (\tilde y ,\tilde t)$ when $t \neq t_{\max}$ and $\tilde t \neq t_{\max}$.}\\
        \For{$t \in \bold{T}\setminus\{t_{\max}\}$}{
            \For{$\tilde t \in \bold{T}\setminus\{t_{\max}\}$}{
                $\hat{\mathcal{P}}_{y, t} (\tilde y ,\tilde t)\leftarrow{\sum_{u\in \{v\in \bold{V} | v\text{ has label }y, v\text{ has time }t\}}|\{v\in \mathcal{N}_u | v\text{ has label }\tilde y, v\text{ has time }\tilde t\}|\over \sum _{u\in \{v\in \bold{V} | v\text{ has label }y, v\text{ has time }t\}}|\mathcal{N}_u|}$\;
            }
        }
        
        \BlankLine
        \textbf{Estimate $g$ function.}\\
        \For{$y \in \bold{Y}$}{
            \For{$\tilde y \in \bold{Y}$}{
                \For{$\Delta \in \{|\tilde t -t | \mid t, \tilde t\in \bold{T}\}$}{
                    $\hat{g}(y, \tilde y, \Delta)\leftarrow {\sum_{t< t_{\max}-\Delta}\hat{\mathcal{P}}_{y,t}(\tilde y, t+\Delta)+\sum_{t<t_{\max}} \hat{\mathcal{P}}_{y,t}(\tilde y, t-\Delta)\over \sum_{t< t_{\max}-\Delta}\sum_{y'\in\bold Y}\hat{\mathcal{P}}_{y,t}(y', t)+\sum_{t<t_{\max}}\sum_{y'\in\bold Y}\hat{\mathcal{P}}_{y,t}(y', t)}$\;
                }
            }
        }
    
        \BlankLine
        \textbf{Estimate $\hat{\mathcal{P}}_{y, t} (\tilde y ,\tilde t)$ when $t = t_{\max}$ or $\tilde t = t_{\max}$.}\\
        \For{$y \in \bold{Y}$}{
            \For{$\tilde y \in \bold{Y}$}{
                \For{$t \in \bold{T}$}{
                    $\hat{\mathcal P}_{y, t}(\tilde y, t_{\max})\leftarrow{\hat{g}(y, \tilde y, |t_{\max}-t|)\over \sum_{y'\in \bold{Y}}\sum_{t'\in\bold{T}}\hat{g}(y, y', |t'-t|)}$\;
                }
            }
        }
        \For{$y \in \bold{Y}$}{
            \For{$\tilde y \in \bold{Y}$}{
                \For{$\tilde t \in \bold{T}$}{
                    $\hat{\mathcal P}_{y, t_{\max}}(\tilde y, \tilde t)\leftarrow{\hat{g}(y, \tilde y, |\tilde t-t_{\max}|)\over \sum_{y'\in \bold{Y}}\sum_{t'\in\bold{T}}\hat{g}(y, y', |t'-t_{\max}|)}$\;
                }
            }
        }
        
    \end{algorithm}












\subsection{Explanation of \PNY}
\label{apdx:PNY}

\subsubsection{1st and 2nd moment of aggregated message obtained through \PNY transform.}
We define the 1st and 2nd moment of \PNY with the identical steps of the 1st and 2nd moment of averaging message passing, as in Appendix \ref{apdx:secondmm}.

\subsubsection{Proof of Theorem \ref{thm:pny}}
% Suppose that the variance and expectation of the representation from the previous layer are invariant with respect to the target node's time $t$. If we can specify $\mathcal{P}_{y t}(\tilde y, \tilde t)$ for all cases, transformation of covariance matrix during the \PMP process could be calculated. \PNY numerically estimates the transformation of the covariance matrix during the \PMP process, and determines an affine transformation that can correct this variation. See \ref{apdx:rel_con} for detailed estimation algorithm to estimate all $\mathcal{P}_{y t}(\tilde y, \tilde t)$.
\begin{align}
\hat{\mathbb{E}} [M_v^{PNY(k+1)}]\overset{(a)}&\overset{(b)}=\mathbb{E}[A_t (M_v^{pmp(k+1)}-\mu_M^{pmp(k+1)})]+\mathbb{E}[M_v^{pmp(k+1)}]\\
&=A_t(\mathbb{E}[M_v^{pmp(k+1)}]-\mu_M^{pmp(k+1)})+\mu_M^{pmp(k+1)}\\
&\overset{(b)}=A_t(\mu_M^{pmp(k+1)}-\mu_M^{pmp(k+1)})+\mu_M^{pmp(k+1)}\\
&=\mu_M^{pmp(k+1)}
\end{align}

\begin{align}
\hat{\text{var}}[M_v^{PNY(k+1)}]&\overset{(a)}=\text{var}\left(A_t(M_v^{pmp(k+1)}-\mu_M^{pmp(k+1)}(y))+\mu_M^{pmp(k+1)}(y)\right)\\ &\overset{(b)}=\mathbb{E}[A_t(M_v^{pmp(k+1)}-\mu_M^{pmp(k+1)}(y))(M_v^{pmp(k+1)}-\mu_M^{pmp(k+1)}(y))^{\top}A_t^{\top}]\\
&=A_t\mathbb{E}[(M_v^{pmp(k+1)}-\mu_M^{pmp(k+1)}(y))(M_v^{pmp(k+1)}-\mu_M^{pmp(k+1)}(y))^{\top}]A_t^{\top}\\
&\overset{(b)}=A_t\hat{\text{var}}(M_v^{pmp(k+1)})A_t^{\top}\\
&=(U_{yt_{max}}\Lambda_{yt_{max}}^{1/2}\Lambda_{yt}^{-1/2}U_{yt}^{\top})\Sigma_{MM}^{pmp(k+1)}(U_{yt}\Lambda_{yt}^{-1/2}\Lambda_{yt_{max}}^{1/2}U_{yt_{max}}^{\top})\\
&=(U_{yt_{max}}\Lambda_{yt_{max}}^{1/2}\Lambda_{yt}^{-1/2}U_{yt}^{\top})(U_{yt}\Lambda_{yt}U_{yt}^{-1})(U_{yt}\Lambda_{yt}^{-1/2}\Lambda_{yt_{max}}^{1/2}U_{yt_{max}}^{\top})\\
&=(U_{yt_{max}}\Lambda_{yt_{max}}^{1/2}\Lambda_{yt}^{-1/2})\Lambda_{yt}(\Lambda_{yt}^{-1/2}\Lambda_{yt_{max}}^{1/2}U_{yt_{max}}^{\top})\\
&=(U_{yt_{max}}\Lambda_{yt_{max}}^{1/2})(\Lambda_{yt_{max}}^{1/2}U_{yt_{max}}^{\top})\\
&=U_{yt_{max}}\Lambda_{yt_{max}}U_{yt_{max}}^{\top}\\
&=\Sigma_{MM}^{pmp(k+1)}(y,t_{max})
\end{align}

% The calculation of the $k+1$-th aggregated message $M_{v}^{pmp(k+1)}$ for the node $v$ described earlier is as follows:

% \begin{align}
% M_{v}^{pmp(k+1)} = {{2\sum_{u\in \mathcal{N}^{\text{single}}_v} X_{u}^{(k)}+\sum_{u\in \mathcal{N}^{\text{double}}_v} X_{u}^{(k)}}\over{2\big|\mathcal{N}^{\text{single}}_v\big| + \big|\mathcal{N}^{\text{double}}_v\big|}}
% \end{align}

% Here, $\mathcal{N}^{\text{single}}_v$=$\{u\in \mathcal{N}_v \big| u \text{ has time in } \bold{T}^{\text{single}}_v \}$, and $\mathcal{N}^{\text{double}}_v$=$\{u\in \mathcal{N}_v\big|u \text{ has time in } \bold{T}^{\text{double}}_v\}$. We previously proved that the expectation of $M_{v}^{pmp(k+1)}$ is time-invariant. Therefore, we can express $\mathbb{E}_{v\in \bold{V}_{yt}} [ M_v^{pmp (k+1)}] =\mu_{M}^{pmp(k+1)}(y)$, where $\bold{V}_{yt}= \{u\in \bold{V} \mid u\text{ has time } t, u \text{ has label } y\}$.

% We will analyze how the covariance matrix of the aggregated message at node $v$ with time $t$, and label $y$, and define affine transformations to make them time-invariant. We derive estimated 2nd moment of aggregated message rigorously, as shown in Appendix \ref{apdx:secondmm}.

% \begin{align}
% \hat{\text{var}}(M_{v}^{pmp(k+1)})=\mathbb{E}\left[(M_v ^{pmp (k+1)}-\mu_{M}^{pmp(k+1)}(y))(M_v ^{pmp (k+1)}-\mu_{M}^{pmp(k+1)}(y))^{\top}\right]
% \end{align}
% We assume independence between representations from the previous layer. Suppose that the 2nd moment of representations from the previous layer is invariant. In other words, if $\text{var}(X_v^{(k)})=\text{var}(X_j^{(k)})\text{ s.t. }y=\tilde y$, then we can denote the 2nd moment as $\text{var}(X_v^{(k)})=\Sigma_{XX}^{pmp(k)}(y)$. Then 2nd moment of the aggregated message through \PMP is as follows: 
% \begin{align}
% \hat{\text{var}}(M_{v}^{pmp(k+1)}) = {\sum_{\tilde y\in \bold{Y}}\left(\sum_{\tilde t\in\bold{T}_{t}^{\text{single}}}4\mathcal{P}_{y t}(\tilde y, \tilde t)+\sum_{\tilde t\in\bold{T}_{t}^{\text{double}}}\mathcal{P}_{y t}(\tilde y, \tilde t)\right)\Sigma_{XX}^{pmp(k)}(\tilde y)
% \over
% \left(\sum_{\tilde y\in \bold{Y}}\sum_{\tilde t\in\bold{T}_{t}^{\text{single}}}2\mathcal{P}_{y t}(\tilde y, \tilde t)+\sum_{\tilde y\in \bold{Y}}\sum_{\tilde t\in\bold{T}_{t}^{\text{double}}}\mathcal{P}_{y t}(\tilde y, \tilde t)\right)^2}
% \end{align}

% This value depends not only on the label $y$ of the target node but also on $t$. Therefore, we can express $\text{var}(M_{v}^{pmp(k+1)})=\Sigma^{pmp(k+1)}_{MM}(y,t)$. Let's design an affine transformation to make it invariant over time. For a time $t$ where $t \neq t_{max}$ and for any $y$, generally $\Sigma^{pmp(k+1)}_{MM}(y,t)\neq\Sigma^{pmp(k+1)}_{MM}(y,t_{max})$.

% Since the covariance matrix is always positive semi-definite, we can always orthogonally diagonalize it as $\Sigma^{pmp(k+1)}_{MM}(y,t)=U_t\Lambda_t U_t^{-1}$ and $\Sigma^{pmp(k+1)}_{MM}(y,t_{max})=U_{t_{max}}\Lambda_{t_{max}} U_{t_{max}}^{-1}$, where the diagonal elements of $\Lambda_{t}$ and $\Lambda_{t_{max}}$ are non-negative. Therefore, when $\hat{\text{var}}(M_v^{pmp (k+1)})=\Sigma^{pmp(k+1)}_{MM}(y,t)$, $\mathbb E[M_v^{pmp(k+1)}]=\mu_M^{pmp (k+1)}({y})$, we can define the following affine transformation:

% $M_{v}^{PNY(k+1)}\leftarrow A_{t} (M_v^{pmp(k+1)}-\mu_{M}^{pmp(k+1)}(y))+\mu_{M}^{pmp(k+1)}(y)$

% At this point, it can be easily shown that $\mathbb{E}[M_{v}^{PNY(k+1)}]=\mu_{M}^{pmp(k+1)}(y)$ and $\hat{\text{var}}(M_{v}^{PNY(k+1)})=A_{t}\Sigma^{pmp(k+1)}_{MM}(y,t)A{t}^{\top} = \Sigma^{pmp(k+1)}_{MM}(y,t_{max})$. In other words, if we can estimate $\Sigma^{pmp(k+1)}_{MM}(y,t)$ for any $y\in \bold{Y}, \ t\in \bold{T}$, then through affine transformation, we can make the 2nd moment of aggregated messages invariant over node time.


% Based on the above estimations, we can formulate an estimator for ${\Sigma}_{MM}^{pmp(k+1)}(y, t)$ as follows.

% \begin{align}
% \hat{\Sigma}^{pmp(k+1)}_{MM}(y,t) = {\sum_{\tilde y\in \bold{Y}}\left(\sum_{\tilde t\in\bold{T}_{t}^{\text{single}}}4\hat{\mathcal{P}}_{y t}(\tilde y, \tilde t)+\sum_{\tilde t\in\bold{T}_{t}^{\text{double}}}\hat{\mathcal{P}}_{y t}(\tilde y, \tilde t)\right)\hat\Sigma_{XX}^{pmp(k)}(\tilde y)
% \over
% \left(\sum_{\tilde y\in \bold{Y}}\sum_{\tilde t\in\bold{T}_{t}^{\text{single}}}2\hat{\mathcal{P}}_{y t}(\tilde y, \tilde t)+\sum_{\tilde y\in \bold{Y}}\sum_{\tilde t\in\bold{T}_{t}^{\text{double}}}\hat{\mathcal{P}}_{y t}(\tilde y, \tilde t)\right)^2}
% \end{align}

% Then, $\hat\Sigma^{pmp(k+1)}_{MM}(y,t)=\hat U_{y t}\hat \Lambda_{y t} \hat U_{y t}^{-1}$, $\hat\Sigma^{pmp(k+1)}_{MM}(y,t_{max})=\hat U_{y t_{max}}\hat \Lambda_{yt_{max}} \hat U_{yt_{max}}^{-1}$ can be orthogonally diagonalized.

% Suppose we have all estimation $\hat{\mathcal P}_{y t}(\tilde y, \tilde t)$ for all $t, \tilde t \in \bold{T}$ and $y, \tilde y\in \bold{Y}$, as explained in \ref{apdx:rel_con}. Than, the \PNY transform can be expressed as follows.

% \begin{align}
% M_v^{PNY(k+1)}\leftarrow  \hat U_{y t_{max}}\hat \Lambda_{y t_{max}}^{1/2}\hat \Lambda_{y t}^{-1/2}\hat U_{y t}^{\top}(M_v^{pmp (k+1)}-\hat\mu_{M}^{pmp(k+1)}(y))+\hat \mu_{M}^{pmp(k+1)}(y)
% \end{align}

% As proven earlier, when the representation in the previous layer has 1st moment and 2nd moment invariant to the node's time, using \PMP and \PNY transform yields $\mathbb{E}[M_v^{PNY(k+1)}]=\mu_{M}^{pmp(k+1)}(y)$ and $\text{var}(M_v^{PNY(k+1)})=\Sigma^{pmp(k+1)}_{MM}(y,t_{max})$, ensuring that both the 1st order moment and 2nd order moment in the aggregated message become invariant to the node's time.


\begin{algorithm}
    \caption{\name PNY transformation}
    \label{alg:PNY}
            \SetKwInOut{Input}{Input}\SetKwInOut{Output}{Output}
        \Input{~Previous layer's representation $X_v, \forall v\in \bold{V}$; Aggregated message $M_v, \forall v\in \bold{V}$, obtained from 1st moment alignment message passing; node time function $time:\bold{V}\rightarrow \bold{T}$; train, test split $\bold{V}^{tr}=\{v\mid v\in \bold{V}, time(v) < t_{\max}\}$ and $\bold{V}^{te}=\{v\mid v\in \bold{V}, time(v) = t_{\max}\}$; node label funtion $label:\bold{V}^{tr} \rightarrow \bold{Y}$; Estimated relative connectivity $\hat{\mathcal{P}}_{y, t}(\tilde y ,\tilde t)$, $\forall y, \tilde y\in \bold{Y},\ t, \tilde t \in \bold{T}$.}
        \BlankLine
        \Output{~Modified aggregated message $M_v', \forall v\in \bold{V}$}
        \BlankLine
        \BlankLine
        
        Let $\bold{V}_{y,t} = \{u \in \bold{V} \mid label(u)=y, time(u)=t\}$\;
        Let $\bold{V}_{\cdot,t} = \{u \in \bold{V} \mid time(u)=t\}$\;
        Let $\bold{T}_{\tau}^{\text{single}}= \{t \in \bold{T}\ \big |\  t =\tau \text{ or }t<2\tau-t_{\max}\}$\;
        Let $\bold{T}_{\tau}^{\text{double}}= \{t \in \bold{T}\ \big | \ |t- \tau| \le|t_{\max}-\tau|, t\neq \tau\}$\;
        Let $|\mathcal{N}_{yt}|={1\over{|\bold{V}_{y,t}|}}\sum_{u\in\bold{V}_{y,t}}|\mathcal{N}_u|$\;
        
        \BlankLine
        \textbf{Estimate covariance matrices of previous layer's representation.}\\
        \For{$t \in \bold{T}$}{
            $\hat\mu_{X}(\cdot,t)\leftarrow \hat\mu_M(\cdot,t) ={1\over {\mid \bold{V}_{\cdot,t} \mid}}\sum_{v\in \bold{V}_{\cdot,t}}X_v$\;
            $\hat\Sigma_{XX}(y)\leftarrow {1\over |\bold{V}_{\cdot,t}|-1}\sum_{v \in \bold{V}_{\cdot,t}} (X_v-\hat\mu_{X}(\cdot,t))(X_v-\hat\mu_{X}(\cdot,t))^{\top}$\;
        }
        
        \BlankLine
        \textbf{Estimate covariance matrices of aggregated message.}\\
        \For{$y \in \bold{Y}$}{
            \For{$t \in \bold{T}$}{
                $\hat{\Sigma}_{MM}(y,t) \leftarrow {\sum_{\tilde y\in \bold{Y}}\left(\sum_{\tilde t\in\bold{T}_{t}^{\text{single}}}4\hat{\mathcal{P}}_{y, t}(\tilde y, \tilde t)+\sum_{\tilde t\in\bold{T}_{t}^{\text{double}}}\hat{\mathcal{P}}_{y, t}(\tilde y, \tilde t)\right)\hat\Sigma_{XX}(\tilde y)\over\left(\sum_{\tilde y\in \bold{Y}}\sum_{\tilde t\in\bold{T}_{t}^{\text{single}}}2\hat{\mathcal{P}}_{y, t}(\tilde y, \tilde t)+\sum_{\tilde y\in \bold{Y}}\sum_{\tilde t\in\bold{T}_{t}^{\text{double}}}\hat{\mathcal{P}}_{y, t}(\tilde y, \tilde t)\right)^2|\mathcal{N}_{yt}|}$\;
            }
        }
    
        \BlankLine
        \textbf{Orthogonal diagonalization.}\\
        \For{$y \in \bold{Y}$}{
            \For{$t \in \bold{T}$}{
                Find $\hat P_{y, t},\ \hat D_{y,t}$ s.t. $\hat\Sigma_{MM}(y,t)=\hat P_{y, t}\hat D_{y,t} \hat P_{y, t}^{-1}$ and $\hat P_{y, t}^{-1}=\hat P_{y, t}^{\top}$\;
            }
        }
        
        \BlankLine
        \textbf{Update aggregated message.}\\
        \For{$v \in \bold{V}\setminus\bold{V}_{\cdot,t_{\max}}$}{
            Let $y= label(v)$\;
            Let $t= time(v)$\;
            $M_v^{'} \leftarrow  \hat P_{y, t_{\max}}\hat D_{y, t_{\max}}^{1/2}\hat D_{y, t}^{-1/2}\hat P_{y, t}^{\top}(M_v-\hat\mu_{M}(y))+\hat \mu_{M}(y)$\;
        }
    \end{algorithm}
    










\subsection{Explanation of \JJnorm}
\subsubsection{1st and 2nd moment of aggregated message obtained through \JJnorm.}\label{apdx:moments_in_jjnorm}
We define the first moment of \JJnorm message as the following steps:\\
\textbf{1st moment of aggregated message obtained through \JJnorm.}\\
(a) Take the expectation of the averaged message.\\
(b) Approximate the expectation of every \PMP message to the 1st moment of \PMP message.\\
\begin{align}
\hat{\hat{\mathbb{E}}}[M_v ^{JJ}]&\overset{(a)}=\mathbb{E}[\alpha_t (M_v ^{pmp(K)}-\mu_M^{JJ}(y,t))+\mu_M^{JJ}(y,t)]\\
&=\alpha_t \mathbb{E} [M_v^{pmp(K)}]+(1-\alpha_t)\mathbb{E}[\mu_M^{JJ}(y,t)]\\
&=\alpha_t \mathbb{E} [M_v^{pmp(K)}]+(1-\alpha_t){1\over{|\bold{V}_{y,t}|}}\mathbb{E}\left[\sum_{x\in\bold{V}_{y,t}}M^{pmp(K)}_w\right]\\
&=\alpha_t \mathbb{E}[M^{pmp(K)}]+(1-\alpha_t){1\over{|\bold{V}_{y,t}|}}\sum_{x\in\bold{V}_{y,t}}\mathbb{E}\left[M^{pmp(K)}_w\right]\\
&\overset{(b)}=\alpha_t \mathbb{E}[M^{pmp(K)}]+(1-\alpha_t){1\over{|\bold{V}_{y,t}|}}\sum_{x\in\bold{V}_{y,t}}\hat{\mathbb{E}}\left[M^{pmp(K)}_w\right]\\
&=\alpha_t \mathbb{E}[M^{pmp(K)}]+(1-\alpha_t){1\over{|\bold{V}_{y,t}|}}\sum_{w\in\bold{V}_{y,t}}\mu_M^{pmp(K)}(y)\\
&=\alpha_t \mu_M^{pmp(K)}(y)+(1-\alpha_t) \mu_M^{pmp(K)}(y)\\
&=\mu_M^{pmp(K)}(y)
\end{align}


\textbf{2nd moment of aggregated message obtained through \JJnorm.}\\
We define the second moment of \JJnorm message as the following steps:\\

(a) Take the variance of the averaged message.\\
(b) Consider $\mu_M^{JJ} (y, t)$ as a constant.\\
(c) Approximate the variance of \PMP message to the 2nd moment of \PMP message.\\

\begin{align}
\hat{\hat{\text{var}}}(M_v^{JJ})\overset{(a)}&=  \text{var}(\alpha_t(M_v^{pmp(k)}-\mu_M^{JJ})+\mu_M^{JJ}(y,t))\\
&\overset{(b)}=\text{var}(\alpha_tM_v^{pmp(K)})\\
&=\alpha_t^2 \text{var}(M_v^{pmp(K)})\\
&\overset{(c)}=\alpha_t^2 \hat{\text{var}}(M_v^{pmp(K)})\\
&=\alpha_t^2 \Sigma_{MM}^{pmp(K)}(y,t)
\end{align}



\subsubsection{Proof of Lemma \ref{lem:jj}}\label{apdx:JJnormlemma}
Consider GNNs with linear semantic aggregation functions.
\begin{align}
	&M_v^{pmp(k+1)} \leftarrow \text{\PMP}(X_w^{pmp(k)},w\in \mathcal{N}_v)\\
	&X_v^{pmp(k+1)} \leftarrow A^{(k+1)}M_v^{pmp(k+1)}, \ \forall k<K, v\in \bold{V}
\end{align}
	
Let's use mathematical induction. First, for initial features, $\Sigma_{XX}^{pmp (0)}(y,t_{max})= \Sigma_{XX}^{pmp (0)}(y,t)$ holds. Suppose that in the $k$-th layer, representation $X^{(k)}$ satisfies $\beta_{t}^{(k)}\Sigma_{XX}^{pmp (k)}(y,t_{max})= \Sigma_{XX}^{pmp (k)}(y,t)$. This assumes that the expected covariance matrix of representations of nodes with identical labels but differing time information only differs by a constant factor.



\vspace{-15pt}

\begin{align}
\Sigma^{pmp(k+1)}_{MM}(y,t) = {\sum_{\tilde{y}\in \bold{Y}}\left(\sum_{\tilde{t}\in\bold{T}_{t}^{\text{single}}}4\mathcal{P}_{yt}\left(\tilde{y}, \tilde{t}\right)+\sum_{\tilde{t}\in\bold{T}_{t}^{\text{double}}}\mathcal{P}_{yt}\left(\tilde{y}, \tilde{t}\right)\right)\Sigma_{XX}^{pmp(k)}(\tilde{y})
\over
\left(\sum_{\tilde{y}\in \bold{Y}}\sum_{\tilde{t}\in\bold{T}_{t}^{\text{single}}}2\mathcal{P}_{yt}\left(\tilde{y}, \tilde{t}\right)+\sum_{\tilde{y}\in \bold{Y}}\sum_{\tilde{t}\in\bold{T}_{t}^{\text{double}}}\mathcal{P}_{yt}\left(\tilde{y}, \tilde{t}\right)\right)^2{|\mathcal{N}_{v}|}
\end{align}


\vspace{-5pt}


\begin{align}
\Sigma^{pmp(k+1)}_{MM}(y,t) &= {\sum_{\tilde y\in \bold{Y}}\left(\sum_{\tilde t\in\bold{T}_{t}^{\text{single}}}4\mathcal{P}_{y t}(\tilde y, \tilde t)\Sigma^{pmp(k)}_{XX}(\tilde y,\tilde t)+\sum_{\tilde t\in\bold{T}_{t}^{\text{double}}}\mathcal{P}_{y t}(\tilde y, \tilde t)\Sigma^{pmp(k)}_{XX}(\tilde y,\tilde t)\right)
\over
\left(\sum_{\tilde y\in \bold{Y}}\left(\sum_{\tilde t\in\bold{T}_{t}^{\text{single}}}2\mathcal{P}_{y t}(\tilde y, \tilde t)+\sum_{\tilde t\in\bold{T}_{t}^{\text{double}}}\mathcal{P}_{y t}(\tilde y, \tilde t)\right)\right)^2 |\mathcal N_{yt}|}\\
 &= {\sum_{\tilde y\in \bold{Y}}\left(\sum_{\tilde t\in\bold{T}_{t}^{\text{single}}}4\mathcal{P}_{y t}(\tilde y, \tilde t)\beta_{t}^{(k)}+\sum_{\tilde t\in\bold{T}_{t}^{\text{double}}}\mathcal{P}_{y t}(\tilde y, \tilde t)\beta_{t}^{(k)}\right)\Sigma_{XX}^{pmp (k)}(y,t_{max})
\over
\left(\sum_{\tilde y\in \bold{Y}}\left(\sum_{\tilde t\in\bold{T}_{t}^{\text{single}}}2\mathcal{P}_{y t}(\tilde y, \tilde t)+\sum_{\tilde t\in\bold{T}_{t}^{\text{double}}}\mathcal{P}_{y t}(\tilde y, \tilde t)\right)\right)^2 |\mathcal N_{yt}|}
\end{align}

% By Assumption 4, following value is invariant to $y$.

% \begin{align}
% {\sum_{t\in\bold{T}_{t}^{\text{single}}}4\mathcal{P}_{y t}(y, t)\beta_t^{(k)}+\sum_{t\in\bold{T}_{t}^{\text{double}}}\mathcal{P}_{y t}(y, t)\beta_t^{(k)}\over \sum_{t\in\bold{T}}4\mathcal{P}_{y t}(y, t)\beta_t^{(k)}}=\gamma_{t}^{(k)}
% \end{align}

% Furthermore, using the previously defined $\lambda_{t}$,
\begin{align}
&{\sum_{\tilde t \in\bold{T}_{t}^{\text{single}}}4\mathcal{P}_{yt}(\tilde y,\tilde t)\beta_{\tilde t}^{(k)}+\sum_{\tilde t \in\bold{T}_{t}^{\text{double}}}\mathcal{P}_{yt}(\tilde y,\tilde t)\beta_{\tilde t}^{(k)}\over\sum_{\tilde t \in\bold{T}}4\mathcal{P}_{yt_{max}}(\tilde y,\tilde t)\beta_{\tilde t}^{(k)}}\\
&={\sum_{\tilde t \in\bold{T}_{t}^{\text{single}}}4g(y,\tilde y, |\tilde t -t|)\beta_{\tilde t}^{(k)}+\sum_{\tilde t \in\bold{T}_{t}^{\text{double}}}g(y,\tilde y, |\tilde t -t|)\beta_{\tilde t}^{(k)}\over\sum_{\tilde t \in\bold{T}}4g(y,\tilde y, |\tilde t -t_{max}|)\beta_{\tilde t}^{(k)}}
\end{align}

Since it is unrelated to $y$ by Assumption 4, we can define it as $\gamma_t^{(k)}$.

\begin{align}
&{\sqrt{|\mathcal{N}_{yt}|}\over \sqrt{|\mathcal{N}_{yt_{max}}|}}{\sum_{\tilde t \in\bold{T}_{t}^{\text{single}}}2\mathcal{P}_{yt}(\tilde y,\tilde t)+\sum_{\tilde t \in\bold{T}_{t}^{\text{double}}}\mathcal{P}_{yt}(\tilde y,\tilde t)\over\sum_{\tilde t \in\bold{T}}2\mathcal{P}_{yt_{max}}(\tilde y,\tilde t)}\\
&\overset{(c)}={\sqrt{P(t)}\over \sqrt{P(t_{max})}}{\sum_{\tilde t \in\bold{T}_{t}^{\text{single}}}2\mathcal{P}_{yt}(\tilde y,\tilde t)+\sum_{\tilde t \in\bold{T}_{t}^{\text{double}}}\mathcal{P}_{yt}(\tilde y,\tilde t)\over\sum_{\tilde t \in\bold{T}}2\mathcal{P}_{yt_{max}}(\tilde y,\tilde t)}\\
&={\sqrt{P(t)}\over \sqrt{P(t_{max})}}{\sum_{\tilde t \in\bold{T}_{t}^{\text{single}}}2g(y,\tilde y,|\tilde t-t|)+\sum_{\tilde t \in\bold{T}_{t}^{\text{double}}}g(y,\tilde y,|\tilde t-t|)\over\sum_{\tilde t \in\bold{T}}2g(y,\tilde y,|\tilde t-t_{max}|)(\tilde y,\tilde t)}
\end{align}

Since it is unrelated to $y$ by assumption 4, we can define it as $\lambda_t$.

1st equality holds by step (c) of 2nd moment of \PMP, as defined in Appendix \ref{apdx:secondmm}

{\scriptsize
\begin{align}
\Sigma^{pmp(k+1)}_{MM}(y,t) = {\gamma_{t}^{(k)}\over\lambda_{t}^2} {\sum_{\tilde y\in \bold{Y}}\sum_{\tilde t\in\bold{T}}4\mathcal{P}_{y t}(\tilde y, \tilde t)\beta_t^{(k)}\Sigma_{XX}^{pmp (k)}(\tilde y,t_{max})
\over
\left(\sum_{\tilde y\in \bold{Y}}\sum_{\tilde t\in\bold{T}}2\mathcal{P}_{y t}(\tilde y, \tilde t)\right)^2} = {\gamma_{t}^{(k)}\over\lambda_{t}^2} \Sigma^{pmp(k+1)}_{MM}(y,t_{max}) 
\end{align}
}%

Using $T_{t_{max}}^{\text{double}} = \phi$, 

\begin{align}
\Sigma^{pmp(k+1)}_{MM}(y,t) = {\gamma_{t}^{(k)}\over\lambda_{t}^2} \Sigma^{pmp(k+1)}_{MM}(y,t_{max}) 
\end{align}

Since $X_v^{(k+1)}=A^{(k+1)}M_v^{(k+1)}$, the following equation holds.

\begin{align}
\Sigma^{pmp(k+1)}_{XX}(y,t)&= A^{(k+1)}\Sigma^{pmp(k+1)}_{MM}(y,t)A^{(k+1)\top}\\
&=A^{(k+1)}{\gamma_{t}^{(k)}\over\lambda_{t}^2}\Sigma^{pmp(k+1)}_{MM}(y,t_{max})A^{(k+1)\top} \\
&={\gamma_{t}^{(k)}\over\lambda_{t}^2} \Sigma^{pmp(k+1)}_{XX}(y,t_{max}) 
\end{align}

Therefore, we proved that if $\beta_{t}^{(k)}\Sigma_{XX}^{pmp (k)}(y,t_{max})= \Sigma_{XX}^{pmp (k)}(y,t)$ holds for $k$, then for constants $\gamma_{t}^{(k)}, \lambda_{t}, \beta_{t}^{(k+1)}$ which depends only on time and layer, $\Sigma^{pmp(k+1)}_{MM}(y,t) = {\gamma_{t}^{(k)}\over\lambda_{t}^2} \Sigma^{pmp(k+1)}_{MM}(y,t_{max})$ and $\beta_{t}^{(k+1)}\Sigma_{XX}^{pmp (k+1)}(y,t_{max})= \Sigma_{XX}^{pmp (k+1)}(y,t)$ holds. By induction, lemma is proved.


\subsubsection{Proof of Theorem \ref{thm:jj}}
\label{apdx:JJnorm}
In this discussion, we will regard $\mu_M^{JJ}(\cdot,t)$ and $\mu_M^{JJ}(y,t)$ as constant, since generally there are sufficient number of samples in each community, especially for large-scale graphs. \\
As shown earlier, when passing through \PMP, the covariance matrix of the aggregated message is as follows.
\begin{figure}[hbt!]
	\centering
	\includegraphics[width=0.99\textwidth]{figs/JJ_norm_hor.png}
	\vspace{-0.1in}
	\caption{Graphical explanation of \JJnorm. Under assumption 4, covariance matrices of aggregated message on each community differs only by a constant factor $\alpha_t$.}
	 \label{fig:JJ}
\end{figure}

% \begin{align}
% \Sigma^{pmp(k+1)}_{MM}(y,t) = {\sum_{\tilde y\in \bold{Y}}\left(\sum_{\tilde t\in\bold{T}_{t}^{\text{single}}}4\mathcal{P}_{y t}(\tilde y, \tilde t)+\sum_{\tilde t\in\bold{T}_{t}^{\text{double}}}\mathcal{P}_{y t}(\tilde y, \tilde t)\right)\Sigma^{pmp(k)}_{XX}(\tilde y)
% \over
% \left(\sum_{\tilde y\in \bold{Y}}\left(\sum_{\tilde t\in\bold{T}_{t}^{\text{single}}}2\mathcal{P}_{y t}(\tilde y, \tilde t)+\sum_{\tilde t\in\bold{T}_{t}^{\text{double}}}\mathcal{P}_{y t}(\tilde y, \tilde t)\right)\right)^2}
% \end{align}

% However, when $t=t_{max}$, $\bold{T}_{t_{max}}^{\text{double}}= \{t \in \bold{T}\ \big | \ |t- t_{max}|\le |t_{max}-t_{max}|, t\neq t_{max}\}=\phi$, making the covariance matrix simpler as follows.

% \begin{align}
% \Sigma^{pmp(k+1)}_{MM}(y,t_{max})= {\sum_{\tilde y\in \bold{Y}}\sum_{\tilde t\in\bold{T}}4\mathcal{P}_{y t}(\tilde y, \tilde t)\Sigma_{XX}^{pmp(k)}(\tilde y)
% \over
% \left(\sum_{\tilde y\in \bold{Y}}\sum_{\tilde t\in\bold{T}}2\mathcal{P}_{y t}(\tilde y, \tilde t)\right)^2}
% \end{align}

% To examine how the covariance matrix varies with time, let's consider the following two ratios.

% \begin{flalign}
% {\sum_{\tilde t\in\bold{T}_{t}^{\text{single}}}4\mathcal{P}_{y t}(\tilde y, \tilde t)+\sum_{\tilde t\in\bold{T}_{t}^{\text{double}}}\mathcal{P}_{y t}(\tilde y, \tilde t)\over \sum_{\tilde t\in\bold{T}}4\mathcal{P}_{y t}(\tilde y, \tilde t)}\\={4g(y, \tilde y, 0)+2\sum_{0<\tau\le |t_{max}-t|}g(y, \tilde y, \tau)+4\sum_{|t_{max}-t|<\tau}g(y, \tilde y, \tau)\over 4\sum_{0\le\tau}g(y,\tilde y,\tau)}=\gamma_{t}
% \end{flalign}

% \begin{flalign}
% {\sum_{\tilde t\in\bold{T}_{t}^{\text{single}}}2\mathcal{P}_{y t}(\tilde y, \tilde t)+\sum_{\tilde t\in\bold{T}_{t}^{\text{double}}}\mathcal{P}_{y t}(\tilde y,\tilde t)\over \sum_{\tilde t\in\bold{T}}2\mathcal{P}_{y t}(\tilde y, \tilde t)}\\={2g(y, \tilde y, 0)+\sum_{0<\tau\le |t_{max}-t|}g(y, \tilde y, \tau)+2\sum_{|t_{max}-t|<\tau}g(y, \tilde y, \tau)\over 2\sum_{0\le\tau}g(y,\tilde y,\tau)}=\lambda_{t}
% \end{flalign}

% Here, we can denote these values as $\gamma_{t}$ and $\lambda_{t}$ because the value of $g(y, y, \tau)$ is invariant to $y$ and $y$ due to Assumption 4. Utilizing this, we can transform the equation as follows:

% \begin{align}
% \Sigma^{pmp(k+1)}_{MM}(y,t)= {\gamma_{t} \over \lambda_{t}^2}\Sigma^{pmp(k+1)}_{MM}(y,t_{max})
% \end{align}

% In other words, when Assumption 4 holds true, the covariance matrix of the aggregated message differs only by a constant factor, and this constant depends solely on the node's time. For simplicity, let's define $\alpha_{t} = {\lambda_{t}^2 \over \gamma_t}$, then we can express it as follows:

% \begin{align}
% \Sigma^{pmp(k+1)}_{MM}(y,t_{max})=\alpha_{t}\Sigma^{pmp(k+1)}_{MM}(y,t)
% \end{align}

Unlike \PNY, which estimates an affine transformation using $\hat{\mathcal{P}}_{y t}(\tilde y, \tilde t)$ to align the covariance matrix to be invariant, \JJnorm provides a more direct method to obtain an estimate $\hat{\alpha}_{t}$ of $\alpha_{t}$. Since the objective of this section is to get a sufficiently good estimator for implementation, the equations here may be heuristic but are proceeded with intuitive reasons.

Since we know that the covariance matrix differs only by a constant factor, we can simply use norms in multidimensional space rather than the covariance matrix to estimate $\alpha_{t}$.


Firstly, let's define $\bold{V}_{y,t} = \{u \in \bold{V} \mid u \text{ has label }y, u\text{ has time }t\}$, $\bold{V}_{\cdot,t} = \{u \in \bold{V} \mid u\text{ has time }t\}$.

% We can compute the mean of the aggregated message for each label and time: define $\mu_M(t) = \mathbb{E}_{v\in \bold{V}_{\cdot,t}}\left[M_v\right]$ and $\mu_M(y,t) = \mathbb{E}_{v\in \bold{V}_{y,t}}\left[M_v\right]$. 
Let us define
\begin{align}
\sigma_{y,t}^2=\mathbb{E}_{v\sim \bold{V}_{y,t}}[(M_v-\mu_M(y,t))^2]={1\over|\bold{V}_{y,t}|}\sum_{v\in\bold{V}_{y,t}}(M_v - \mu_{M}(y,t))^2
\end{align}

\begin{align}
\sigma_{\cdot,t}^2=\mathbb{E}_{v\sim \bold{V}_{\cdot,t}}[(M_v-\mu_M(t))^2]={1\over|\bold{V}_{\cdot,t}|}\sum_{v\in\bold{V}_{\cdot,t}}(M_v - \mu_{M}(t))^2
\end{align}

\begin{align}
\mu_{y,t}=\mathbb{E}_{v\sim \bold{V}_{y,t}}[M_v]={1\over|\bold{V}_{y,t}|}\sum_{v\in\bold{V}_{y,t}}M_v
\end{align}

\begin{align}
\mu_{y,t}=\mathbb{E}_{v\sim \bold{V}_{\cdot,t}}[M_v]={1\over|\bold{V}_{y,t}|}\sum_{v\in\bold{V}_{\cdot,t}}M_v
\end{align}

Note that definition of mean and variance here, are different with the definitions stated in \ref{apdx:moments_in_jjnorm}. Here, \JJnorm is a process of transforming the aggregated message, which is aggregated through \PMP, into a time-invariant representation. Hence, we can suppose that $\mu_M(y,t)$ is invariant to $t$. That is, for all $t\in\bold{T}$, $\mu_M(y,t)=\mu_M(y,t_{max})$. Additionally, we can define the variance of distances as follows: $\sigma_{y,t}^2=\mathbb{E}_{v\in \bold{V}_{y,t}}\left[(M_v-\mu_M(y,t))^2\right]$ and $\sigma_{\cdot,t}^2=\mathbb{E}_{v\in \bold{V}_{\cdot,t}}\left[(M_v-\mu_M(t))^2\right]$. Here, the square operation denotes the L2-norm.

\begin{flalign}
\mathbb{E}_{v\in \bold{V}_{\cdot,t}}\left[(M_v-\mu_M(t))^2\right] = \sum_{y\in \bold{Y}}P(y)\mathbb{E}_{v\in \bold{V}_{y,t}}\left[ (M_v - \mu_M(y,t)+\mu_M(y,t)-\mu_M(t))^2\right]\\=\sum_{y\in \bold{Y}}P(y)\Big(\mathbb{E}_{v\in \bold{V}_{y,t}}\left[ (M_v - \mu_M(y,t))^2 \right] +(\mu_M(y,t)-\mu_M(t))^2\Big)
\end{flalign}

Since $\mathbb{E}_{v\in \bold{V}_{y,t}}\left[ (M_v - \mu_M(y,t))^{\top}(\mu_M(y,t)-\mu_M(t))\right]=0$.

Here, mean of the aggregated messages during training and testing times satisfies the following equation: $\mu_M(t) = \mu_M(t_{max})$

\begin{align}
\mu_M(t)=\sum_{y\in\bold{Y}}P(y)\mu_M(y,t)=\sum_{y\in\bold{Y}}P(y)\mu_M(y,t_{max})=\mu_M(t_{max})
\end{align}

This equation is derived from the assumption that $\mu_M(y,t)$ is invariant to $t$ and from Assumption 1 regarding $P(y)$. Furthermore, by using Assumption 1 again, we can show that the variance of the mean computed for each label is also invariant to $t$:

\begin{align}
\sum_{y\in\bold{Y}}P(y)\mathbb{E}_{v\in \bold{V}_{y,t}}\left[(\mu_M(y,t)-\mu_M(t))^2 \right]=\sum_{y\in\bold{Y}}P(y)\mathbb{E}_{v\in \bold{V}_{y,t_{max}}}\left[(\mu_M(y,t_{max})-\mu_M(t_{max}))^2 \right]
\end{align}

\begin{align}
\mathbb{E}_{v\in \bold{V}_{y,t}}\left[(\mu_M(y,t)-\mu_M(t))^2 \right]=\mathbb{E}_{v\in \bold{V}_{y,t_{max}}}\left[(\mu_M(y,t_{max})-\mu_M(t_{max}))^2\right] =\nu^2,\ t\in \bold{T}
\end{align}

Here, $\nu^2$ can be interpreted as the variance of the mean of messages from nodes with the same $t\in \bold{T}$ for each label. According to the above equality, this is a value invariant to $t$.

Meanwhile, from Assumption 4,

\begin{align}
\alpha_t \mathbb{E}_{v\in \bold{V}_{y,t}}\left[ (M - \mu_M(y,t))^2 \right] = \mathbb{E}_{v\in \bold{V}_{y,t_{max}}}\left[ (M - \mu_M(y,t_{max}))^2\right], \forall t\in \bold{T}
\end{align}

\begin{align}
\alpha_t\sum_{y\in\bold{Y}}P(y)\mathbb{E}_{v\in \bold{V}_{y,t}}\left[ (M_v - \mu_M(y,t))^2 \right]=\sum_{y\in\bold{Y}}P(y)\mathbb{E}_{v\in \bold{V}_{y,t_{max}}}\left[ (M_v - \mu_M(y,t_{max}))^2\right]
\end{align}

Adding $\nu^2$ to both sides,

\begin{align}
\alpha_t\sum_{y\in \bold{Y}}P(y)\mathbb{E}_{v\in \bold{V}_{y,t}}\left[ (M_v - \mu_M(y,t))^2 \right] +\sum_{y\in \bold{Y}}P(y)\mathbb{E}_{v\in \bold{V}_{y,t}}\left[(\mu_M(y,t)-\mu_M(t))^2 \right] =\sigma_{\cdot,t_{max}}^2 
\end{align}

Thus,

\begin{align}
\alpha_t = { \sigma_{\cdot,t_{max}}^2  - \nu^2\over\sum_{y\in \bold{Y}}P(y)\mathbb{E}_{v\in \bold{V}_{y,t}}\left[ (M_v- \mu_M(y,t))^2 \right]}
\end{align}

Here, $\hat{\alpha}_t$ is an unbiased estimator of $\alpha_t$.

\begin{align}
\hat{\nu}^2={1\over \mid{\bold{V}_{\cdot,t}}\mid-1} \sum_{y\in \bold{Y}}\sum_{v \in \bold{V}_{y,t}}(\hat\mu_M(y,t) -\hat\mu_M(t) )^2  
\end{align}


\begin{align}
\hat{\alpha}_t = {\left( {1\over \mid{\bold{V}_{\cdot,t_{max}}}\mid-1}\sum_{v\in \bold{V}_{\cdot,t_{max}}}(M_v-\hat\mu_M(t_{max}))^2  -\hat{\nu}^2 \right)\over{1\over \mid{\bold{V}_{\cdot,t}}\mid-1} \sum_{y\in\bold{Y}} \sum_{v \in \bold{V}_{y,t}}(M_v-\hat\mu_M(y,t))^2}
\end{align}

Where $\hat\mu_M(y,t)={1\over {\mid \bold{V}_{y,t} \mid}}\sum_{v\in \bold{V}_{y,t}}M_v$  and $\hat\mu_M(t) ={1\over {\mid \bold{V}_{\cdot,t} \mid}}\sum_{v\in \bold{V}_{\cdot,t}}M_v$ . 

Note that all three terms in the above equation can be directly computed without requiring test labels.

By using $\hat{\alpha_t}$, we can update the aggregated message from \PMP to align the second-order statistics.

\begin{align}
\ M_v^{JJnorm} \leftarrow \hat\mu_M(y,t) +\hat{\alpha}_{t} (M_v - \hat\mu_M(y,t)),\ \forall i \in \bold{V}\setminus\bold{V}_{\cdot,t_{max}}
\end{align}


\begin{algorithm}
    \caption{\name JJ normalization}
    \label{alg:JJ-Norm}
        \SetKwInOut{Input}{Input}\SetKwInOut{Output}{Output}
        \Input{~Aggregated message $M_v, \forall v\in \bold{V}$, obtained from 1st moment alignment message passing; node time function $time:\bold{V}\rightarrow \bold{T}$; train, test split $\bold{V}^{tr}=\{v\mid v\in \bold{V}, time(v) < t_{\max}\}$ and $\bold{V}^{te}=\{v\mid v\in \bold{V}, time(v) = t_{\max}\}$; node label funtion $label:\bold{V}^{tr} \rightarrow \bold{Y}$.}
        \BlankLine
        \Output{~Modified aggregated message $M_v', \forall v\in \bold{V}$.}
        \BlankLine
        \BlankLine
        Let $\bold{V}_{y,t} = \{u \in \bold{V} \mid label(u)=y, time(u)=t\}$\;
        Let $\bold{V}_{\cdot,t} = \{u \in \bold{V} \mid time(u)=t\}$\;
    
        \BlankLine
        \textbf{Estimate mean and variance for each community.}\\
        \For{$t \in \bold{T}$}{
            $\hat\mu_{M}(\cdot,t)\leftarrow \hat\mu_M(\cdot,t) ={1\over {\mid \bold{V}_{\cdot,t} \mid}}\sum_{v\in \bold{V}_{\cdot,t}}M_v $\;
        }
        \For{$y \in \bold{Y}$}{
            \For{$t \in \{\dots,t_{\max}-1\}$}{
                $\hat\nu_t^2\leftarrow {1\over \mid{\bold{V}_{\cdot,t}}\mid-1} \sum_{y\in \bold{Y}}\sum_{v\in \bold{V}_{y,t}}(\hat\mu_M(y,t) -\hat\mu_M(\cdot,t) )^2 $\;
            }
        }
        \For{$y \in \bold{Y}$}{
            \For{$t \in \{\dots,t_{\max}-1\}$}{
                $\hat\mu_{M}(y,t)\leftarrow {1\over {\mid \bold{V}_{y,t} \mid}}\sum_{v\in \bold{V}_{y,t}}M_v $\;
                $\hat\sigma_{y,t}^2\leftarrow {1\over \mid{\bold{V}_{\cdot,t}}\mid-1} \sum_{y\in\bold{Y}} \sum_{v\in \bold{V}_{y,t}}(M_v-\hat\mu_M(y,t))^2$\;
            }
        }
    
        $\hat\sigma_{t_{\max}}^2\leftarrow {1\over \mid{\bold{V}_{\cdot,t_{\max}}}\mid-1}\sum_{v\in \bold{V}_{\cdot,t_{\max}}}(M_v-\hat\mu_M(\cdot,t_{\max}))^2  -{1\over \mid{\bold{V}_{\cdot,t}}\mid-1} \sum_{y\in \bold{Y}}\sum_{v\in \bold{V}_{y,t}}(\hat\mu_M(y,t) -\hat\mu_M(\cdot,t) )^2$\;
        
        \BlankLine
        \textbf{Estimate $\hat\alpha_t$ for $t<t_{\max}$.}\\
        \For{$t \in \{\dots,t_{\max}-1\}$}{
            $\hat\alpha_t^2 \leftarrow {\hat\sigma_{t_{\max}}^2 - \hat\nu_t^2 \over \hat\sigma_{y,t}^2}$\;
        }
        
        \BlankLine
        \textbf{Update aggregated message.}\\
        \For{$v\in \bold{V}\setminus\bold{V}_{\cdot,t_{\max}}$}{
            Let $y= label(i)$\;
            Let $t= time(i)$\;
            $M_v^{'} \leftarrow \hat\mu_M(y,t) +\hat{\alpha}_{t} (M_v - \hat\mu_M(y,t)),\ \forall v\in \bold{V}\setminus\bold{V}_{\cdot,t_{\max}}$\;
        }
    \end{algorithm}






% \subsection{Lazy operation property of \JJnorm}
% \label{apdx:Lazy}
% Let's use mathematical induction. First, for initial features, $\Sigma_{XX}^{pmp (0)}(y,t_{max})= \Sigma_{XX}^{pmp (0)}(y,t)$ holds. Suppose that in the $k$-th layer, representation $X^{(k)}$ satisfies $\beta_{t}^{(k)}\Sigma_{XX}^{pmp (k)}(y,t_{max})= \Sigma_{XX}^{pmp (k)}(y,t)$. This assumes that the expected covariance matrix of representations of nodes with identical labels but differing time information only differs by a constant factor.

% \begin{align}
% \Sigma^{pmp(k+1)}_{MM}(y,t) = {\sum_{y\in \bold{Y}}\left(\sum_{t\in\bold{T}_{t}^{\text{single}}}4\mathcal{P}_{y t}(y, t)\Sigma^{pmp(k)}_{XX}(y,t)+\sum_{t\in\bold{T}_{t}^{\text{double}}}\mathcal{P}_{y t}(y, t)\Sigma^{pmp(k)}_{XX}(y,t)\right)
% \over
% \left(\sum_{y\in \bold{Y}}\left(\sum_{t\in\bold{T}_{t}^{\text{single}}}2\mathcal{P}_{y t}(y, t)+\sum_{t\in\bold{T}_{t}^{\text{double}}}\mathcal{P}_{y t}(y, t)\right)\right)^2}
% \end{align}

% \begin{align}
%  = {\sum_{y\in \bold{Y}}\left(\sum_{t\in\bold{T}_{t}^{\text{single}}}4\mathcal{P}_{y t}(y, t)\beta_{t}^{(k)}+\sum_{t\in\bold{T}_{t}^{\text{double}}}\mathcal{P}_{y t}(y, t)\beta_{t}^{(k)}\right)\Sigma_{XX}^{pmp (k)}(y,t_{max})
% \over
% \left(\sum_{y\in \bold{Y}}\left(\sum_{t\in\bold{T}_{t}^{\text{single}}}2\mathcal{P}_{y t}(y, t)+\sum_{t\in\bold{T}_{t}^{\text{double}}}\mathcal{P}_{y t}(y, t)\right)\right)^2}
% \end{align}

% By Assumption 4, following value is invariant to $y$.

% \begin{align}
% {\sum_{t\in\bold{T}_{t}^{\text{single}}}4\mathcal{P}_{y t}(y, t)\beta_t^{(k)}+\sum_{t\in\bold{T}_{t}^{\text{double}}}\mathcal{P}_{y t}(y, t)\beta_t^{(k)}\over \sum_{t\in\bold{T}}4\mathcal{P}_{y t}(y, t)\beta_t^{(k)}}=\gamma_{t}^{(k)}
% \end{align}

% Furthermore, using the previously defined $\lambda_{t}$,

% \begin{align}
% \Sigma^{pmp(k+1)}_{MM}(y,t) = {\gamma_{t}^{(k)}\over\lambda_{t}^2} {\sum_{y\in \bold{Y}}\sum_{t\in\bold{T}}4\mathcal{P}_{y t}(y, t)\beta_t^{(k)}\Sigma_{XX}^{pmp (k)}(y,t_{max})
% \over
% \left(\sum_{y\in \bold{Y}}\sum_{t\in\bold{T}}2\mathcal{P}_{y t}(y, t)\right)^2} = {\gamma_{t}^{(k)}\over\lambda_{t}^2} \Sigma^{pmp(k+1)}_{MM}(y,t_{max}) 
% \end{align}

% Since $X_v^{(k+1)}=A^{(k+1)}M_v^{(k+1)}$, the following equation holds.

% \begin{align}
% \Sigma^{pmp(k+1)}_{XX}(y,t)= A^{(k+1)}\Sigma^{pmp(k+1)}_{MM}(y,t)A^{(k+1)\top}=\\A^{(k+1)}{\gamma_{t}^{(k)}\over\lambda_{t}^2}\Sigma^{pmp(k+1)}_{MM}(y,t_{max})A^{(k+1)\top} ={\gamma_{t}^{(k)}\over\lambda_{t}^2} \Sigma^{pmp(k+1)}_{MM}(y,t_{max}) 
% \end{align}

% Here, $\beta_t^{(k+1)}$ is recursively defined as follows.

% \begin{align}
% \beta_t^{(k+1)} = {\gamma_{t}^{(k)}\over\lambda_{t}^2}={\sum_{t\in\bold{T}_{t}^{\text{single}}}4\mathcal{P}_{y t}(y, t)\beta_t^{(k)}+\sum_{t\in\bold{T}_{t}^{\text{double}}}\mathcal{P}_{y t}(y, t)\beta_t^{(k)}\over \lambda_{t}^2\sum_{t\in\bold{T}}4\mathcal{P}_{y t}(y, t)\beta_t^{(k)}}
% \end{align}

% Therefore, it is proven that $\beta_{t}^{(k)}\Sigma_{XX}^{pmp (k)}(y,t_{max})= \Sigma_{XX}^{pmp (k)}(y,t)$ holds for all representations for any $k\le K$.



\subsection{Detailed experimental setup for synthetic graph experiments.}
\label{apdx:synthetic_setup}
In our experiments, we set $f = 5$, $k_{y}$ was sampled from a uniform distribution in $[0, 8]$, and the center of features for each label $\mu(y) \in \mathbb{R}^f$ was sampled from a standard normal distribution. Each graph consisted of 2000 nodes, with a possible set of times $\bold{T} = \{0, 1, \dots, 9\}$ and a set of labels $\bold{Y} = \{0, 1, \dots, 9\}$, with time and label uniformly distributed. Therefore, the number of communities is 100, each comprising 20 nodes. Additionally, we defined $\bold{V}_{\text{te}} = \{u \in \bold{V} \mid u\text{ has time } \ge 8\}$ and $\bold{V}_{\text{tr}} = \{u \in \bold{V} \mid u\text{ has time } < 8\}$. When communities have an equal number of nodes, the following relationship holds:

\vspace{-15pt}
\begin{align}
    \bold{P}_{t \tilde t y \tilde y} = \gamma_{y, \tilde y}^{|t - \tilde t|} \bold{P}_{t \tilde t y \tilde y} ,\ \forall |t - \tilde t |>0
\end{align}
\vspace{-5pt}

To fully determine the tensor $\bold{P}_{t \tilde t y \tilde y}$, we needed to specify the values when $t = \tilde t$. In order to imbue the graph with topological information, we defined two hyperparameters, $\mathcal{K}$ and $\mathcal{G}$, such that $\mathcal{K} < \mathcal{G}$. For any $y, \tilde y \in \bold{Y}$, if $y = \tilde y$, we sampled $\mathcal{P}_{y, t, \tilde y, t}$ from a uniform distribution in $[0, \mathcal{K}]$, and if $y \ne \tilde y$, we sampled $\mathcal{P}_{y, t, \tilde y, t}$ from a uniform distribution in $[0, \mathcal{G}]$. In our experiments, we used $\mathcal{K} = 0.6$ and $\mathcal{G} = 0.24$. 

For cases where Assumption 4 was not satisfied, $\gamma_{y, \tilde y}$ was sampled from a uniform distribution $[0.4, 0.7]$. For cases where Assumption 4 was satisfied, all decay factors were the same, i.e., $\gamma_{y, \tilde y} = \gamma, \ \forall y, \tilde y \in \bold{Y}$. In this case, $\gamma$ indicates the extent to which the connection probability varies with the time difference between two nodes. A smaller $\gamma$ corresponds to a graph where the connection probability decreases drastically. We also compared the trends in the performance of each \IMPaCT method by varying the value of $\gamma$. The baseline SGC consisted of 2 layers of message passing and 2 layers of MLP, with the hidden layer dimension set to 16. The baseline GCN also consisted of 2 layers with the hidden layer dimension set to 16. Adam optimizer was used for training with a learning rate of $0.01$ and a weight decay of $0.0005$. Each model was trained for 200 epochs, and each data was obtained by repeating experiments on 200 random graph datasets generated through TSBM. The training of both models were conducted on a 2X Intel Xeon Platinum 8268 CPU with 48 cores and 192GB RAM. 


\subsection{Scalability of invariant message passing methods}
\label{apdx:scalability}
First moment alignment methods such as \MMP and \PMP have the same complexity and can be easily applied by modifying the graph. By adding or removing edges according to the conditions, only $\mathcal{O}(|E|)$ additional preprocessing time is required, which is necessary only once throughout the entire training process. If the graph cannot be modified and the message passing function needs to be modified instead, it would require $\mathcal{O}(|E|fK)$, which is equivalent to the traditional averaging message passing. Similarly, the memory complexity remains $\mathcal{O}(|E|fK)$, consistent with traditional averaging message passing. Despite having the same complexity, \PMP is much more expressive than \MMP. Unless there are specific circumstances, \PMP is recommended for first moment alignment.

In \PNY, estimating the relative connectivity $\hat{\mathcal{P}}_{y, t}(\tilde y, \tilde t)$ requires careful consideration. If both $t\neq t_{max}$ and $\tilde t\neq t_{max}$, calculating the relative connectivity for all pairs involves $\mathcal{O}((N+|E|)f)$ operations, while computing for cases where either time is $t_{max}$ requires $\mathcal{O}(|Y|^2|T|^2)$ computations. Therefore, the total time complexity becomes $\mathcal{O}(|Y|^2|T|^2+(N+|E|)f)$. Additionally, for each message passing step, the covariance matrix of the previous layer's representation and the aggregated message needs to be computed for each label-time pair. Calculating the covariance matrix of the representation from the previous layer requires $\mathcal{O}((|Y||T|+N)f^2)$ operations. Subsequently, computing the covariance matrix of the aggregated message obtained through \PMP via relative connectivity requires $\mathcal{O}(|Y|^2|T|^2f^2)$ operations. Diagonalizing each of them to create affine transforms requires $\mathcal{O}(|Y||T|f^3)$, and transforming all representations requires $\mathcal{O}(Nf^2)$. Thus, with a total of $K$ layers of topological aggregation, the time complexity for applying \PNY becomes $\mathcal{O}(K(|Y||T|f^3+|Y|^2|T|^2f^2+Nf^2)+|E|f)$. Additionally, the memory complexity includes storing covariance matrices based on relative connectivity and label-time information, which is $\mathcal{O}(|Y||T|f^2 + |Y|^2|T|^2)$.

Now, let's consider applying \PNY to real-world massive graph data. For instance, in the ogbn-mag dataset, $|Y|=349$, $|T|=11$, $N=629571$, and $|E|=21111007$. Assuming a representation dimension of $f=512$, it becomes apparent that performing at least several trillion floating-point operations is necessary. Without approximation or transformations, applying \PNY to large graphs becomes challenging in terms of scalability.

Lastly, for \JJnorm, computing the sample mean of aggregated messages for each label and time pair requires $\mathcal{O}(Nf)$ operations. Based on this, computing the total variance, variance of the mean, and mean of representations with each time requires $\mathcal{O}(Nf)$ operations. Calculating each $\hat\alpha_t$ requires $O(|T|)$ operations, and modifying the aggregated message based on this requires $\mathcal{O}(Nf)$ operations, resulting in a total of $\mathcal{O}(Nf+|T|) \simeq \mathcal{O}(Nf)$ operations. For GNNs with nonlinear node-wise semantic aggregation function with a total of $K$ layers, layer-wise \JJnorm have to be applied, which results in $\mathcal{O}(NfK)$ operations. Additionally, the memory complexity becomes $\mathcal{O}(|Y||T|f)$. Considering that most operations in \JJnorm can be parallelized, it exhibits excellent scalability.

In experiments with synthetic graphs, it was shown that invariant message passing methods can be applied to general spatial GNNs, not just decoupled GNNs. For 1st moment alignment methods such as \PMP and \MMP, which can be applied by reconstructing the graph, they have the same time and memory complexity as calculated above. However, for 2nd moment alignment methods such as \JJnorm or \PNY, transformation is required for each message passing step, resulting in a time complexity multiplied by the number of epochs as calculated above. Therefore, when using general spatial GNNs on real-world graphs, only 1st moment alignment methods may be realistically applicable.

\textbf{Guidelines for deciding which \IMPaCT method to use.} Based on these findings, we propose guidelines for deciding which invariant message passing method to use. If the graph exhibits differences in environments due to temporal information, we recommend starting with \PMP to make the representation's 1st moment invariant during training. \MMP is generally not recommended. Next, if using Decoupled GNNs, \PNY and \JJnorm should be compared. If the graph is too large to apply \PNY, compare the results of using \PMP alone with using both \PMP and \JJnorm. In cases where there are no nonlinear operations in the message passing stage, \JJnorm needs to be applied only once at the end. Using 2nd moment alignment methods with General Spatial GNNs may be challenging unless scalability is improved.

Caution is warranted when applying invariant message passing methods to real-world data. If Assumptions do not hold or if the semantic aggregation functions between layers exhibit loose Lipschitz continuity, the differences in the distribution of final representations over time cannot be ignored. Therefore, rather than relying on a single method, exploring various combinations of the proposed invariant message passing methods to find the best-performing approach is recommended.






\clearpage
\section*{NeurIPS Paper Checklist}
\input{99-Checklist}
\end{document}